\documentclass{article}
\usepackage{epsf}\usepackage{here}
\usepackage{graphicx}
\oddsidemargin 0.25in \evensidemargin 0.25in
\topmargin 0.0in
\textwidth 6.5in \textheight 8.5in
\headheight 0.18in \footskip 0.16in
\leftmargin -0.5in \rightmargin -0.5in

%
% KEYWORD
%
\newcommand{\keywordtable}[1]{
        \sloppy
        \hyphenation{ca-pac-i-t-an-ce}
        \begin{center}
    \sf
        \begin{tabular}[t]
        {|p{0.58in}|p{3.07in}|p{0.55in}|p{0.60in}|}
        \hline
        \multicolumn{1}{|c}{\bf Name} &
        \multicolumn{1}{|c}{\parbox{2.77in}{\bf Description}}  &
        \multicolumn{1}{|c}{\bf Units} &
        \multicolumn{1}{|c|}{\bf Default} \X
        #1
        \end{tabular}
        \end{center}
    }

\newcommand{\keywordtwotable}[2]{
        \sloppy
        \hyphenation{ca-pac-i-t-an-ce}
        \begin{center}
    \sf
        \begin{tabular}[t]
        {|p{0.58in}|p{2.38in}|p{0.55in}|p{0.60in}|p{0.53in}|}
        \hline
        \multicolumn{1}{|c}{\bf Name} &
        \multicolumn{1}{|c}{\parbox{2.20in}{\bf Description}}  &
        \multicolumn{1}{|c}{\bf Units} &
        \multicolumn{1}{|c}{\bf Default} &
        \multicolumn{1}{|c|}{\bf #1} \X
        #2
        \end{tabular}
        \end{center}
    }

\newcommand{\kw}[2]{
     \samepage{
     \noindent {\sl #1} \vspace{-0.5in} \\
     \keywordtable{#2} }}

\newcommand{\kwtwo}[3]{
     \samepage{
     \noindent {\sl #1} \vspace{-0.4in} \\
     \keywordtwotable{#2}{#3} }}

\newcommand{\keyword}[1]{\kw{Keywords:}{#1}}
\newcommand{\keywordtwo}[2]{\kwtwo{Keywords:}{#1}{#2}}
\newcommand{\modelkeyword}[1]{\kw{Model Keywords}{#1}}
\newcommand{\modelkeywordtwo}[2]{\kwtwo{Model Keywords}{#1}{#2}}

\newcommand{\myline}{\\[-0.1in]
\noindent \rule{\textwidth}{0.01in} \newline}

\newcommand{\myThickLine}{\\[-0.1in]
\noindent \rule{\textwidth}{0.02in} \newline}


% FORM
\newcommand{\form}[1]{\samepage{\noindent
 {\sl Form} \myline
% \hspace*{\fill} % For some reason \fill = 0 when \pspiceform{} is used?
\offset
\it  \offsetparbox{#1}}
\\[0.1in]}

% ELEMENT FORM
\newcommand{\elementform}[1]{\samepage{\noindent
 {\sl Element Form} \myline
% \hspace*{\fill} % For some reason \fill = 0 when \pspiceform{} is used?
\offset
\it  \offsetparbox{#1}}
\\[0.1in]}

% MODEL FORM
\newcommand{\modelform}[1]{\samepage{\noindent
 {\sl Model Form} \myline
% \hspace*{\fill} % For some reason \fill = 0 when \pspiceform{} is used?
\offset
\it  \offsetparbox{#1}}
\\[0.1in]}

% LIMITS
\newcommand{\mylimits}[1]{\samepage{\noindent
 {\sl Limits} \myline
 \hspace*{\fill} \it  \offsetparbox{#1}}
 \vshift}

% EXAMPLE
\newcommand{\example}[1]{\samepage{\noindent
{\sl Example} \myline
\offset \tt  \offsetparbox{#1}}
 \vshift}

% PSPICE88 EXAMPLE
\newcommand{\pspiceexample}[1]{\samepage{\noindent
{\sl \pspice\ Example} \myline
\offset \tt  \offsetparbox{#1}}
 \vshift}

% MODEL TYPES
\newcommand{\modeltype}[1]{\samepage{\noindent
{\sl Model Type} \myline
 \hspace*{\fill} \tt \offsetparbox{#1}}
 \\[0.1in]}

% MODEL TYPES
\newcommand{\modeltypes}[1]{\samepage{\noindent
{\sl Model Types:} \myline
 \hspace*{\fill} \tt \offsetparbox{\tt #1}}
 \vshift}

% OFFSET ENUMERATE
\newcommand{\offsetenumerate}[1]{
     \offset \hspace*{-0.1in} {\begin{enumerate} #1 \end{enumerate}}}

% NOTE
\newcommand{\note}[1]{
\vshift\samepage{\noindent {\sl Note}\myline\vspace{-0.24in}}
 \offsetenumerate{#1} }

% SPECIAL NOTE
\newcommand{\specialnote}[2]{
\vshift\samepage{\noindent {\sl #1}\myline\vspace{-0.24in}}\\#2}

\newcommand{\dc}{\mbox{\tt DC}}
\newcommand{\ac}{\mbox{\tt AC}}
\newcommand{\SPICE}{\mbox{\tt SPICE}}
\newcommand{\m}[1]{{\bf #1}}                           % matrix command  \m{}

% ////// Changing nodes to terminals///////
% print terminals in \tt and enclose in a circle use outside
\newcommand{\terminal}[1]{\: \mbox{\tt #1} \!\!\!\! \bigcirc }
%
% set up environment for example
%
\newcounter{excount}
\newcounter{dummy}
\newenvironment{eg}{\vspace{0.1in}\noindent\rule{\textwidth}{.5mm}
   \begin{list}
   {{\addtocounter{excount}{1}
   \em Example\/ \arabic{chapter}.\arabic{excount}\/}:}
   {\usecounter{dummy}
   \setlength{\rightmargin}{\leftmargin}}
   }{\end{list} \rule{\textwidth}{.5mm}\vspace{0.1in}}
%
% set up environment for block
% currently this draws a horizontal line at the start of block and another
% at the end of block.
%
\newenvironment{block}{\vspace{0.1in}\noindent\rule{\textwidth}{.5mm}
   }{\rule{\textwidth}{.5mm}\vspace{0.1in}}
%


%
% set up wide descriptive list
%
\newenvironment{widelist}
    {\begin{list}{}{\setlength{\rightmargin}{0in} \setlength{\itemsep}{0.1in}
    \setlength{\labelwidth}{0.95in} \setlength{\labelsep}{0.1in}
\setlength{\listparindent}{0in} \setlength{\parsep}{0in}
    \setlength{\leftmargin}{1.0in}}
    }{\end{list}}

\newcommand{\STAR}{\hspace*{\fill} * \hspace*{\fill}}

\newcommand{\sym}[1]{\hspace*{\fill} ($#1$)}

\newcommand{\optionitem}[2]{
\item[{\tt #1}{#2}]\label{.OPTION#1}\index{.OPTIONS, #1}\index{#1}}

\newcommand{\error}[1]{\vspace{0.1in}\noindent{\tt #1}\\}


\begin{document}
\noindent{\LARGE \textbf{Voltage-Controlled Voltage Source}
\hspace{\fill}\textbf{vcvs}}\\
\hrulefill \linethickness{0.5mm}\line(1,0){425}
\normalsize
\newline
% the resistor figure
\begin{figure}[h]
\centerline{\epsfxsize=2.5in\epsfbox{figures/e_spice.ps}}
\caption{Voltage-controlled voltage source element.}
\end{figure}
\newline
% form for \FDA
\linethickness{0.5mm} \line(1,0){425}
\newline
\textit{Form:}
\newline
$\tt e$:$\langle \tt{instance\ name}\rangle$ $n_1\ n_2\ \cdots$
$\langle \tt{parameter\ list}\rangle$
\newline
\begin{tabular}{r l}
$n_1$, $n_2\ \cdots$ & are the element nodes. \\
\end{tabular}
% Parameter list
\newline
\textit{Parameters:}
\begin{table}[H]
\begin{tabular}{|c|c|c|c|}
\hline
Parameter&Type&Default value&Required?\\
\hline
k: gain & DOUBLE & 1 & no\\
\hline
ri: Input resistance value(Ohms) & DOUBLE & 0 & no\\
\hline
ro: Output resistance value(Ohms) & DOUBLE & 0 & no\\
\hline $\mathrm{poly_{coeff}}$: Coefficients of polynomial &
DOUBLE VECTOR
& See source file. & no\\
\hline
polydimension: Dimension of polynomial & INTEGER & 1 & no\\
\par
\hline
\end{tabular}
\end{table}
% example in \FDA
\noindent\linethickness{0.5mm} \line(1,0){425}
\newline
\textit{Example (when called in spice mode):}
\newline
\texttt{E1\ 5\ 0\ POLY(1)\ 3\ 2\ 1\ 2.5}
\newline
\linethickness{0.5mm} \line(1,0){425}
\newline
\textit{Description:}\\
The voltage controlled voltage source is either a linear or
nonlinear function of controlling node voltages, depending on
whether the polynomial is used or not.\\
\textit{\underline{Polynomial Functions}:}\\
The controlled element statement allows the definition of the
controlled voltage source as a polynomial function of one or more
voltages. Three polynomial equations can be used through the
POLY(N) parameter. POLY(1) one-dimensional equation, POLY(2)
two-dimensional equation, POLY(3) three-dimensional equation. The
POLY(1) polynomial equation specifies a polynomial equation as a
function of one controlling variable, POLY(2) as a function of two
controlling variables, and POLY(3) as a function of three
controlling variables. Along with each polynomial equation are
polynomial coefficient parameters ($P_0, P_1 \cdots P_n$) that can
be set to explicitly define the equation.\\
\textit{\underline{One-Dimensional Function}:}\\
If the function is one-dimensional (a function of one node
voltage), the function value FV is determined by the following
expression:
\begin{equation}
FV = P_0 + (P_1.FA) + (P_2.{FA}^2) + (P_3.{FA}^3)+ (P_4.{FA}^4) +
(P_5.{FA}^5) + \cdots
\end{equation}
\begin{tabular}{r l}
FV & controlled voltage from the controlled source, \\
&  \\
$P_0 \cdots P_n$ & coefficients of polynomial equation, \\
&  \\
FA & controlling nodal voltage.  \\
\end{tabular}
\newline
If the polynomial is one-dimensional and exactly one coefficient
is specified, \FDA assumes it to be $P_1 (P_0 = 0.0)$ to
facilitate the input of linear controlled sources.\\
\textit{\underline{One-Dimensional Example}:}\\
The example given above is a one-dimensional function. The above
voltage-controlled voltage source is connected to nodes 5 and 0.
The single dimension polynomial function parameter, POLY(1), means
that E1 is a function of the difference of one nodal voltage pair,
in this the voltage difference between nodes 3 and 2, hence
$FA=V(3,2)$. The dependent source statement then specifies that
P0=1 and P1=2.5. From the one-dimensional polynomial equation
above, the defining equation for V(5,0) is $V(5,0) = 1 +
2.5*V(3,2)$.\\
\textit{\underline{Two-Dimensional Function}:}\\
Where the function is two-dimensional (a function of two node
voltages), FV is determined by the following expression:
\begin{eqnarray}
FV &=& P_0 + (P_1.FA) + (P_2.FB) + (P_3.{FA}^2) + (P_4.FA.FB) + \nonumber\\
& &(P_5.{FB}^2) + (P_6.{FA}^3) + (P_7.{FA}^2.FB) + (P_8.FA.{FB}^2)
+(P_9.{FB}^3)+ \cdots \nonumber\\
\end{eqnarray}
For a two-dimensional polynomial, the controlled voltage source is
a function of two nodal voltages. To specify a two-dimensional
polynomial, set POLY(2) in the controlled source statement.\\
\textit{\underline{Three-Dimensional Function}:}\\
For a three-dimensional polynomial function with arguments FA, FB,
and FC, the function value FV is determined by the following
expression:
\begin{eqnarray}
FV &= &P_0 + (P_1.FA) + (P_2.FB) + (P_3.FC) + (P_4.{FA}^2) +\nonumber\\
& &(P_5.FA.FB) + (P_6.FA.FC) + (P_7.{FB}^2) + (P_8.FB.FC) +
(P_9.{FC}^2)
+ \nonumber\\
& &(P_{10}.{FA}^3) + (P_{11}.{FA}^2.FB) + (P_{12}.{FA}^2.FC) + (P_{13}.FA.{FB}^2) + \nonumber\\
& &(P_{14}.FA.FB.FC) + (P_{15}.FA.{FC}^2) + (P_{16}.{FB}^3) + (P_{17}.{FB}^2.FC) + \nonumber\\
& &(P_{18}.FB.{FC}^2) + (P_{19}.{FC}^3) + (P_{20}.{FA}^4) + \cdots \nonumber\\
\end{eqnarray}
\newline
\linethickness{0.5mm} \line(1,0){425}
\newline
\textit{Notes:}\\
This is the \texttt{E} element in the SPICE compatible netlist.\\
\linethickness{0.5mm} \line(1,0){425}
\newline
\textit{Version:}\\
2002.05.01 \\
% Credits
\linethickness{0.5mm} \line(1,0){425}
\newline
\textit{Credits:}\\
\begin{tabular}{l l l l}
Name & Affiliation & Date & Links \\
Satish Uppathil & NC State University & May 2002 & \epsfxsize=1in\epsfbox{figures/logo.eps} \\
svuppath@eos.ncsu.edu & & & www.ncsu.edu    \\
\end{tabular}
\end{document}
