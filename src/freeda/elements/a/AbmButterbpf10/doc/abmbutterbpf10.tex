%This is a model template for any element that you write.
\documentclass{article}
\usepackage{epsf}\usepackage{here}
\usepackage{graphicx}
\oddsidemargin 0.25in \evensidemargin 0.25in
\topmargin 0.0in
\textwidth 6.5in \textheight 8.5in
\headheight 0.18in \footskip 0.16in
\leftmargin -0.5in \rightmargin -0.5in

%
% KEYWORD
%
\newcommand{\keywordtable}[1]{
        \sloppy
        \hyphenation{ca-pac-i-t-an-ce}
        \begin{center}
    \sf
        \begin{tabular}[t]
        {|p{0.58in}|p{3.07in}|p{0.55in}|p{0.60in}|}
        \hline
        \multicolumn{1}{|c}{\bf Name} &
        \multicolumn{1}{|c}{\parbox{2.77in}{\bf Description}}  &
        \multicolumn{1}{|c}{\bf Units} &
        \multicolumn{1}{|c|}{\bf Default} \X
        #1
        \end{tabular}
        \end{center}
    }

\newcommand{\keywordtwotable}[2]{
        \sloppy
        \hyphenation{ca-pac-i-t-an-ce}
        \begin{center}
    \sf
        \begin{tabular}[t]
        {|p{0.58in}|p{2.38in}|p{0.55in}|p{0.60in}|p{0.53in}|}
        \hline
        \multicolumn{1}{|c}{\bf Name} &
        \multicolumn{1}{|c}{\parbox{2.20in}{\bf Description}}  &
        \multicolumn{1}{|c}{\bf Units} &
        \multicolumn{1}{|c}{\bf Default} &
        \multicolumn{1}{|c|}{\bf #1} \X
        #2
        \end{tabular}
        \end{center}
    }

\newcommand{\kw}[2]{
     \samepage{
     \noindent {\sl #1} \vspace{-0.5in} \\
     \keywordtable{#2} }}

\newcommand{\kwtwo}[3]{
     \samepage{
     \noindent {\sl #1} \vspace{-0.4in} \\
     \keywordtwotable{#2}{#3} }}

\newcommand{\keyword}[1]{\kw{Keywords:}{#1}}
\newcommand{\keywordtwo}[2]{\kwtwo{Keywords:}{#1}{#2}}
\newcommand{\modelkeyword}[1]{\kw{Model Keywords}{#1}}
\newcommand{\modelkeywordtwo}[2]{\kwtwo{Model Keywords}{#1}{#2}}

\newcommand{\myline}{\\[-0.1in]
\noindent \rule{\textwidth}{0.01in} \newline}

\newcommand{\myThickLine}{\\[-0.1in]
\noindent \rule{\textwidth}{0.02in} \newline}


% FORM
\newcommand{\form}[1]{\samepage{\noindent
 {\sl Form} \myline
% \hspace*{\fill} % For some reason \fill = 0 when \pspiceform{} is used?
\offset
\it  \offsetparbox{#1}}
\\[0.1in]}

% ELEMENT FORM
\newcommand{\elementform}[1]{\samepage{\noindent
 {\sl Element Form} \myline
% \hspace*{\fill} % For some reason \fill = 0 when \pspiceform{} is used?
\offset
\it  \offsetparbox{#1}}
\\[0.1in]}

% MODEL FORM
\newcommand{\modelform}[1]{\samepage{\noindent
 {\sl Model Form} \myline
% \hspace*{\fill} % For some reason \fill = 0 when \pspiceform{} is used?
\offset
\it  \offsetparbox{#1}}
\\[0.1in]}

% LIMITS
\newcommand{\mylimits}[1]{\samepage{\noindent
 {\sl Limits} \myline
 \hspace*{\fill} \it  \offsetparbox{#1}}
 \vshift}

% EXAMPLE
\newcommand{\example}[1]{\samepage{\noindent
{\sl Example} \myline
\offset \tt  \offsetparbox{#1}}
 \vshift}

% PSPICE88 EXAMPLE
\newcommand{\pspiceexample}[1]{\samepage{\noindent
{\sl \pspice\ Example} \myline
\offset \tt  \offsetparbox{#1}}
 \vshift}

% MODEL TYPES
\newcommand{\modeltype}[1]{\samepage{\noindent
{\sl Model Type} \myline
 \hspace*{\fill} \tt \offsetparbox{#1}}
 \\[0.1in]}

% MODEL TYPES
\newcommand{\modeltypes}[1]{\samepage{\noindent
{\sl Model Types:} \myline
 \hspace*{\fill} \tt \offsetparbox{\tt #1}}
 \vshift}

% OFFSET ENUMERATE
\newcommand{\offsetenumerate}[1]{
     \offset \hspace*{-0.1in} {\begin{enumerate} #1 \end{enumerate}}}

% NOTE
\newcommand{\note}[1]{
\vshift\samepage{\noindent {\sl Note}\myline\vspace{-0.24in}}
 \offsetenumerate{#1} }

% SPECIAL NOTE
\newcommand{\specialnote}[2]{
\vshift\samepage{\noindent {\sl #1}\myline\vspace{-0.24in}}\\#2}

\newcommand{\dc}{\mbox{\tt DC}}
\newcommand{\ac}{\mbox{\tt AC}}
\newcommand{\SPICE}{\mbox{\tt SPICE}}
\newcommand{\m}[1]{{\bf #1}}                           % matrix command  \m{}

% ////// Changing nodes to terminals///////
% print terminals in \tt and enclose in a circle use outside
\newcommand{\terminal}[1]{\: \mbox{\tt #1} \!\!\!\! \bigcirc }
%
% set up environment for example
%
\newcounter{excount}
\newcounter{dummy}
\newenvironment{eg}{\vspace{0.1in}\noindent\rule{\textwidth}{.5mm}
   \begin{list}
   {{\addtocounter{excount}{1}
   \em Example\/ \arabic{chapter}.\arabic{excount}\/}:}
   {\usecounter{dummy}
   \setlength{\rightmargin}{\leftmargin}}
   }{\end{list} \rule{\textwidth}{.5mm}\vspace{0.1in}}
%
% set up environment for block
% currently this draws a horizontal line at the start of block and another
% at the end of block.
%
\newenvironment{block}{\vspace{0.1in}\noindent\rule{\textwidth}{.5mm}
   }{\rule{\textwidth}{.5mm}\vspace{0.1in}}
%


%
% set up wide descriptive list
%
\newenvironment{widelist}
    {\begin{list}{}{\setlength{\rightmargin}{0in} \setlength{\itemsep}{0.1in}
    \setlength{\labelwidth}{0.95in} \setlength{\labelsep}{0.1in}
\setlength{\listparindent}{0in} \setlength{\parsep}{0in}
    \setlength{\leftmargin}{1.0in}}
    }{\end{list}}

\newcommand{\STAR}{\hspace*{\fill} * \hspace*{\fill}}

\newcommand{\sym}[1]{\hspace*{\fill} ($#1$)}

\newcommand{\optionitem}[2]{
\item[{\tt #1}{#2}]\label{.OPTION#1}\index{.OPTIONS, #1}\index{#1}}

\newcommand{\error}[1]{\vspace{0.1in}\noindent{\tt #1}\\}


\begin{document}
\noindent{\LARGE \textbf{abmButterbpf10}%name of your element and the following is the
%element's initial for referencing
\hspace{\fill}\textbf{abmbutterbpf10}}\\
\hrulefill \linethickness{0.5mm}\line(1,0){425}
\normalsize
\newline
% the filter figure
\begin{figure}[h]
\centerline{\epsfxsize=2.5in \epsfbox{butter10_fig.eps}}
\caption{Butterbpf10 -- Butterworth bandpass filter, tenth-order.}
%The figure of the element and its caption comes here
\end{figure}
\newline
% form for \FDA
\linethickness{0.5mm}\line(1,0){425}
\newline %the way it is written in the netlist
\textit{Form:} $\tt abmbutterbpf10$:$\langle \tt{instance\
name}\rangle$ $n_1\ n_2\ n_3\ n_4\ $ $\langle \tt{parameter\
list}\rangle$
\newline
\begin{tabular}{r l}
$n_1$ & is the positive terminal of port 1, \\
&  \\
$n_2$ & is the negative terminal of port 1, \\
&  \\
$n_3$ & is the positive terminal of port 2, \\
&  \\
$n_4$ & is the negative terminal of port 2. \\
%&  \\
%parameter list & see table 1 for parameter list
\end{tabular}
%Parameter list
\newline
\textit{Parameters:}%list of parameters that you have in tabular form
\begin{table}[H]
\begin{tabular}{|c|c|c|c|}
\hline
Parameter&Type&Default value&Required?\\
\hline
f0: Filter center frequency (Hertz) & DOUBLE & N/A & yes \\
\hline
bw: Filter 3 dB bandwidth (Hertz) & DOUBLE & N/A & yes \\
\par
\hline
\end{tabular}
\end{table}
%example in \FDA %FDA stands for fREEDA
\noindent\linethickness{0.5mm}\line(1,0){425}
\newline
\textit{Example:}%example of the way in which your element is mentioned in the netlist
\newline
\texttt{abmButterbpf10:f1\ 11\ 0\ 20\ 0\ f0=1e3 bw=200}
\newline
\linethickness{0.5mm} \line(1,0){425}
\newline
\textit{Notes:}\\ %any additional notes that you might like to add
This element is a tenth-order Butterworth bandpass filter.  The
transfer function is in the form of a transconductance so that the
output of the filter is a current at port two, controlled by the
voltage at port one.  The transfer function is expressed as sum of
rational fractions of the following form:
\begin{equation} H_i = \frac{a_is^3 + b_is^2 + c_is}{s^4 + d_is^3 + e_is^2 +
f_is + h_i}
\end{equation}
The coefficients of each of the five rational fractions in the
implementation of this filter are computed at object
initialization from the user-defined center frequency and
bandwidth.

The element works in both Harmonic Balance and AC analysis
simulations for any frequency, but due to limitations in the
linear solver package, the time domain MNAM becomes unfactorable
when the center frequency is set above 50 kHz.  Also, when
performing simulations in the time domain, the trapezoidal
integration method does not converge for this element so the
backward Euler method must be used.
\linethickness{0.5mm} \line(1,0){425}
\newline
\textit{Version:} 2003.05.15\\
% Credits - authors of this element, their e-mail addresses and affiliation
\linethickness{0.5mm} \line(1,0){425}
\newline
\textit{Credits:}\\
\begin{tabular}{l l l l}
Name & Affiliation & Date & Links \\
Aaron L. Walker & NC State University & April 2003 & \epsfxsize=1in\epsfbox{logo.eps}  \\
alwalke3@ncsu.edu & & & www.ncsu.edu    \\
\end{tabular}
\end{document}
