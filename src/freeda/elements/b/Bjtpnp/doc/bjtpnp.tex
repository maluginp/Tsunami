\documentclass{article}
\usepackage{epsf}\usepackage{here}
\usepackage{graphicx}
\oddsidemargin 0.25in \evensidemargin 0.25in
\topmargin 0.0in
\textwidth 6.5in \textheight 8.5in
\headheight 0.18in \footskip 0.16in
\leftmargin -0.5in \rightmargin -0.5in

%
% KEYWORD
%
\newcommand{\keywordtable}[1]{
        \sloppy
        \hyphenation{ca-pac-i-t-an-ce}
        \begin{center}
    \sf
        \begin{tabular}[t]
        {|p{0.58in}|p{3.07in}|p{0.55in}|p{0.60in}|}
        \hline
        \multicolumn{1}{|c}{\bf Name} &
        \multicolumn{1}{|c}{\parbox{2.77in}{\bf Description}}  &
        \multicolumn{1}{|c}{\bf Units} &
        \multicolumn{1}{|c|}{\bf Default} \X
        #1
        \end{tabular}
        \end{center}
    }

\newcommand{\keywordtwotable}[2]{
        \sloppy
        \hyphenation{ca-pac-i-t-an-ce}
        \begin{center}
    \sf
        \begin{tabular}[t]
        {|p{0.58in}|p{2.38in}|p{0.55in}|p{0.60in}|p{0.53in}|}
        \hline
        \multicolumn{1}{|c}{\bf Name} &
        \multicolumn{1}{|c}{\parbox{2.20in}{\bf Description}}  &
        \multicolumn{1}{|c}{\bf Units} &
        \multicolumn{1}{|c}{\bf Default} &
        \multicolumn{1}{|c|}{\bf #1} \X
        #2
        \end{tabular}
        \end{center}
    }

\newcommand{\kw}[2]{
     \samepage{
     \noindent {\sl #1} \vspace{-0.5in} \\
     \keywordtable{#2} }}

\newcommand{\kwtwo}[3]{
     \samepage{
     \noindent {\sl #1} \vspace{-0.4in} \\
     \keywordtwotable{#2}{#3} }}

\newcommand{\keyword}[1]{\kw{Keywords:}{#1}}
\newcommand{\keywordtwo}[2]{\kwtwo{Keywords:}{#1}{#2}}
\newcommand{\modelkeyword}[1]{\kw{Model Keywords}{#1}}
\newcommand{\modelkeywordtwo}[2]{\kwtwo{Model Keywords}{#1}{#2}}

\newcommand{\myline}{\\[-0.1in]
\noindent \rule{\textwidth}{0.01in} \newline}

\newcommand{\myThickLine}{\\[-0.1in]
\noindent \rule{\textwidth}{0.02in} \newline}


% FORM
\newcommand{\form}[1]{\samepage{\noindent
 {\sl Form} \myline
% \hspace*{\fill} % For some reason \fill = 0 when \pspiceform{} is used?
\offset
\it  \offsetparbox{#1}}
\\[0.1in]}

% ELEMENT FORM
\newcommand{\elementform}[1]{\samepage{\noindent
 {\sl Element Form} \myline
% \hspace*{\fill} % For some reason \fill = 0 when \pspiceform{} is used?
\offset
\it  \offsetparbox{#1}}
\\[0.1in]}

% MODEL FORM
\newcommand{\modelform}[1]{\samepage{\noindent
 {\sl Model Form} \myline
% \hspace*{\fill} % For some reason \fill = 0 when \pspiceform{} is used?
\offset
\it  \offsetparbox{#1}}
\\[0.1in]}

% LIMITS
\newcommand{\mylimits}[1]{\samepage{\noindent
 {\sl Limits} \myline
 \hspace*{\fill} \it  \offsetparbox{#1}}
 \vshift}

% EXAMPLE
\newcommand{\example}[1]{\samepage{\noindent
{\sl Example} \myline
\offset \tt  \offsetparbox{#1}}
 \vshift}

% PSPICE88 EXAMPLE
\newcommand{\pspiceexample}[1]{\samepage{\noindent
{\sl \pspice\ Example} \myline
\offset \tt  \offsetparbox{#1}}
 \vshift}

% MODEL TYPES
\newcommand{\modeltype}[1]{\samepage{\noindent
{\sl Model Type} \myline
 \hspace*{\fill} \tt \offsetparbox{#1}}
 \\[0.1in]}

% MODEL TYPES
\newcommand{\modeltypes}[1]{\samepage{\noindent
{\sl Model Types:} \myline
 \hspace*{\fill} \tt \offsetparbox{\tt #1}}
 \vshift}

% OFFSET ENUMERATE
\newcommand{\offsetenumerate}[1]{
     \offset \hspace*{-0.1in} {\begin{enumerate} #1 \end{enumerate}}}

% NOTE
\newcommand{\note}[1]{
\vshift\samepage{\noindent {\sl Note}\myline\vspace{-0.24in}}
 \offsetenumerate{#1} }

% SPECIAL NOTE
\newcommand{\specialnote}[2]{
\vshift\samepage{\noindent {\sl #1}\myline\vspace{-0.24in}}\\#2}

\newcommand{\dc}{\mbox{\tt DC}}
\newcommand{\ac}{\mbox{\tt AC}}
\newcommand{\SPICE}{\mbox{\tt SPICE}}
\newcommand{\m}[1]{{\bf #1}}                           % matrix command  \m{}

% ////// Changing nodes to terminals///////
% print terminals in \tt and enclose in a circle use outside
\newcommand{\terminal}[1]{\: \mbox{\tt #1} \!\!\!\! \bigcirc }
%
% set up environment for example
%
\newcounter{excount}
\newcounter{dummy}
\newenvironment{eg}{\vspace{0.1in}\noindent\rule{\textwidth}{.5mm}
   \begin{list}
   {{\addtocounter{excount}{1}
   \em Example\/ \arabic{chapter}.\arabic{excount}\/}:}
   {\usecounter{dummy}
   \setlength{\rightmargin}{\leftmargin}}
   }{\end{list} \rule{\textwidth}{.5mm}\vspace{0.1in}}
%
% set up environment for block
% currently this draws a horizontal line at the start of block and another
% at the end of block.
%
\newenvironment{block}{\vspace{0.1in}\noindent\rule{\textwidth}{.5mm}
   }{\rule{\textwidth}{.5mm}\vspace{0.1in}}
%


%
% set up wide descriptive list
%
\newenvironment{widelist}
    {\begin{list}{}{\setlength{\rightmargin}{0in} \setlength{\itemsep}{0.1in}
    \setlength{\labelwidth}{0.95in} \setlength{\labelsep}{0.1in}
\setlength{\listparindent}{0in} \setlength{\parsep}{0in}
    \setlength{\leftmargin}{1.0in}}
    }{\end{list}}

\newcommand{\STAR}{\hspace*{\fill} * \hspace*{\fill}}

\newcommand{\sym}[1]{\hspace*{\fill} ($#1$)}

\newcommand{\optionitem}[2]{
\item[{\tt #1}{#2}]\label{.OPTION#1}\index{.OPTIONS, #1}\index{#1}}

\newcommand{\error}[1]{\vspace{0.1in}\noindent{\tt #1}\\}


\begin{document}
\noindent{\LARGE \textbf{Bipolar Junction Transistor}
\hspace{\fill}\textbf{bjtpnp}}
\hrulefill\linethickness{0.5mm}\line(1,0){425} \normalsize
\newline
% the bjt figure
\begin{figure}[h]
\centerline{\epsfxsize=4in\pfig{q1.eps}} \caption[Q --- bipolar
junction transistor element]{Q --- Bipolar Polar Junction
Transistor: (a) NPN transistor; (b) PNP transistor.}
\end{figure}
\newline
% form for Transim
\linethickness{0.5mm}\line(1,0){425}
\newline
\textit{\FDA Form:}
%\newline
$\tt bjtpnp$:$\langle \tt{instance\ name}\rangle$ $n_1\ n_2\ n_3\
$ $\langle \tt{parameter\ list}\rangle$
\newline
% explanation of the parameters
\begin{tabular}{r l}
$n_1$ & is the base node \\
&  \\
$n_2$ & is the collector node \\
&  \\
$n_3$ & is the emitter node \\
& \\
%parameter list & see table 1 for parameter list
\end{tabular}
% SPICE form
\newline
\noindent\texttt{SPICE} \textit{Form:}
%\newline
%\linethickness{0.5mm}
%\line(1,0){425}
\newline
{\tt Q}name  NCollector NBase NEmitter  [NSubstrate]  ModelName
[Area] [{\tt OFF}] {\tt} [{\tt IC=}Vbe,Vce] \\
where \\
\begin{tabular}{r l}
{\it NCollector} & is the collector node. \\
& \\
{\it NBase} & is the base node. \\
& \\
{\it NEmitter} & is the emitter node. \\
& \\
{\it NSubstrate} & is the optional substrate node. If not specified, then the ground is used as the \\
& substrate node. If {\it NSubstrate} is a name as allowed in it must be enclosed in \\
& square brackets, e.g. {\tt [{\it NSubstrate}]}, to distinguish it from {\it ModelName}. \\
& \\
{\it ModelName} & is  the  model  name. \\
& \\
{\it Area} & is  the  area  factor. If the area  factor  is  omitted,  a  value of 1.0 is assumed. \\
& (Units: none; Optional; Default: 1; Symbol: $Area$) \\
& \\
{\tt OFF} & indicates an (optional) initial condition on the device for the dc\ analysis. If \\
& specified the dc\ operating point is calculated with the terminal voltages set to \\
& zero. Once convergence is obtained, the program continues to iterate to obtain \\
& the exact value of the  terminal  voltages.  The OFF option is used to enforce \\
& the solution to  correspond to  a  desired state if the circuit has \\
& more than one stable state. \\
& \\
{\tt IC} & is the optional initial condition specification using  {\tt IC=}$V_{BE},V_{CE}$ is  intended \\
& for use with the {\tt UIC} option on the {\tt .TRAN} line, when a transient analysis is \\
& desired  starting  from  other than  the  quiescent  operating  point. See  the {\it .IC} \\
& line description for a better way to set transient initial conditions.
\end{tabular}
\newpage
%\vspace{4mm}
% example in SPICE
%\newline
\noindent\linethickness{0.5mm}\line(1,0){425}
\newline
\noindent\textit{Example:}
\newline
%\newline
bjtpnp:Q20 10 50 0\\
bjtpnp:QFAST IC=0.65,15.0 \\
bjtpnp:Q5PUSH 10 29 14 200 MODEL1
% Parameter table
%\newpage
\newline
\linethickness{0.5mm}\line(1,0){425}

\noindent\textit{Model Parameters:}\\
\newline
\begin{tabular}{|r|l|c|c|} \hline
\textbf{Name} & \textbf{Description} & \textbf{Units} & \textbf{Default} \\
\hline
\texttt{AREA} & Current multiplier & & 1.0 \\
\hline
\texttt{BF} & Ideal maximum forward beta ($B_F$) & & 100.0 \\
\hline
\texttt{BR} & Ideal maximum reverse beta ($B_R$) & & 1.0  \\
\hline
\texttt{C2} & Base-emitter leakage saturation coefficient & & $I_{SE} / I_S$ \\
\hline
\texttt{C4} & Base-collector leakage saturation coefficient & & ($I_{SC} / I_S$) \\
\hline
\texttt{CJC} & Base collector zero bias p-n capacitance ($C_{JC}$) & F &  0.0\\
\hline
\texttt{CJE} & Base emitter zero bias p-n capacitance ($C_{JE}$) & F & 0.0\\
\hline
\texttt{EG} & Bandgap voltage ($E_G$) & eV & 1.11 \\
\hline
\texttt{FC} & Forward bias depletion capacitor coefficient ($F_C$) & & 0.5 \\
\hline
\texttt{IKF} & Corner of forward beta high-current roll-off ($I_{KF}$) & A & $10^{-10}$ \\
\hline
\texttt{IKR} & Corner for reverse-beta high current roll off ($I_{KR}$) & & $10^{-10}$ \\
\hline
\texttt{IS} & Transport saturation current ($I_S$) & A & $10^{-16}$  \\
\hline
\texttt{ISC} & Base collector leakage saturation current ($I_{SC}$) & A & 0.0 \\
\hline
\texttt{ISE} & Base-emitter leakage saturation current ($I_{SE}$) & A & 0.0\\
\hline
\texttt{IRB} & Current at which \texttt{RB} falls to half of \texttt{$R_{BM}$} ($I_{RB}$) & A & $10^{-10}$ \\
\hline
\texttt{ITF} & Transit time dependency on \texttt{IC} ($I_{TF}$) & A & 0.0 \\
\hline
\texttt{MJC} & Base collector p-n grading factor ($M_{JC}$) & & 0.33 \\
\hline
\texttt{MJE} & Base emitter p-n grading factor ($M_{JE}$) & & 0.33 \\
\hline
\texttt{NC} & Base-collector leakage emission coefficient ($N_C$) & & 2.0\\
\hline
\texttt{NE} & Base-emitter leakage emission coefficient ($N_E$) &  & 1.5 \\
\hline
\texttt{NF} & Forward current emission coefficient ($N_F$) & & 1.0 \\
\hline
\texttt{NR} & Reverse current emission coefficient ($N_R$) & & 1.0 \\
\hline
\texttt{RB} & Zero bias base resistance ($R_B$) & $\Omega$ & 0.0 \\
\hline
\texttt{RBM} & Minimum base resistance ($R_{BM}$) & $\Omega$ & $R_B$ \\
\hline
\texttt{RE} & Emitter ohmic resistance ($R_E$) & $\Omega$ & 0.0 \\
\hline
\texttt{RC} & Collector ohmic resistance ($R_C$) & $\Omega$ &  0.0 \\
\hline
\texttt{T} & Operating Temperature $T$& K &  300 \\
\hline
\texttt{TF} & Ideal forward transit time ($T_S$) & secs & 0.0 \\
\hline
\texttt{TNOM} & Nominal temperature ($T_{NOM}$) & K & 300 \\
\hline
\texttt{TR} & Ideal reverse transit time ($T_R$) & S & 0.0 \\
\hline
\texttt{TRB1} & \texttt{RB} temperature coefficient (linear) ($T_{RB1}$) & & 0.0 \\
\hline
\texttt{TRB2} & \texttt{RB} temperature coefficient  (quadratic) ($T_{RB2}$) & & 0.0 \\
\hline
\texttt{TRC1} & \texttt{RC} temperature coefficient (linear) ($T_{RC1}$) & & 0.0 \\
\hline
\texttt{TRC2} & \texttt{RC} temperature coefficient (linear) ($T_{RC2}$) & & 0.0 \\
\hline
\texttt{TRE1} & \texttt{RE} temperature coefficient (linear) ($T_{RE1}$) & & 0.0 \\
\hline
\texttt{TRE2} & \texttt{RE} temperature coefficient (quadratic) ($T_{RE2}$) & & 0.0 \\
\hline
\texttt{TRM1} & \texttt{RBM} temperature coefficient (linear) ($T_{RM1}$) & & 0.0 \\
\hline
\texttt{TRM2} & \texttt{RBM} temperature coefficient (quadratic) ($T_{RM2}$) & & 0.0 \\
\hline
\end{tabular}

\begin{tabular}{|r|l|c|c|}
\hline
\textbf{Name} & \textbf{Description} & \textbf{Units} & \textbf{Default} \\
\hline
\texttt{VA} & alternative keyword for \texttt{VAF} ($V_A$) & V & $10^{-10}$ \\
\hline
\texttt{VAF} & Forward early voltage ($V_{AF}$) & V & $10^{-10}$ \\
\hline
\texttt{VAR} & Reverse early voltage ($V_{AR}$) & & $10^{-10}$ \\
\hline
\texttt{VB} & alternative keyword for \texttt{VAR} ($V_B$) & & $10^{-10}$ \\
\hline
\texttt{VJC} & Base collector built in potential ($V_{JC}$) & V & 0.75\\
\hline
\texttt{VJE} & Base emitter built in potential ($V_{JE}$) & V & 0.75 \\
\hline
\texttt{VTF} & Transit time dependency on \texttt{VBC} ($V_{TF}$) & V & $10^{-10}$ \\
\hline
\texttt{XCJC} & Fraction of \texttt{CBC} connected internal to \texttt{RB} ($X_{CJC}$) & & 1.0 \\
\hline
\texttt{XTB} & Forward and reverse beta temperature coefficient ($X_{TB}$) & & 0.0\\
\hline
\texttt{XTF} & Transit time bias dependence coefficient ($X_{TF}$) & & 0.0 \\
\hline
\texttt{XTI} & \texttt{IS} temperature effect exponent ($X_{TI}$) & & 3.0 \\
\hline
\end{tabular}
\newline
%\vspace{6mm}
\newpage
%\linethickness{0.5mm}
\textbf{ELEMENT Model}
\newline
%\line(1,0){425}
%\vspace{2mm}
%\newline
\begin{figure}[h]
\centerline{\epsfxsize=4in\pfig{bipolar1.eps}}
\caption{Schematic
of the BJT Model}
\end{figure}
\newline
%\vspace{5mm}
% Device equations start
\newpage
\noindent\underline{\large Standard Calculations}\\

The physical constants used in the model evaluation are
\begin{center}
\begin{tabular}{|l|l|l|}
\hline
$k$ & Boltzman's constant &  $1.3806226\,10^{-23}$~J/K\\
$q$ & electronic charge & $1.6021918\,10^{-19}$~C\\
\hline
\end{tabular}
\end{center}
Absolute temperatures (in kelvins, K) are used.
The thermal voltage
\begin{equation}
V_{TH}(T_{NOM}) = {{k\ T_{NOM}} / q} .
\end{equation}
\newline
%\vspace{6mm}
% Current charactersitics
\noindent\underline{Current Characteristics}\\

\noindent{The base-emitter current, }
\begin{equation}
I_{BE} = {{\textstyle I_{BF} } / {\textstyle \beta_F }} + I_{LE}
\end{equation}

\noindent{the base-collector current, }
\begin{equation}
I_{BC} = {{\textstyle I_{BR} } / {\textstyle \beta_R }} + I_{LC}
\end{equation}

\noindent{and the collector-emitter current,}
\begin{equation}
I_{CE} = {{\textstyle I_{BF} - I_{BR} } / {\textstyle K_{QB} }}
\end{equation}

\noindent{where the forward diffusion current,}
\begin{equation}
I_{BF} = I_S\left(e^{\textstyle V_{BE}/(N_F V_{TH}) - 1} \right)
\end{equation}

\noindent{the nonideal base-emitter current,}
\begin{equation}
I_{LE}=I_{SE}\left(e^{\textstyle V_{BE}/(N_E V_{TH}) - 1} \right)
\end{equation}

\noindent{the reverse diffusion current,}
\begin{equation}
I_{BR} = I_S\left(e^{\textstyle V_{BC}/(N_R V_{TH}) - 1} \right)
\end{equation}

\noindent{the non-ideal base-collector current,}
\begin{equation}
I_{LC}=I_{SC}\left(e^{\textstyle V_{BC}/(N_C V_{TH}) - 1} \right)
\end{equation}

\noindent{and the base charge factor,}
\begin{equation}
K_{QB} = {{\textstyle 1}/{2}} \left[1 -
{{\textstyle V_{BC}}/{\textstyle V_{AF}}}- {{\textstyle V_{BE}}/{\textstyle V_{AB}}}
    \right]^{-1} \left(1 + \sqrt{1 + 4\left(
        {{\textstyle I_{BF}}/{\textstyle I_{KF}}}+
         {{\textstyle I_{BR}}/{\textstyle I_{KR}}}
        \right)}\right)\\
\end{equation}

\noindent{Thus the conductive current flowing into the base,}
\begin{equation}
I_B = I_{BE}+I_{BC}
\end{equation}

\noindent{the conductive current flowing into the collector,}
\begin{equation}
I_C = I_{CE}-I_{BC}
\end{equation}

\noindent{and the conductive current flowing into the emitter,}
\begin{equation}
I_C = I_{BE}+I_{CE}
\end{equation}

%\vspace{0.1in}
% Capacitances
\noindent\underline{\large Capacitances}\\

$C_{BE} = Area( C_{BE\tau} + C_{BEJ})$
where the base-emitter transit time or diffusion capacitance

\begin{equation}
C_{BE\tau} = \tau_{F,EFF} {{\textstyle\partial I_{BF}} /
       {\textstyle\partial V_{BE}}}
\end{equation}

the effective base transit time is empirically modified to account for base
puchout, space-charge limited current flow, quasi-saturation and lateral
spreading which tend to increase $\tau_F$

\begin{equation}
\tau_{F,EFF} =\tau_F\left[ 1+X_{TF}(3x^2-2x^3)
     e^{\textstyle (V_{BC}/(1.44V_{TF})}\right]
\end{equation}
and $x = {I_{BF}}/(I_{BF} + Area I_{TF})$.


The base-emitter junction (depletion) capacitance
\begin{equation}
C_{BEJ} = \left\{ \!\! \begin{array}{ll}
C_{JE} \left(1-{{\textstyle V_{BE}} / {\textstyle V_{JE}}}\right)^{\textstyle -M_{JE}} & \! V_{BE} \le F_C V_{JE}\\
C_{JE} \left(1-F_C\right)^{\textstyle -(1+M_{JE})} \left(1-F_C(1+M_{JE})+M_{JE}{{\textstyle V_{BE}} / {\textstyle V_{JE}}} \right) & \! V_{BE} > F_C V_{JE}
                         \end{array}
         \right.
\end{equation}


The base-collector capacitance,
$C_{BC} = Area(C_{BC\tau} +X_{CJC} C_{BCJ})$
where the base-collector transit time or diffusion capacitance
\begin{equation}
C_{BC\tau} = \tau_R {{\textstyle\partial I_{BR}} /
       {\textstyle\partial V_{BC}}}
\end{equation}


The base-collector junction (depletion) capacitance
\begin{equation}
C_{BCJ} = \left\{ \! \! \begin{array}{ll}
C_{JC} \left(1-{{\textstyle V_{BC}}/{\textstyle V_{JC}}}\right)^{\textstyle -M_{JC}} & \! V_{BC} \le F_C V_{JC}\\
C_{JC} \left(1-F_C\right)^{\textstyle -(1+M_{JC})}\left(1-F_C(1+M_{JC})+M_{JC}{{\textstyle V_{BC}}/ {\textstyle V_{JC}}} \right) & \! V_{BC} > F_C V_{JC}
                       \end{array}
         \right.
\end{equation}



The capacitance between the extrinsic base and the intrinsic collector
\begin{equation}
C_{BX} = \left\{ \begin{array}{ll}
Area(1-X_{CJC}) C_{JC} \left(1-{{\textstyle V_{BX}} / {\textstyle V_{JC}}}\right)^{\textstyle -M_{JC}} & V_{BX} \le F_C V_{JC}\\ \\
(1-X_{CJC}) C_{JC} \left(1-F_C\right)^{\textstyle -(1+M_{JC})} & V_{BX} > F_C V_{JC}\\
\;\;\;\;\;\times\; \left(1-F_C(1+M_{JC})+M_{JC}{{\textstyle V_{BX}} / {\textstyle V_{JC}}} \right)  &
                \end{array}
        \right.
\end{equation}\\
%\vspace{6mm}
%The substrate junction capacitance
%\begin{equation}
%C_{JS} = \left\{ \begin{array}{ll}
%Area C_{JS} \left(1-{{\textstyle V_{CJS}} / {\textstyle V_{JS}}}\right)^{\textstyle -M_{JS}} & V_{CJS} \le 0\\
%Area C_{JS} \left(1+M_{JS}{{\textstyle V_{CJS}} / {\textstyle V_{JS}}} \right) & V_{CJS} > 0
%                   \end{array}
%       \right.
%\end{equation}

% Bugs
%\newline
\noindent \linethickness{0.5mm}\line(1,0){425}
\newline
\noindent\textit{Bugs:}\\
\newline
Parameters: OFF and IC are not functional.\\
\linethickness{0.5mm} \line(1,0){425}
\newline
\textit{Version:}\\
2002.09.01 \\
% Credits
\newline
\linethickness{0.5mm} \line(1,0){425}
\newline
\noindent\textit{Credits:}
\newline
\newline
\begin{tabular}{l l l l}
Name & Affiliation & Date &  \\
Senthil Velu & North Carolina State University & Sept 2002 & \epsfxsize=1in\pfig{logo.eps} \\
\end{tabular}
%Publications
\newline
\noindent \linethickness{0.5mm} \line(1,0){425}
\newline
\textit{Publications:}
\begin{enumerate}
\item C. Christoffersen, S. Velu and M. B. Steer, `` A Universal Parameterized Nonlinear
Device Model Formulation for Microwave Circuit Simulation,'' 2002
IEEE Int. Microwave Symp. Digest, June 2002, pp 2189-2192.
\end{enumerate}
\end{document}
