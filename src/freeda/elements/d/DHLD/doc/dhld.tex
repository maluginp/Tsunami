\documentclass{article}
\usepackage{epsf}\usepackage{here}
\usepackage{graphicx}
\oddsidemargin 0.25in \evensidemargin 0.25in
\topmargin 0.0in
\textwidth 6.5in \textheight 8.5in
\headheight 0.18in \footskip 0.16in
\leftmargin -0.5in \rightmargin -0.5in

%
% KEYWORD
%
\newcommand{\keywordtable}[1]{
        \sloppy
        \hyphenation{ca-pac-i-t-an-ce}
        \begin{center}
    \sf
        \begin{tabular}[t]
        {|p{0.58in}|p{3.07in}|p{0.55in}|p{0.60in}|}
        \hline
        \multicolumn{1}{|c}{\bf Name} &
        \multicolumn{1}{|c}{\parbox{2.77in}{\bf Description}}  &
        \multicolumn{1}{|c}{\bf Units} &
        \multicolumn{1}{|c|}{\bf Default} \X
        #1
        \end{tabular}
        \end{center}
    }

\newcommand{\keywordtwotable}[2]{
        \sloppy
        \hyphenation{ca-pac-i-t-an-ce}
        \begin{center}
    \sf
        \begin{tabular}[t]
        {|p{0.58in}|p{2.38in}|p{0.55in}|p{0.60in}|p{0.53in}|}
        \hline
        \multicolumn{1}{|c}{\bf Name} &
        \multicolumn{1}{|c}{\parbox{2.20in}{\bf Description}}  &
        \multicolumn{1}{|c}{\bf Units} &
        \multicolumn{1}{|c}{\bf Default} &
        \multicolumn{1}{|c|}{\bf #1} \X
        #2
        \end{tabular}
        \end{center}
    }

\newcommand{\kw}[2]{
     \samepage{
     \noindent {\sl #1} \vspace{-0.5in} \\
     \keywordtable{#2} }}

\newcommand{\kwtwo}[3]{
     \samepage{
     \noindent {\sl #1} \vspace{-0.4in} \\
     \keywordtwotable{#2}{#3} }}

\newcommand{\keyword}[1]{\kw{Keywords:}{#1}}
\newcommand{\keywordtwo}[2]{\kwtwo{Keywords:}{#1}{#2}}
\newcommand{\modelkeyword}[1]{\kw{Model Keywords}{#1}}
\newcommand{\modelkeywordtwo}[2]{\kwtwo{Model Keywords}{#1}{#2}}

\newcommand{\myline}{\\[-0.1in]
\noindent \rule{\textwidth}{0.01in} \newline}

\newcommand{\myThickLine}{\\[-0.1in]
\noindent \rule{\textwidth}{0.02in} \newline}


% FORM
\newcommand{\form}[1]{\samepage{\noindent
 {\sl Form} \myline
% \hspace*{\fill} % For some reason \fill = 0 when \pspiceform{} is used?
\offset
\it  \offsetparbox{#1}}
\\[0.1in]}

% ELEMENT FORM
\newcommand{\elementform}[1]{\samepage{\noindent
 {\sl Element Form} \myline
% \hspace*{\fill} % For some reason \fill = 0 when \pspiceform{} is used?
\offset
\it  \offsetparbox{#1}}
\\[0.1in]}

% MODEL FORM
\newcommand{\modelform}[1]{\samepage{\noindent
 {\sl Model Form} \myline
% \hspace*{\fill} % For some reason \fill = 0 when \pspiceform{} is used?
\offset
\it  \offsetparbox{#1}}
\\[0.1in]}

% LIMITS
\newcommand{\mylimits}[1]{\samepage{\noindent
 {\sl Limits} \myline
 \hspace*{\fill} \it  \offsetparbox{#1}}
 \vshift}

% EXAMPLE
\newcommand{\example}[1]{\samepage{\noindent
{\sl Example} \myline
\offset \tt  \offsetparbox{#1}}
 \vshift}

% PSPICE88 EXAMPLE
\newcommand{\pspiceexample}[1]{\samepage{\noindent
{\sl \pspice\ Example} \myline
\offset \tt  \offsetparbox{#1}}
 \vshift}

% MODEL TYPES
\newcommand{\modeltype}[1]{\samepage{\noindent
{\sl Model Type} \myline
 \hspace*{\fill} \tt \offsetparbox{#1}}
 \\[0.1in]}

% MODEL TYPES
\newcommand{\modeltypes}[1]{\samepage{\noindent
{\sl Model Types:} \myline
 \hspace*{\fill} \tt \offsetparbox{\tt #1}}
 \vshift}

% OFFSET ENUMERATE
\newcommand{\offsetenumerate}[1]{
     \offset \hspace*{-0.1in} {\begin{enumerate} #1 \end{enumerate}}}

% NOTE
\newcommand{\note}[1]{
\vshift\samepage{\noindent {\sl Note}\myline\vspace{-0.24in}}
 \offsetenumerate{#1} }

% SPECIAL NOTE
\newcommand{\specialnote}[2]{
\vshift\samepage{\noindent {\sl #1}\myline\vspace{-0.24in}}\\#2}

\newcommand{\dc}{\mbox{\tt DC}}
\newcommand{\ac}{\mbox{\tt AC}}
\newcommand{\SPICE}{\mbox{\tt SPICE}}
\newcommand{\m}[1]{{\bf #1}}                           % matrix command  \m{}

% ////// Changing nodes to terminals///////
% print terminals in \tt and enclose in a circle use outside
\newcommand{\terminal}[1]{\: \mbox{\tt #1} \!\!\!\! \bigcirc }
%
% set up environment for example
%
\newcounter{excount}
\newcounter{dummy}
\newenvironment{eg}{\vspace{0.1in}\noindent\rule{\textwidth}{.5mm}
   \begin{list}
   {{\addtocounter{excount}{1}
   \em Example\/ \arabic{chapter}.\arabic{excount}\/}:}
   {\usecounter{dummy}
   \setlength{\rightmargin}{\leftmargin}}
   }{\end{list} \rule{\textwidth}{.5mm}\vspace{0.1in}}
%
% set up environment for block
% currently this draws a horizontal line at the start of block and another
% at the end of block.
%
\newenvironment{block}{\vspace{0.1in}\noindent\rule{\textwidth}{.5mm}
   }{\rule{\textwidth}{.5mm}\vspace{0.1in}}
%


%
% set up wide descriptive list
%
\newenvironment{widelist}
    {\begin{list}{}{\setlength{\rightmargin}{0in} \setlength{\itemsep}{0.1in}
    \setlength{\labelwidth}{0.95in} \setlength{\labelsep}{0.1in}
\setlength{\listparindent}{0in} \setlength{\parsep}{0in}
    \setlength{\leftmargin}{1.0in}}
    }{\end{list}}

\newcommand{\STAR}{\hspace*{\fill} * \hspace*{\fill}}

\newcommand{\sym}[1]{\hspace*{\fill} ($#1$)}

\newcommand{\optionitem}[2]{
\item[{\tt #1}{#2}]\label{.OPTION#1}\index{.OPTIONS, #1}\index{#1}}

\newcommand{\error}[1]{\vspace{0.1in}\noindent{\tt #1}\\}


\begin{document}
\noindent{\LARGE \textbf{Double Heterojunction Laser Diode (Tucker Model)}\newline
\hspace*{\fill}\textbf{DHLD}}\\
\hrulefill\linethickness{0.5mm}\line(1,0){425} \normalsize
\newline

\begin{figure}[H]
\centerline{\epsfxsize=1.5in\epsfbox{figures/dhld_circuit.eps}} \caption{DHLD --- Double Heterojunction Laser Diode (Tucker model).}
\end{figure}

\noindent\linethickness{0.5mm}\line(1,0){425}
\newline
\textit{fREEDA Form:}
$\tt DHLD $:$\langle \tt{instance\ name}\rangle$ $n_1\ n_2\
n_3\ n_3\ $ $\langle \tt{parameter\ list}\rangle$
\newline
\begin{tabular}{r l}
$n_1$ & is the electrical anode terminal \\
&  \\
$n_2$ & is the electrical cathode or reference electrical \\
&  \\
$n_3$ & is the optical terminal \\
& \\
$n_3$ & is the optical reference terminal \\
& \\
parameter list & see table for parameter list
\end{tabular}\\



\noindent\textbf{Parameter Table}\\[0.1in]

\begin{tabular}{|l| l| l| l|}
\hline
 \textbf{Parameters} & \textbf{Description} & \textbf{Values} & \textbf{Units} \\
 \hline
 {\tt $R_s$} &series resistance & 2 & $\Omega$ \\
 {\tt $R_e$} &equivalent resistance due to carrier degeneracy & 0.468 & $\Omega$ \\
 {\tt $I_{01}$} &equivalent Diode1 leakage current & 2.54e-25& A \\
 {\tt $I_{02}$} &equivalent Diode2 leakage current & 18.13e-3& A \\
 {\tt $b$} &current controlled current source gain & 6.92& A$^{-1}$ \\
 {\tt $\tau_{ns}$} &equivalent recombination Lifetime & 2.25e-9& s \\
 {\tt $C_0$} &diode zero-bias charge capacitance & 10e-12& F \\
 {\tt $V_D$} &junction built-in potential & 1.65& V \\
 {\tt $D$} &constant relating the radiative recombination & & \\
         &current per unit volume to the optical gain& 1.79e-29&
          V$^{-1}$A$^{-1}$m$^{6}$ \\
 {\tt $a$} &fraction of equivalent recombination lifetime& & \\
         &over low-level injection spontaneous& & \\
         &recombination lifetime& 0.125& -\\
 {\tt $R_p$} &equivalent optical resistor& 29.4& $\Omega$\\
 {\tt $C_p$} &equivalent optical capacitor& 0.102e-12& F\\
 {\tt $S_c$} &photon density normalization constant& 1e21& m$^{-3}$\\
 {\tt $\beta$} &fraction of spontaneous emission coupled& & \\
             &into the lasing mode& 1e-3& -\\
\hline
\end{tabular}

\newpage


\begin{figure}[H]
\centerline{\epsfxsize=5in \epsfbox{figures/dh_structure.eps}}
\caption{Double-heterostructure laser. The \textit{p-GaAs} active
layer is usually less than 0.5 $\mu$m thick.
After~\cite{casey:hetero:1978, coldren:diode:1995}.}
\label{fig:dhld:dhs}
\end{figure}

\noindent The DHLD consists of a \textit{p}-type GaAs
active layer of thickness \textit{d} sandwiched between
\textit{n}-type and \textit{p}-type layers of higher bangap
material as shown in Fig.~\ref{fig:dhld:dhs}. The circuit model
for the DHLD is shown in Fig.~\ref{fig:dhld:cet}. It is similar in
many ways to the structure described
in~\cite{joyce:electrical:1978}. The laser diode model is based on
the Tucker large-signal circuit model~\cite{tucker:circuit:1981,
tucker:large:1981}. It is derived from the physics of the
heterojunction and explicitly takes into account the effect of
carrier degeneracy, high level injection, and nonradiative
recombination. The modulation response is determined through the
rate equations of the device's electro-optical dynamics. The
model is described in Chapter 3 of Reference \cite{kanj:thesis}.

\vspace*{0.1in}
\noindent\textbf{Analysis}\newline
\noindent Under the assumption that the thickness \textit{d} of
the active layer is small compared to the carrier diffusion length
and that the variation of carrier densities with position in the
active layer is small enough~\cite{tucker:circuit:1981}, the
carrier densities can be represented by average values. Then the
average total electron density $N$ in the active layer is given by
\begin{equation}
N=N_0+n
\end{equation}
where $N_0$ is the equilibrium electron density and $n$ is the
excess electron density. Corresponding notation can be used for
hole densities.

From the physics of the Heterostructure1
lasers~\cite{joyce:electrical:1978, tucker:circuit:1981}, and
under the above assumptions, the total radiative spontaneous
recombination rate $R$ in the active layer is given by:
\begin{equation}
R=BNP
\end{equation}
where $B$ is a constant and $P$ is the average total hole density
in the active layer~\cite{tucker:circuit:1981}. To obtain the
diode current due to spontaneous radiative recombination, we
define the excess spontaneous radiative recombination rate $r_e$
as:
\begin{equation}
r_e=n/\tau_s + B_1n^2
\end{equation}
where $\tau_s$ is the low-level injection spontaneous
recombination lifetime and $B_1$ is a constant defined
in~\cite{tucker:circuit:1981}.

Also, a significant contribution to the diode current arises from
nonradiative recombination rate $r_n$ along the strip edges and at
the heterointerfaces. Following the analysis
in~\cite{joyce:electrical:1978}, it is assumed here that the
nonradiative recombination rate is proportional to $n$, and is
characterized by a lifetime $\tau_n$. Then the total excess
recombination rate $r_t$ (including radiative and nonradiative
components) is obtained by adding the nonradiative recombination
rate $r_n = n/\tau_n$ to $r_e$:
\begin{equation}
\label{eqn:dhld:etrr}
r_t=(1/\tau_n + 1/\tau_s)n + B_1n^2~.
\end{equation}


Again, since the active layer is assumed to be thin, the diode
recombination current is obtained by multiplying $r_t$ by the
active layer volume $v_a$ and the electron charge $q$. Adding to
it the displacement-current term, we obtain the diode terminal
current below threshold $I$:
\begin{equation}
\label{eqn:dhld:dtc}
I = q v_a\left(r_t + \frac{dn}{dt}\right)~.
\end{equation}
Note that Eqn.~\ref{eqn:dhld:dtc} does not include the effects of
space-charge capacitance. Substituting Eqn.~\ref{eqn:dhld:etrr} in
Eqn.~\ref{eqn:dhld:dtc} gives:
\begin{equation}
I = I_1 + bI_1^2 + \tau_{ns} \frac{dI_1}{dt}
\end{equation}
where
\begin{equation}
\label{eqn:dhld:i1}
I_1 = \frac{q v_a n}{\tau_{ns}}
\end{equation}
\begin{equation}
b = \frac{B_1 \tau_{ns}^2}{q v_a}
\end{equation}
and
\begin{equation}
\tau_{ns} = (\tau_s^{-1} + \tau_n^{-1})^{-1}
\end{equation}

\vspace*{0.1in}
\noindent\textbf{Current/Voltage characteristics}\newline
\noindent The diode junction voltage $V_j$
can be expressed in terms of the electron density in the active
layer as a three series-connected voltage drops $V_1, V_2$, and
$V_3$. That is:
\begin{equation}
V_j=V_1+V_2+V_3
\end{equation}
and
\begin{eqnarray}
\label{eqn:dhld:v1}
V_1=V_T \ln(1 + n/N_0) \\
\label{eqn:dhld:v2}
V_2=V_T \ln\{1 + n/(N_A+N_0)\} \\
\label{eqn:dhld:v3}
V_3=V_T(\alpha_1 + \alpha_3)n
\end{eqnarray}
where $V_T=kT/q$ is the thermal voltage, $N_A$ is the acceptor
impurity concentration, and $\alpha_1$ and $\alpha_2$ are constant
defined in~\cite{tucker:circuit:1981}. The first two of these
elements represent a classical Shockley \textit{p-n} junction
diodes. With Eqn.~\ref{eqn:dhld:i1} substituted,
Eqns.~\ref{eqn:dhld:v1} and~\ref{eqn:dhld:v2} become:
\begin{equation}
\label{eqn:dhld:i1exp}
I_1=I_{01}\{\exp(V_1/V_T)-1\}
\end{equation}
and
\begin{equation}
I_1=I_{02}\{\exp(V_2/V_T)-1\}
\end{equation}
where $I_1$ is the current through the two diodes and the two
diode leakage currents are given by:
\begin{equation}
I_{01}=q v_a N_0/\tau_{ns}
\end{equation}
and
\begin{equation}
I_{02}=q v_a (N_A+N_0)/\tau_{ns}
\end{equation}
The third series-connected element is given by
Eqn.~\ref{eqn:dhld:v3}. Substituting Eqn.~\ref{eqn:dhld:i1} in
Eqn.~\ref{eqn:dhld:v3} gives
\begin{equation}
I_1=V_3/R_e
\end{equation}
where
\begin{equation}
R_e=(\alpha_1+\alpha_3) N_0 V_T/I_{01}~.
\end{equation}

\vspace*{0.1in}
\noindent\textbf{Rate Equations}\newline

\noindent As mentioned earlier, the excess spontaneous
recombination rate per unit volume $r_t$ can be written as the sum
of two components $r_n$ and $r_e$:
\begin{equation}
r_t = r_n + r_e
\end{equation}
Then, the total diode current due to spontaneous recombination is
$I_t = q v_a r_t$, which can be written in the form:
\begin{equation}
\label{eqn:dhld:it}
I_t = I_1 + b I_1^2
\end{equation}
and the diode current due to radiative spontaneous recombination
is $I_e = q v_a r_e$, which reduces to:
\begin{equation}
\label{eqn:dhld:ie}
I_e = a I_1 + b I_1^2
\end{equation}
where
\begin{equation}
a = \tau_{ns}/\tau_s
\end{equation}

The single mode rate equations for an injection
laser~\cite{tucker:large:1981} can be written in the form
\begin{equation}
\label{eqn:dhld:ren}
qv_a\frac{dn}{dt}=I - I_t - qv_agS
\end{equation}
\begin{equation}
\label{eqn:dhld:res}
qv_a\frac{dS}{dt}=qv_agS - \frac{Sqv_a}{\tau_p}+\beta I_e
\end{equation}
where $I$ is the diode terminal current, $g$ is the optical gain,
$S$ is the photon density in the active layer, $\tau_p$ is the
photon lifetime, and $\beta$ is the fraction of spontaneous emission
coupled into the lasing mode. Eqn.~\ref{eqn:dhld:ren} describes
electron-injection and charge-storage effects in the active layer,
and Eqn.~\ref{eqn:dhld:res} describes the corresponding injection
and storage dynamics of photons. These equations form the basis of
the equivalent large signal model. To account for the space-charge
storage in the heterojunction layer, Eqn.~\ref{eqn:dhld:ren} is
generalized to include space-charge capacitance term. Note that
this effect is taken into account by a capacitor $C_s$ and is
different from the charge-storage effect taken into account by the
term $\tau_{ns}dI_1/dt$. Also, a normalized photon density $S_n$
is introduced to obtain better numerical values. With the above
modifications, and substituting
Eqns.~\ref{eqn:dhld:it} and~\ref{eqn:dhld:ie}, the rate equations
become:
\begin{equation}
\label{eqn:dhld:rei}
I=I_1 + bI^2_1 + \tau_{ns}\frac{dI_1}{dt} + C_s\frac{dVj}{dt} + GS_n
\end{equation}
\begin{equation}
\label{eqn:dhld:resn}
GS_n+\beta(aI_1 + bI^2_1)=\frac{S_n}{R_p}+C_p\frac{dS_n}{dt}
\end{equation}
where $C_s=C_0(1-V_j/V_D)^{-1/2}$ is the space-charge capacitance,
$V_J$ is the heterojunction voltage, $C_0$ is the zero bias
space-charge capacitance, $V_D$ is the diode built-in potential,
$C_P=qv_aS_c$, $G=gC_p$, $R_p=\tau_p/C_p$, and $S_n=S/S_c$, where
$S_c$ is the photon-density normalization constant.

\vspace*{0.1in}
\noindent\textbf{Equivalent Large-signal Circuit Model}\newline

\noindent The large-signal circuit model of the injection laser follows from
the rate equations, Eqns.~\ref{eqn:dhld:rei} and~\ref{eqn:dhld:resn}, and
from the current/voltage characteristics of the diode. The electrical equivalent model is shown
to the left of the vertical broken line in
Fig.~\ref{fig:dhld:cet}. It is important to note that the
resistance $R_e$ in series with the two Shockley diodes arises
from carrier degeneracy, and is not associated with the ohmic
regions of the diode. Those regions are modelled by a series
resistance $R_s$ which includes contributions from lead
resistance, bulk resistance in the high-bandgap materials, and the
effective resistance of the near-ohmic \textit{p-P} isotype
heterojunction.
\begin{figure}[ht]
\centerline{\epsfxsize=5in \epsfbox{figures/circuit_equi_tucker.eps}}
\caption{Large-signal two port circuit model of injection laser}
\label{fig:dhld:cet}
\end{figure}
The resulting equivalent circuit model of the photon dynamics is
shown to the right of the vertical broken line in
Fig.~\ref{fig:dhld:cet} and is derived from
Eqn.~\ref{eqn:dhld:resn}.
where
\begin{equation}
I_{sp}=\beta(aI_1^2+bI_1^2)
\end{equation}
and
\begin{equation}
I_g=GS_n~.
\end{equation}

\vspace*{0.1in}
\noindent\textbf{Diode Parameters}\newline
\noindent The laser parameters used in the simulations are the
same as the ones used
in~\cite{tucker:circuit:1981,tucker:large:1981}, and are similar
to the parameters used in~\cite{joyce:electrical:1978}. The excess
electron density in the active layer is assumed to be
$n_s=1.5\times10^{18}$~cm$^{-3}$, and the factor $qv_a$ is taken
as $1.02\times10^{-25}$~mA~cm$^{3}~s$. Thus the threshold current
$I_t$ is approximately $100$~mA. The active layer doping density
is taken as $N_A=4\times10^{17}$~cm$^{-3}$, and the photon
lifetime is $\tau_p=3.0$~ps.

It is also assumed that the optical gain function $G$ has a
square-law dependence on the radiative recombination current per
unit volume $J_{nom}$ as described in~\cite{casey:hetero:1978}
\begin{equation}
\label{eqn:dhld:g}
G=D(J_{nom}-2\times10^{13})^2
\end{equation}
where $D$ is a constant and $J_{nom}=I_e/v_a$~A/m$^3$.A numerical
value of $D$ can be obtained by first determining $S_{n0}$, the
steady-state normalized photon density, and then setting it to
infinity at saturation, that is when $n=n_s$. The steady-state
photon density is obtained by subsrituting $dS_n/dt=0$ in
Eqn.~\ref{eqn:dhld:resn}
\begin{equation}
S_{n0}=\frac{\beta(aI_{10}+bI_{10}^2)}{1/R_p-G}
\end{equation}
where $I_{10}$ is the steady-state value of $I_1$. As we can see,
$S_{n0}$ goes to infinity when $G=1/R_p$ which, when substituted
in Eqn.~\ref{eqn:dhld:g}, yields
\begin{equation}
\label{eqn:dhld:d}
D=R_p^{-1}(J_{noms}-2\times10^{13})^{-2}
\end{equation}
where $J_{noms}$ is the value of $J_{nom}$ at saturation, and is given by
\begin{equation}
\label{eqn:dhld:js}
J_{noms}=\frac{qn_s}{\tau_{ns}}(a+b\frac{qv_an_s}{\tau_{ns}})
\end{equation}
with the known diode parameters substituted in
Eqns.~\ref{eqn:dhld:js} and~\ref{eqn:dhld:g}, we obtain
$J_{noms}=6.359\times10^{13}$ and
$D=1.79\times10^{-29}$~V$^{-1}$~A$^{-1}$~m$^{6}$. Numerical values of
other parameters of the circuit model are listed in
Table~\ref{table:p:dh}.\\[0.1in]

\vspace*{0.1in}
\noindent\textbf{Implementation}\newline

\noindent The key to the implementation of the model is to
consider the voltage on one of the diodes in
Fig.~\ref{fig:dhld:cet} as the first state variable, $V_1$ for
example, and the normalized photon density $S_n$ as the second
state variable, and then write the model equations as a function
of these two state variables and there derivatives, \textit{i.e.}
$dV_1/dt$ and $dS_n/dt$ .

The relation between the drive input voltage $V$ and current $I$
is given by:
\begin{equation}
\label{eqn:dhld:v}
V=IR_s+V_j
\end{equation}
where $I$ can be expressed as:
\begin{equation}
\label{eqn:dhld:i}
I=I_1+bI_1^2+\tau_{ns}\frac{dI_1}{dt}+C_s\frac{dV_j}{dt}+I_g
\end{equation}
To write Eqns.~\ref{eqn:dhld:v} and~\ref{eqn:dhld:i} in terms of
the state variables and there derivatives, we need to find $I_1$,
$dI_1/dt$, $V_j$, and $dV_j/dt$ in terms of these state variables.
From Eqn.~\ref{eqn:dhld:i1exp}, we know that
$I_1=I_{01}\{\exp{(V_1/V_T)}-1\}$, then
\begin{equation}
\label{eqn:dhld:di1dt}
\frac{dI_1}{dt}=\frac{I_{01}}{V_T}\exp(V_1/V_T)\frac{dV_1}{dt}
\end{equation}
and $V_j$ can be expressed as:
\begin{equation}
\label{eqn:dhld:vj}
V_j=V_1+V_2+V_3
\end{equation}
where
\begin{equation}
\label{eqn:dhld:v2ln}
V_2=V_Tln(I_1/I_{02}+1)
\end{equation}
and
\begin{equation}
\label{eqn:dhld:v3i1re}
V_3=I_1R_e~\cdot
\end{equation}
We still have to write $dV_j/dt$ as a function of the state
variables and their derivatives:
\begin{equation}
\frac{dV_j}{dt}=\frac{dV_1}{dt}+\frac{dV_2}{dt}+\frac{dV_3}{dt}
\end{equation}
where from Eqn.~\ref{eqn:dhld:v2ln}, we have
\begin{equation}
\frac{dV_2}{dt}=\frac{I_{01}}{I_{02}}\exp\{\frac{(V_1-V_2)}{V_T}\}\frac{dV_1}{dt}
\end{equation}
and from Eqn.~\ref{eqn:dhld:v3i1re}, we have
\begin{equation}
\frac{dV_3}{dt}=R_e\frac{dI_1}{dt}~.
\end{equation}
Finally, we have to find $I_g$ as a function of the state
variables. We know that $I_g=GS_n=D(J_{nom}-2\times10^{13})^2S_n$,
and that $J_{nom}=I_e/v_a=(aI_1+bI_1^2)/v_a$, then
\begin{equation}
I_g=D(\frac{aI_1+bI_1^2}{v_a}-2\times10^{13})^2S_n~.
\end{equation}
Now that we have expressed the current $I$ and voltage $V$ at the
electrical port of the diode as a function of the state variables,
we have to express the current and voltage at the optical port of
the diode as a function of those variables. The voltage at the
optical port is chosen to be $S_n$, however, the current $I_{sn}$
has no meaning and it is forced to be zero ($I_{sn}\equiv0$) by
connecting an open circuit to the optical port. The model,
however, will not function properly unless the
Eqn.~\ref{eqn:dhld:resn} is satisfied. This is done by using the
fact that $I_{sn}\equiv0$ and by rewriting
Eqn.~\ref{eqn:dhld:resn} in the form\footnote{Another way to
satisfy Eqn.~\ref{eqn:dhld:resn} is to connect a 1~$\Omega$
resistor at the optical port and make use of the fact that
$I_{sn}-S_n\equiv0$, where $I_{sn}=R_pGS_n+R_p\beta(aI_1 +
bI^2_1)-R_pC_p\frac{dS_n}{dt}$. This implementation actually will
not result in a singular matrix in Harmonic-balance simulations
and alleviate the need to use a large resistance instead of the
open circuit in that case.}
\begin{equation}
I_{sn}=GS_n+\beta(aI_1 + bI^2_1) - \frac{S_n}{R_p} - C_p\frac{dS_n}{dt}
\end{equation}

One last thing we did not talk about above is parameterization or
variable transformations. They both refer to an
algebraic transformation of the device equations that leads to a
better convergence properties, and enables universal device
modeling. The parameterization employed here is the one suggested
in~\cite{rizzoli:state:1992} and that converts the strong
nonlinear current-voltage relationship of the diode to two
smoother functions of current and voltage as functions of the
state-variable $x$. Specifically, $V_1$ is not taken as the state
variable in the actual coding, and Eqn.~\ref{eqn:dhld:i1exp} is
parameterized as follows

\begin{equation}
I_1=\cases{I_{01}\{\exp(\alpha x)-1\}  & if $x\leq V_{pr}$ \cr
I_{01}\exp(\alpha V_{pr})\{1+\alpha(x-V_{pr})\} - I_{01} & if
$x>V_{pr}$}
\end{equation}

\begin{equation}
V_1=\cases{x  & \hspace{.58in}if $x\leq V_{pr}$ \cr
V_{pr}+\frac{1}{\alpha}\ln\{1+\alpha(x-V_{pr})\}  & \hspace{.53in} if $x>V_{pr}$}
\end{equation}
where $\alpha=1/V_T$ and $V_{pr}$ plays the role of a free
parameter chosen appropriately to optimize the performance of the
HB algorithm specifically. Experience
shows~\cite{rizzoli:state:1992} that $V_{pr}=\ln(1/\alpha
I_s)/\alpha$ results in excellent behavior of the model in most
practical situations. As shown in Fig.~\ref{fig:dhld:iv},
~\ref{fig:dhld:ix}, and~\ref{fig:dhld:vx}, the strong nonlinearity
between $i$ \& $v$ is converted to moderate nonlinearities between
$i$ \& $x$ and $v$ \& $x$, and the problem becomes well behaved.
Please refer to~\cite{christoffersen:global:2001}
and~\cite{velu:charge:2002} for more information on universal
device modeling, and how the same piece of code is used in
fREEDA with different
simulation algorithms, $i.e.$ HB, Transient, DC analysis, etc.




\begin{figure}[htbp]
\centerline{\epsfxsize=3.8in \epsfbox{figures/diode_iv_big.eps}}
\caption{Relation between $v$ and $i$ in a diode.}
\label{fig:dhld:iv}
\end{figure}
\begin{figure}[htbp]
\centerline{\epsfxsize=3.5in \epsfbox{figures/diode_ix_big.eps}}
\caption{Relation between $x$ and $i$ in a diode.}
\label{fig:dhld:ix}
\end{figure}
\begin{figure}[htbp]
\centerline{\epsfxsize=4.0in \epsfbox{figures/diode_vx_big.eps}}
\caption{Relation between $x$ and $v$ in a diode.}
\label{fig:dhld:vx}
\end{figure}

%The second parameterization~\cite{mena:rate:1997}

\vspace*{0.1in}\noindent\textbf{Results}\newline
\noindent The following sections present the simulation results of
the DHLD model. The diode is driven by an input current pulse of finite
rise and fall time. Graphs of the input terminal voltage and of
the normalized photon density are shown for different values of
$\beta$ and compared with HSPICE$^{\scriptsize\textregistered}$.

A Harmonic Balance analysis is
performed on the implemented DHLD model. First, the model is
driven by a DC bias source and single tone sine wave. Plots of the
input terminal voltage and of the normalized photon density are
shown and compared with
fREEDA's transient
analysis and HSPICE$^{\scriptsize\textregistered}$. Second, the
model is driven by a DC bias source and two tone input sine waves.
Plots of the optical output power spectrum is presented. Also, the
power ratio of the second harmonic to the fundamental $P_{2f}/P_f$
and of the intermodulation distortion to the fundamental
$P_{IM3}/P_f$ as a function of bias current are shown.

\vspace*{0.1in}\noindent\textbf{Transient Analysis}\newline

\noindent In the analysis and design of laser diode transmitter,
it is very important to determine laser turn-on delay and other
switching and modulation characteristics especially for high-speed
application where the switching waveform is affect by the finite
bandwidth of the drive circuits~\cite{tucker:circuit:1981}. This
is why transient simulation is very important in the design of
optoelectronic ICs.

The DHLD model is driven by a current pulse that has a peak value
of 150 mA and a rise time of 0.1 ns and the simulations
are presented for different values of $\beta$. The plots in
Fig.~\ref{fig:dhld:vpulse} shows the input terminal voltage versus
time while the plots in Fig.~\ref{fig:dhld:snpulse} shows the
normalized output photon density. As we can see, a very small
change in the input voltage correspond to a large ringing effect
in the output power and this is due to exponential current/voltage
relationship of the diode. Also, Fig.~\ref{fig:dhld:snpulse} shows
the laser turn-on delay.

As shown in all of the plots, there is excellent agreement
between fREEDA and
HSPICE$^{\scriptsize\textregistered}$.

\begin{figure}
\centerline{\epsfxsize=4.5in
\epsfbox{figures/DHLD_v_pulse.eps}}
\caption{Transient analysis comparison of the terminal voltage}
\label{fig:dhld:vpulse}
\end{figure}

\begin{figure}
\centerline{\epsfxsize=4.5in \epsfbox{figures/DHLD_sn_pulse.eps}}
\caption{Transient Analysis comparison of the light output}
\label{fig:dhld:snpulse}
\end{figure}
\vspace*{0.1in}
\noindent\textbf{Harmonic Balance}\newline

\begin{figure}
\centerline{\epsfxsize=6in \epsfbox{figures/LD_Tucker_parasitics.eps}}
\caption{Parasitics and matching network used in HB simulation.
After~\cite{way:large:1987}.}
\label{fig:dhld:ltp}
\end{figure}

\noindent Fiber-optic microwave links have the potential to be
used in a large number of applications such as cable television
systems and personal communication systems.
That is why it is important to characterize the behavior
of the laser diode under direct microwave intensity modulation,
and one of the most important tools in the simulations of
nonlinear models at microwave frequencies is Harmonic Balance.

The laser diode was connected to the parasitics and matching
network as shown in Fig.~\ref{fig:dhld:ltp} and harmonic balance
simulations with a single and two tone sine wave input were
performed.

\begin{figure}
\centerline{\epsfxsize=4.3in \epsfbox{figures/DHLD_v_beta_1e_3_hb_tr.eps}}
\caption{Comparison of the input terminal voltage between HB
analysis and transient analysis in fREEDA.}
\label{fig:dhld:dhvhbtr}
\end{figure}

\begin{figure}
\centerline{\epsfxsize=4.3in \epsfbox{figures/DHLD_sn_beta_1e_3_hb_tr.eps}}
\caption{Comparison of the output photon density between HB
analysis and transient analysis in fREEDA.}
\label{fig:dhld:dhsnhbtr}
\end{figure}

\begin{figure}
\centerline{\epsfxsize=4.5in \epsfbox{figures/LS_Int_Mod.eps}}
\caption{Large-signal intensity modulation response.}
\label{fig:dhld:dhintmod}
\end{figure}

The intensity modulation response of a double heterojunction laser
diode to an rf-input input power of 7 dBm at 1 GHz at a bias
current of 125 mA was simulated. The time domain results are shown in
Figs.~\ref{fig:dhld:dhvhbtr} and~\ref{fig:dhld:dhsnhbtr}
and compared to transient analysis. The calculated optical output
power spectrum is shown in Fig.~\ref{fig:dhld:dhintmod}, with the
second harmonic being approximately �7.59 dBc.


\begin{figure}
\centerline{\epsfxsize=4.5in \epsfbox{figures/P2f_Pf.eps}}
\caption{Power ratio of second harmonic to fundamental as a
function of bias current.}
\label{fig:dhld:dhp2f}
\end{figure}

Fig.~\ref{fig:dhld:dhp2f} shows the power ratio of the second
harmonic to the fundamental $P_{2f}/P_f$ as a function of the bias
current for an rf-input input power of 3 dBm at input frequency of
1 GHz. The threshold current of this device is 100 mA.


\begin{figure}
\centerline{\epsfxsize=4.5in \epsfbox{figures/Pim3_Pf.eps}}
\caption{Power ratio of third-order intermodulation products
to carrier as a function of bias current.}
\label{fig:dhld:dhpim3}
\end{figure}

Finally, Fig.~\ref{fig:dhld:dhpim3} shows the power ratio of the
third-order intermodulation products to the carrier $P_{IM3}/P_f$
as a function of the bias current. Equal inputs of �-1 dBm at 1.0
GHz and 1.04 GHz were used. In general, there is an improvement in
linearity with increasing bias current.

As we can see, there is an excellent agreement in the single tone
simulations between HB and transient analysis except at the beginning
with HB which
truncates the transient response. In addition, the two tone
simulations shows a close agreement with the reported nonlinear
distortion simulations in the literature~\cite{way:large:1987,
iezekiel:nonlinear:1990}.

\begin{thebibliography}{99}

\bibitem{kanj:thesis}
H. Kanj, Circuit-Level Modeling of Laser Diodes , M.S., North Carolina State University, 2003

\bibitem{Bewtra:modeling:1995} N. Bewtra, D. A. Suda, G. L. Tan,
F. Chatenoud, and J. M. Xu, ``Modeling of Quantum-Well Lasers with
Electro-Opto-Thermal Interaction,'' {\em IEEE J. of Selected
Topics in Quantum Electronics}, Vol. 1, No. 2, pp. 331-340, June
1995.

\bibitem{morikuni:simulation:1998} J. Morikuni, G. Dare, P. Mena,
A. Harton, K. Wyatt, ``Simulation of optical interconnect devices,
circuits, and systems using analog behavioral modeling tools,''
{\em IEEE Lasers and Electro-Optics Society Annual Meeting}, Vol.
1, pp. 235--236, Dec 1998.

\bibitem{freeda:2002} {\em http://www.freeda.org}.

\bibitem{griewank:adol-c:1996} A. Griewank, D. Juedes and J. Utke,
``Adol-C: A Package for the Automatic Differenciation of
Algorithms Written in C/C++'' ACM TOMS, Vol. 22(2), pp. 131�-167,
June 1996.

\bibitem{hecht:laser:1992} J. Hecht,
``Laser Pioneers'' {\em Boston: Academic Press}, 1992.

\bibitem{dupuis:anintro:1987} R. D. Dupuis ``An Introduction to
the Developement of the Semiconductor Laser'' {\em IEEE J. of
Quantum Electronics}, Vol. QE-23, No. 6, June 1987.

\bibitem{morikuni:optoelectronic:1998} J. Morikuni, P. Mena,
A. Harton, K. Wyatt, ``Optoelectronic Computer-Aided Design,''
{\em IEEE Lasers and Electro-Optics Society Summer Topical
Meetings}, pp. 53--54, July 1998.

\bibitem{seung-woo:optical:1999} L. Seung-Woo, C. Eun-Chang, C.
Woo-Young, ``Optical interconnection system analysis using SPICE,
'' {\em The Pacific Rim Conference on Lasers and Electro-Optics},
Vol. 2, pp. 391--392, Sep 1999.

\bibitem{christoffersen:global:2001} C. E. Christoffersen,
{\em Global Modeling of Nonlinear Microwave Circuits}, Ph.D.
Dissertation, Dept. of Electrical Engineering, North Carolina
State University, 2001.

\bibitem{joyce:electrical:1978} W. B. Joyce, R. W. Dixon,
``Electrical characterization of heterostructure lasers,'' {\em in
J. Appl. Phys.}, Vol. 49, No. 7, July 1978.

\bibitem{tucker:circuit:1981} R. S. Tucker, ``Circuit model of
double-heterojunction laser below threshold,'' {\em in IEE Proc.
on Solid-State \& Electron Devices--Part I}, Vol. 128, No. 3, pp.
101--106, 1981.

\bibitem{tucker:large:1981} R. S. Tucker, ``Large-signal circuit
model for simulation of injection-laser modulation dynamics,''
{\em in IEE Proc. on Solid-State \& Electron Devices--Part I},
Vol. 128, No. 5, pp. 180--184, 1981.

\bibitem{casey:hetero:1978} H.C. Casey, M.B. Panish,
``Heterosructure Lasers, part A and part B'' {\em Academic Press},
1978.

\bibitem{coldren:diode:1995} L.A. Coldren, S.W. Corzine,
``Diode Lasers and Photonic Integrated Circuits,''
{\em John Wiley \& Sons, Inc.}, 1995.

\bibitem{rizzoli:state:1992} V. Rizzoli, A. Lipparini, A. Costanzo,
F. Mastri, C. Cecchetti, A. Neri, D. Masotti, ``State-of-the-art
harmonic-balance simulation of forced nonlinear microwave circuits
by the piecewise technique,'' {\em IEEE Transactions on Microwave
Theory and Techniques}, Vol. 40, issue 1, pp. 12--28, Jan. 1992.

\bibitem{velu:charge:2002} S. N. Velu,
{\em Charge Based Modeling in State Variable Based Simulator},
M.S. Thesis, Dept. of Electrical Engineering, North Carolina State
University, 2002.


\bibitem{way:large:1987} W. I. Way, ``Large Signal Nonlinear
Distortion Prediction for a  Single-Mode Laser Diode Under
Microwave Intensity Modulation,'' {\em IEEE J. of Lightwave
Technology}, Vol. LT-5, No. 3, March 1987.

\bibitem{iezekiel:nonlinear:1990} S. Iezekiel, C. M. Snowden,
``Nonlinear Circuit Analysis of Laser Diodes Under Microwave
Direct Modulation,'' {\em IEEE MTT-S International}, Vol. 2, pp.
937--940, 1990.

\bibitem{towe:ahistorical:2000} E. Towe, R. F. Leheny, A. Yang,
``A Historical Perspective of the Developement of the
Vertical-Cavity Surface-Emitting Laser,'' {\em IEEE J. on Selected
Topics in Quantum Electronics}, Vol.6, No. 6, Nov/Dec 2000.

\bibitem{ohiso:flip-chip:1996} Y. Ohiso, K. Tateno, Y. Kohama,
A. Wakatsuki, H. Tsunetsugu, and T. Kurokawa, ``Flip-chip bonded
0.85-$\mu m$ bottom-emitting vertical-cavity laser array on an
AlGaAs substrate,'' {\em IEEE Photon. Technol. Lett.}, Vol. 8, pp.
1115-1117, Sep 1996.

\bibitem{mena:asimple:1999} P. V. Mena, J. J. Morikuni, S.-M.
Kang, A. V. Harton, K. W. Wyatt, ``A Simple Rate-Equation-Based
Thermal VCSEL Model,'' {\em IEEE J. of Lightwave Technology}, Vol.
17, No. 5, May 1999.

\bibitem{neifeld:vcselst:2003} M. A. Neifeld (private communication),
2003.

\bibitem{hasnain:performance:1991} G. Hasnain, K. Tai, L. Yang,
Y. H. Wang, R. J. Fisher, J. D. Wynn, B. Weir, N. K. Dutta, and A.
Y. Cho, ``Performance of Gain-guided Surface Emitting Lasers with
Semiconductor Distributed Bragg Reflectors,'' {\em IEEE J. of
Quantum Electron.}, Vol. 27, No. 6, pp. 1377-1385, 1991.

\bibitem{agrawal:semi:1993} G. P. Agrawal, and N. K. Dutta,
``Semiconductor Laser, 2nd ed.'' {\em New York: Van Nostrand
Reinhold}, 1995.

\bibitem{neifeld:packaging:1995} M. A. Neifeld and W. C. Chou,
``Electrical packaging impact on source components in optical
interconnect,'' {\em IEEE Transactions on Components, Packaging,
and Manufacturing Technology --- Part B}, Vol. 18, pp. 578--595,
Aug. 1995.

\bibitem{bruensteiner:extraction:1999} M. Bruensteiner, G. C. Papen,
``Extraction of VCSEL Rate-Equation Parameters for Low-Bias System
Simulation,'' {\em IEEE J. of Selected Topics in Quantum Electron},
Vol. 5, No. 3, 1999.

\bibitem{carlsson:analog:2002} C. Carlsson, H. Martinsson, R. Schatz, J. Halonen,
A. Larsson ``Analog Modulation Properties of Oxide Confined VCSELs at
Microwave Frequencies,'' {\em IEEE J. of Lightwave
Technology}, Vol. 20, No. 9, September 2002.

\bibitem{fernandez:toward:2002} A. F. Fernandez, F. Berghmans, B. Brichard,
M. Decreton, ``Toward The Developement of Radiation-Tolerant Instrumentation
Data Links for Thermonuclear Fusion Experiments,'' {\em IEEE Transactin on
Nuclear Science}, Vol. 49, No. 6, December 2002.

\end{thebibliography}

\noindent\linethickness{0.5mm} \line(1,0){425}
\newline
\textit{Version:}\\
2003.01.01 \\
% Credits
\linethickness{0.5mm} \line(1,0){425}
\newline
\textit{Credits:}\\
\begin{tabular}{l l l l}
Name & Affiliation & Date & Logo \\
Houssam Kanj & NC State University & 2003 & \epsfxsize=1in\epsfbox{figures/logo.eps} \\
www.ncsu.edu & & & \\
\end{tabular}

\end{document}
