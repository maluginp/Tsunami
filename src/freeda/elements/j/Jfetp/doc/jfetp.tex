\documentclass{article}
\usepackage{epsf}\usepackage{here}
\usepackage{graphicx}
\oddsidemargin 0.25in \evensidemargin 0.25in
\topmargin 0.0in
\textwidth 6.5in \textheight 8.5in
\headheight 0.18in \footskip 0.16in
\leftmargin -0.5in \rightmargin -0.5in

%
% KEYWORD
%
\newcommand{\keywordtable}[1]{
        \sloppy
        \hyphenation{ca-pac-i-t-an-ce}
        \begin{center}
    \sf
        \begin{tabular}[t]
        {|p{0.58in}|p{3.07in}|p{0.55in}|p{0.60in}|}
        \hline
        \multicolumn{1}{|c}{\bf Name} &
        \multicolumn{1}{|c}{\parbox{2.77in}{\bf Description}}  &
        \multicolumn{1}{|c}{\bf Units} &
        \multicolumn{1}{|c|}{\bf Default} \X
        #1
        \end{tabular}
        \end{center}
    }

\newcommand{\keywordtwotable}[2]{
        \sloppy
        \hyphenation{ca-pac-i-t-an-ce}
        \begin{center}
    \sf
        \begin{tabular}[t]
        {|p{0.58in}|p{2.38in}|p{0.55in}|p{0.60in}|p{0.53in}|}
        \hline
        \multicolumn{1}{|c}{\bf Name} &
        \multicolumn{1}{|c}{\parbox{2.20in}{\bf Description}}  &
        \multicolumn{1}{|c}{\bf Units} &
        \multicolumn{1}{|c}{\bf Default} &
        \multicolumn{1}{|c|}{\bf #1} \X
        #2
        \end{tabular}
        \end{center}
    }

\newcommand{\kw}[2]{
     \samepage{
     \noindent {\sl #1} \vspace{-0.5in} \\
     \keywordtable{#2} }}

\newcommand{\kwtwo}[3]{
     \samepage{
     \noindent {\sl #1} \vspace{-0.4in} \\
     \keywordtwotable{#2}{#3} }}

\newcommand{\keyword}[1]{\kw{Keywords:}{#1}}
\newcommand{\keywordtwo}[2]{\kwtwo{Keywords:}{#1}{#2}}
\newcommand{\modelkeyword}[1]{\kw{Model Keywords}{#1}}
\newcommand{\modelkeywordtwo}[2]{\kwtwo{Model Keywords}{#1}{#2}}

\newcommand{\myline}{\\[-0.1in]
\noindent \rule{\textwidth}{0.01in} \newline}

\newcommand{\myThickLine}{\\[-0.1in]
\noindent \rule{\textwidth}{0.02in} \newline}


% FORM
\newcommand{\form}[1]{\samepage{\noindent
 {\sl Form} \myline
% \hspace*{\fill} % For some reason \fill = 0 when \pspiceform{} is used?
\offset
\it  \offsetparbox{#1}}
\\[0.1in]}

% ELEMENT FORM
\newcommand{\elementform}[1]{\samepage{\noindent
 {\sl Element Form} \myline
% \hspace*{\fill} % For some reason \fill = 0 when \pspiceform{} is used?
\offset
\it  \offsetparbox{#1}}
\\[0.1in]}

% MODEL FORM
\newcommand{\modelform}[1]{\samepage{\noindent
 {\sl Model Form} \myline
% \hspace*{\fill} % For some reason \fill = 0 when \pspiceform{} is used?
\offset
\it  \offsetparbox{#1}}
\\[0.1in]}

% LIMITS
\newcommand{\mylimits}[1]{\samepage{\noindent
 {\sl Limits} \myline
 \hspace*{\fill} \it  \offsetparbox{#1}}
 \vshift}

% EXAMPLE
\newcommand{\example}[1]{\samepage{\noindent
{\sl Example} \myline
\offset \tt  \offsetparbox{#1}}
 \vshift}

% PSPICE88 EXAMPLE
\newcommand{\pspiceexample}[1]{\samepage{\noindent
{\sl \pspice\ Example} \myline
\offset \tt  \offsetparbox{#1}}
 \vshift}

% MODEL TYPES
\newcommand{\modeltype}[1]{\samepage{\noindent
{\sl Model Type} \myline
 \hspace*{\fill} \tt \offsetparbox{#1}}
 \\[0.1in]}

% MODEL TYPES
\newcommand{\modeltypes}[1]{\samepage{\noindent
{\sl Model Types:} \myline
 \hspace*{\fill} \tt \offsetparbox{\tt #1}}
 \vshift}

% OFFSET ENUMERATE
\newcommand{\offsetenumerate}[1]{
     \offset \hspace*{-0.1in} {\begin{enumerate} #1 \end{enumerate}}}

% NOTE
\newcommand{\note}[1]{
\vshift\samepage{\noindent {\sl Note}\myline\vspace{-0.24in}}
 \offsetenumerate{#1} }

% SPECIAL NOTE
\newcommand{\specialnote}[2]{
\vshift\samepage{\noindent {\sl #1}\myline\vspace{-0.24in}}\\#2}

\newcommand{\dc}{\mbox{\tt DC}}
\newcommand{\ac}{\mbox{\tt AC}}
\newcommand{\SPICE}{\mbox{\tt SPICE}}
\newcommand{\m}[1]{{\bf #1}}                           % matrix command  \m{}

% ////// Changing nodes to terminals///////
% print terminals in \tt and enclose in a circle use outside
\newcommand{\terminal}[1]{\: \mbox{\tt #1} \!\!\!\! \bigcirc }
%
% set up environment for example
%
\newcounter{excount}
\newcounter{dummy}
\newenvironment{eg}{\vspace{0.1in}\noindent\rule{\textwidth}{.5mm}
   \begin{list}
   {{\addtocounter{excount}{1}
   \em Example\/ \arabic{chapter}.\arabic{excount}\/}:}
   {\usecounter{dummy}
   \setlength{\rightmargin}{\leftmargin}}
   }{\end{list} \rule{\textwidth}{.5mm}\vspace{0.1in}}
%
% set up environment for block
% currently this draws a horizontal line at the start of block and another
% at the end of block.
%
\newenvironment{block}{\vspace{0.1in}\noindent\rule{\textwidth}{.5mm}
   }{\rule{\textwidth}{.5mm}\vspace{0.1in}}
%


%
% set up wide descriptive list
%
\newenvironment{widelist}
    {\begin{list}{}{\setlength{\rightmargin}{0in} \setlength{\itemsep}{0.1in}
    \setlength{\labelwidth}{0.95in} \setlength{\labelsep}{0.1in}
\setlength{\listparindent}{0in} \setlength{\parsep}{0in}
    \setlength{\leftmargin}{1.0in}}
    }{\end{list}}

\newcommand{\STAR}{\hspace*{\fill} * \hspace*{\fill}}

\newcommand{\sym}[1]{\hspace*{\fill} ($#1$)}

\newcommand{\optionitem}[2]{
\item[{\tt #1}{#2}]\label{.OPTION#1}\index{.OPTIONS, #1}\index{#1}}

\newcommand{\error}[1]{\vspace{0.1in}\noindent{\tt #1}\\}


\begin{document}
\noindent{\LARGE \textbf{P Channel JFET}
\hspace{\fill}\textbf{jfetp}}
\hrulefill\linethickness{0.5mm}\line(1,0){425}
\normalsize
\newline
% the figure
\begin{figure}[h]
\centerline{\epsfxsize=4in\epsfbox{j.ps}} \caption{JFET Types (a)
N-channel FET (b) P-channel FET}
\end{figure}
\newline
% form for \FDA
\linethickness{0.5mm} \line(1,0){425}
\newline
\textit{Form:}
%\newline
$\tt jfetp$:$\langle \tt{instance\ name}\rangle$ $n_1\ n_2\ n_3\ $
$\langle \tt{parameter\ list}\rangle$
\newline
\begin{tabular}{r l}
$n_1$ & is the drain node, \\
$n_2$ & is the gate node, \\
$n_3$ & is the source node, \\
\end{tabular}
% Parameter list
\newline
\textit{Parameters:}
\begin{table}[H]
\begin{tabular}{|c|c|c|c|}
\hline
Parameter&Type&Default value&Required?\\
\hline
af: Flicker noise exponent & DOUBLE & 1 & no\\
\hline
area: Device area ($\textrm{m}^2$) & DOUBLE & 1 & no\\
\hline
beta: Transconductance parameter ($\textrm{A}/\textrm{V}^2$)& DOUBLE & 0.0001 & no\\
\hline
cgs: Zero bias gate source junction capacitance (F) & DOUBLE & 0 & no\\
\hline
cgd: Zero bias gate drain junction capacitance (F)& DOUBLE & 0 & no\\
\hline
eg: Barrier height at 0 K (eV) & DOUBLE & 0.8 & no\\
\hline
fc: Coefficient for forward bias depletion capacitance & DOUBLE & 0.5 & no\\
\hline
is: Gate junction saturation current (A)& DOUBLE & $1\times10^{-14}$ & no\\
\hline
kf: Flicker noise coefficient & DOUBLE & 0 & no\\
\hline
pb: Gate junction potential (1/V)& DOUBLE & 0 & no\\
\hline
rd: Drain ohmic resistance ($\omega$)& DOUBLE & 0 & no\\
\hline
rs: Source ohmic resistance ($\omega$) & DOUBLE & 0 & no\\
\hline
vt0: Threshold voltage (V)& DOUBLE & -2 & no\\
\hline
m: Gate p-n grading coefficient & DOUBLE & 0.5 & no\\
\hline
vt0tc: Temperature coefficient for vt0 (V/K) & DOUBLE & 0.0 & no\\
\hline
tnom: Nominal temperature (K)& DOUBLE & 300 & no\\
\hline
b: Doping tail parameter & DOUBLE & 1 & no\\
\hline
t: Device temperature (K) & DOUBLE & 300 & no\\
\hline
lambda: Channel length modulation parameter (1/V) & DOUBLE & 0 & no\\
\par
\hline
\end{tabular}
\end{table}
%\newpage
%\textit{Parameters:}
% example in \FDA
\noindent\linethickness{0.5mm}\line(1,0){425}
\newline
\textit{Example:}
\newline
\texttt{jfetp:j1\ 3\ 4\ 2\ beta=0.0001}
\newline
\linethickness{0.5mm} \line(1,0){425}
\newline
\textit{Description:}\\
\FDA has the PJFET model based on the PJF model in SPICE. \\
% Equations follow below
\newpage
\noindent\textbf{DC Calculations:} \\
\textit{Constants used are:} \\
\begin{eqnarray}
q & = & 1.6021918 \times 10^{-19} (As) \\
k & = & 1.3806226 \times 10^{-23} (J/K)
\end{eqnarray}
All parameters used are indicated in \texttt{this} font. The
equations for the P-channel FET are identical to the N-channel
except that the signs on the voltages and the output are reversed.
\\

\noindent The current/voltage characteristics are evaluated after
first determining the mode (normal: $V_{SD} \ge 0$ or inverted:
$V_{SD} < 0$) and the region (cutoff, linear or saturation) of the
current $(V_{SD}, V_{SG})$ operating point.\\

\noindent{\sl Normal Mode: ($V_{SD} \ge 0$)}\\

Regions of operation:
\newline
\hspace*{12mm}{
\begin{tabular}{l l}
$V_{SG} - V_{T0} \leq 0$ & Cutoff Region \\
$0 \leq V_{SD} < V_{SG} - V_{T0}$ & Linear Region \\
$0 < V_{SG} - V_{T0} \leq V_{SD}$ & Saturation Region
\end{tabular}}\\
Then
\begin{equation}
I_{D} = \left\{ \begin{array}{ll}
      0  & \mbox{cutoff region} \\ \\
      \texttt{AREA} \times\, \texttt{BETA} \left(1 + \texttt{LAMBDA} \, V_{SD}\right)V_{SD}
      \left[2\left(V_{SG}- \texttt{VT0}\right)-V_{SD}\right]
         &\mbox{linear region}\\
      \texttt{AREA} \times \, \texttt{BETA} \left(1 + \texttt{LAMBDA} \, V_{SD}\right)
      \left(V_{SG}- \texttt{VT0}\right)^2
         &\mbox{saturation region} \end{array} \right. %}
\end{equation}\\

\noindent{\sl Inverted Mode: ($V_{SD} < 0)$}\\

Regions of operation:
\newline
\hspace*{12mm}{
\begin{tabular}{l l}
$V_{DG} - V_{T0} \leq 0$ & Cutoff Region \\
$0 \leq -V_{SD} < V_{DG} - V_{T0}$ & Linear Region \\
$0 < V_{DG} - V_{T0} \leq -V_{SD}$ & Saturation Region
\end{tabular}}\\
\begin{equation}
I_{D} = \left\{ \begin{array}{ll}
      0  & \mbox{cutoff region} \\ \\
      \texttt{AREA}\times \,\texttt{BETA} \left(1 - \texttt{LAMBDA} V_{SD}\right)V_{SD}
      \left[2\left(V_{DG}- \texttt{VT0}\right)+V_{SD}\right]
         &\mbox{linear region}\\
      \texttt{AREA} \times \,(-\texttt{BETA}) \left(1 - \texttt{LAMBDA} V_{SD}\right)
      \left(V_{DG} - \texttt{VT0}\right)^2
         &\mbox{saturation region} \end{array} \right. %}
\end{equation}\\

% Leakage Currents
\noindent \textit{Leakage Currents}\\
Current flows across the normally reverse biased source-bulk and
drain-bulk junctions. The gate-source leakage current
\begin{equation}
I_{GS} = \texttt{AREA} \times \, I_{S} \, e^{(\textstyle
V_{SG}/\texttt{VT0} -1)}
\end{equation}
and the gate-source leakage current
\begin{equation}
I_{GD} = \texttt{AREA} \times \, I_{S} \, e^{(\textstyle
V_{DG}/\texttt{VT0} -1)}
\end{equation}
\newline
% Capacitances
\noindent \textit{Capacitances}\\
The drain-source capacitance
\begin{equation}
C_{DS} = \texttt{AREA} \times \, \texttt{CDS}
\end{equation}
The gate-source capacitance
\begin{equation}
C_{GS} = \left\{ \begin{array}{ll}
         \texttt{AREA} \times \,\texttt{CGS}\left(1 - {{\textstyle V_{SG}}\over{\textstyle \texttt{PB}}}
         \right)^{-\texttt{M}}
         & V_{SG} \le \texttt{FC} \times \texttt{PB}\\
         \texttt{AREA} \times \,\texttt{CGS}\left(1 -\texttt{FC}\right)^{-(1+\texttt{M})}
         \left[1-\texttt{FC}(1+\texttt{M})+\texttt{M} {{\textstyle V_{SG}}\over{\textstyle \texttt{PB}}}
         \right]^{-\texttt{M}}
         & V_{SG} > \texttt{FC} \times \texttt{PB}
         \end{array} \right. %}
\end{equation}
models charge storage at the gate-source depletion layer. The
gate-drain capacitance
\begin{equation}
C_{GD} = \left\{ \begin{array}{ll}
         \texttt{AREA} \times \,\texttt{CGD}\left(1 - {{\textstyle V_{DG}}\over{\textstyle \texttt{PB}}}
         \right)^{-\texttt{M}}
         & V_{DG} \le \texttt{FC} \times \texttt{PB}\\
         \texttt{AREA} \times \,\texttt{CGD}\left(1 -\texttt{FC}\right)^{-(1+\texttt{M})}
         \left[1-\texttt{FC}(1+\texttt{M})+\texttt{M} {{\textstyle V_{DG}}\over{\textstyle \texttt{PB}}}
         \right]^{-\texttt{M}}
         & V_{DG} > \texttt{FC} \times \texttt{PB}
         \end{array} \right. %}
\end{equation}
models charge storage at the gate-drain depletion layer.


\noindent\linethickness{0.5mm} \line(1,0){425}
\newline
\textit{Notes:}\\
This is the \texttt{J} element in the SPICE compatible netlist.\\
\linethickness{0.5mm} \line(1,0){425}
\newline
\textit{Version:}\\
2001.05.15 \\
%% Credits
\linethickness{0.5mm} \line(1,0){425}
\newline
\textit{Credits:}\\
\begin{tabular}{l l l l}
Name & Affiliation & Date & Links \\
Nikhil Kriplani & NC State University & May 2001 & \epsfxsize=1in\epsfbox{logo.eps} \\
nmkripla@unity.ncsu.edu & & & www.ncsu.edu    \\
\end{tabular}
\end{document}
