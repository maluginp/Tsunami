\documentclass{article}
\usepackage{epsf}\usepackage{here}
\usepackage{graphicx}
\oddsidemargin 0.25in \evensidemargin 0.25in
\topmargin 0.0in
\textwidth 6.5in \textheight 8.5in
\headheight 0.18in \footskip 0.16in
\leftmargin -0.5in \rightmargin -0.5in

%
% KEYWORD
%
\newcommand{\keywordtable}[1]{
        \sloppy
        \hyphenation{ca-pac-i-t-an-ce}
        \begin{center}
    \sf
        \begin{tabular}[t]
        {|p{0.58in}|p{3.07in}|p{0.55in}|p{0.60in}|}
        \hline
        \multicolumn{1}{|c}{\bf Name} &
        \multicolumn{1}{|c}{\parbox{2.77in}{\bf Description}}  &
        \multicolumn{1}{|c}{\bf Units} &
        \multicolumn{1}{|c|}{\bf Default} \X
        #1
        \end{tabular}
        \end{center}
    }

\newcommand{\keywordtwotable}[2]{
        \sloppy
        \hyphenation{ca-pac-i-t-an-ce}
        \begin{center}
    \sf
        \begin{tabular}[t]
        {|p{0.58in}|p{2.38in}|p{0.55in}|p{0.60in}|p{0.53in}|}
        \hline
        \multicolumn{1}{|c}{\bf Name} &
        \multicolumn{1}{|c}{\parbox{2.20in}{\bf Description}}  &
        \multicolumn{1}{|c}{\bf Units} &
        \multicolumn{1}{|c}{\bf Default} &
        \multicolumn{1}{|c|}{\bf #1} \X
        #2
        \end{tabular}
        \end{center}
    }

\newcommand{\kw}[2]{
     \samepage{
     \noindent {\sl #1} \vspace{-0.5in} \\
     \keywordtable{#2} }}

\newcommand{\kwtwo}[3]{
     \samepage{
     \noindent {\sl #1} \vspace{-0.4in} \\
     \keywordtwotable{#2}{#3} }}

\newcommand{\keyword}[1]{\kw{Keywords:}{#1}}
\newcommand{\keywordtwo}[2]{\kwtwo{Keywords:}{#1}{#2}}
\newcommand{\modelkeyword}[1]{\kw{Model Keywords}{#1}}
\newcommand{\modelkeywordtwo}[2]{\kwtwo{Model Keywords}{#1}{#2}}

\newcommand{\myline}{\\[-0.1in]
\noindent \rule{\textwidth}{0.01in} \newline}

\newcommand{\myThickLine}{\\[-0.1in]
\noindent \rule{\textwidth}{0.02in} \newline}


% FORM
\newcommand{\form}[1]{\samepage{\noindent
 {\sl Form} \myline
% \hspace*{\fill} % For some reason \fill = 0 when \pspiceform{} is used?
\offset
\it  \offsetparbox{#1}}
\\[0.1in]}

% ELEMENT FORM
\newcommand{\elementform}[1]{\samepage{\noindent
 {\sl Element Form} \myline
% \hspace*{\fill} % For some reason \fill = 0 when \pspiceform{} is used?
\offset
\it  \offsetparbox{#1}}
\\[0.1in]}

% MODEL FORM
\newcommand{\modelform}[1]{\samepage{\noindent
 {\sl Model Form} \myline
% \hspace*{\fill} % For some reason \fill = 0 when \pspiceform{} is used?
\offset
\it  \offsetparbox{#1}}
\\[0.1in]}

% LIMITS
\newcommand{\mylimits}[1]{\samepage{\noindent
 {\sl Limits} \myline
 \hspace*{\fill} \it  \offsetparbox{#1}}
 \vshift}

% EXAMPLE
\newcommand{\example}[1]{\samepage{\noindent
{\sl Example} \myline
\offset \tt  \offsetparbox{#1}}
 \vshift}

% PSPICE88 EXAMPLE
\newcommand{\pspiceexample}[1]{\samepage{\noindent
{\sl \pspice\ Example} \myline
\offset \tt  \offsetparbox{#1}}
 \vshift}

% MODEL TYPES
\newcommand{\modeltype}[1]{\samepage{\noindent
{\sl Model Type} \myline
 \hspace*{\fill} \tt \offsetparbox{#1}}
 \\[0.1in]}

% MODEL TYPES
\newcommand{\modeltypes}[1]{\samepage{\noindent
{\sl Model Types:} \myline
 \hspace*{\fill} \tt \offsetparbox{\tt #1}}
 \vshift}

% OFFSET ENUMERATE
\newcommand{\offsetenumerate}[1]{
     \offset \hspace*{-0.1in} {\begin{enumerate} #1 \end{enumerate}}}

% NOTE
\newcommand{\note}[1]{
\vshift\samepage{\noindent {\sl Note}\myline\vspace{-0.24in}}
 \offsetenumerate{#1} }

% SPECIAL NOTE
\newcommand{\specialnote}[2]{
\vshift\samepage{\noindent {\sl #1}\myline\vspace{-0.24in}}\\#2}

\newcommand{\dc}{\mbox{\tt DC}}
\newcommand{\ac}{\mbox{\tt AC}}
\newcommand{\SPICE}{\mbox{\tt SPICE}}
\newcommand{\m}[1]{{\bf #1}}                           % matrix command  \m{}

% ////// Changing nodes to terminals///////
% print terminals in \tt and enclose in a circle use outside
\newcommand{\terminal}[1]{\: \mbox{\tt #1} \!\!\!\! \bigcirc }
%
% set up environment for example
%
\newcounter{excount}
\newcounter{dummy}
\newenvironment{eg}{\vspace{0.1in}\noindent\rule{\textwidth}{.5mm}
   \begin{list}
   {{\addtocounter{excount}{1}
   \em Example\/ \arabic{chapter}.\arabic{excount}\/}:}
   {\usecounter{dummy}
   \setlength{\rightmargin}{\leftmargin}}
   }{\end{list} \rule{\textwidth}{.5mm}\vspace{0.1in}}
%
% set up environment for block
% currently this draws a horizontal line at the start of block and another
% at the end of block.
%
\newenvironment{block}{\vspace{0.1in}\noindent\rule{\textwidth}{.5mm}
   }{\rule{\textwidth}{.5mm}\vspace{0.1in}}
%


%
% set up wide descriptive list
%
\newenvironment{widelist}
    {\begin{list}{}{\setlength{\rightmargin}{0in} \setlength{\itemsep}{0.1in}
    \setlength{\labelwidth}{0.95in} \setlength{\labelsep}{0.1in}
\setlength{\listparindent}{0in} \setlength{\parsep}{0in}
    \setlength{\leftmargin}{1.0in}}
    }{\end{list}}

\newcommand{\STAR}{\hspace*{\fill} * \hspace*{\fill}}

\newcommand{\sym}[1]{\hspace*{\fill} ($#1$)}

\newcommand{\optionitem}[2]{
\item[{\tt #1}{#2}]\label{.OPTION#1}\index{.OPTIONS, #1}\index{#1}}

\newcommand{\error}[1]{\vspace{0.1in}\noindent{\tt #1}\\}


\begin{document}
\noindent{\LARGE \textbf{N Channel MOSFET BSIM4 model}
\hspace{\fill} \textbf{mosnbsim4}}
\hrulefill\linethickness{0.5mm}\line(1,0){425} \normalsize
\newline
% the resistor figure
\begin{figure}[h]
\centerline{\epsfxsize=4in\epsfbox{figures/m.ps}} \caption{N Channel MOSFET
BSIM4 model}
\end{figure}
\newline
% form for \FDA
\linethickness{0.5mm} \line(1,0){425}
\newline
\textit{Form:}
%\newline
$\tt mosnbsim4$:$\langle \tt{instance\ name}\rangle$ $n_1\ n_2\
n_3\ n_4\ $ $\langle \tt{parameter\ list}\rangle$
\newline
\begin{tabular}{r l}
$n_1$ & is the drain node, \\
$n_2$ & is the gate node, \\
$n_3$ & is the source node, \\
$n_4$ & is the bulk node. \\
\end{tabular}
% Parameter list
\newpage
%\newline
\noindent\textit{Model Parameters:}
\newline
%The parameters used in the model are listed in the tables
%~\ref{bsim4.params1} to ~\ref{bsim4.params8}.

\begin{table}[H]
\begin{tabular}{|c| c| c| c|}
\hline
\textbf{Parameter} & \textbf{Description} & \textbf{Default} & \textbf{Units} \\
\hline
\texttt{TOXE} & Electrical gate equivalent oxide & & \\
              & thickness & 3.0e-9 & $m$\\
\texttt{TOXP} & Physical gate equivalent oxide & & \\
              & thickness & TOXE & $m$\\
\texttt{EPSROX} & Gate dielectric constant relative & & \\
                & to vacuum & 3.9 & -\\
\texttt{VFB} & Flat-band voltage & -1.0 & $V$\\
\texttt{VTH0} & Long-channel threshold voltage & 0.7 & $V$ \\
\texttt{NGATE} & Poly Si gate doping concentration & 0.0 & $cm^{-3}$\\
\texttt{XL} & Channel length offset due to & & \\
            & mask/etch effect & 0.0 & $m$\\
\texttt{XW} & Channel width offset due to & & \\
            & mask/etch effect & 0.0 & $m$\\
\texttt{NF} & Number of device fingers & 1.0 & -\\
\texttt{W} & Width of the device & 5.0e-6 & $m$\\
\texttt{L} & Length of the device & 5.0e-6 & $m$\\
\texttt{DWG} & Coefficient of gate bias & & \\
             & dependence of Weff & 0.0 & $m/V$\\
\texttt{DWB} & Coefficient of body bias & & \\
             & dependence of Weff & 0.0 & $m/V$\\
\texttt{WINT} & Channel-width offset parameter & 0.0 & $m$\\
\texttt{WLN} & Power of length dependence of & & \\
             & width offset & 1.0 & $m$\\
\texttt{WL} & Coefficient of length dependence & & \\
            & for width offset & 0.0 & $m$\\
\texttt{WWN} & Power of width dependence & & \\
                             & of width offset & 1.0 & $m$\\
\texttt{WW} & Coefficient of width dependence & & \\
            & for width offset & 0.0 & $m$\\
\texttt{WWL} & Coefficient of length and & & \\
             & width cross term dependence & & \\
             & for width offset & & $m$\\
\texttt{LINT} & Channel-length offset parameter & 0.0 & $m$\\
\texttt{LLN} & Power of length dependence & & \\
             & for length offset & 0.0 & $m$\\
\texttt{LL} & Coefficient of length dependence & & \\
            & for length offset & 0.0 & $m$\\
\texttt{LW} & Coefficient of width dependence & & \\
            & for length offset & 0.0 & $m$\\
\texttt{LWN} & Power of width dependence & & \\
             & for length offset & 1.0 & $m$\\
\texttt{LWL} & Coefficient of length and & & \\
             & width cross term dependence & & \\
             & for length offset & 0.0 & $m$\\
\texttt{K1} & First-order body bias coefficient & 0.0 & $V^{-0.5}$\\
\texttt{K2} & Second-order body bias coefficient & 0.0 & -\\
\hline
\end{tabular}
\caption{\label{bsim4.params1} MOS Model Parameter table 1}
\end{table}


\newpage
\begin{table}[H]
\begin{tabular}{|l| l| l| l|}
\hline
\textbf{Parameter} & \textbf{Description} & \textbf{Default} & \textbf{Units} \\
\hline
\texttt{LPEB} & Lateral non-uniform doping & & \\
              & effect on K1 & 0.0 & $m$\\
\texttt{LPE0} & Lateral non-uniform doping & & \\
              & parameter at Vbs=0 & 1.74e-7 & $m$\\
\texttt{K3} & Narrow width coefficient & 80.0 & -\\
\texttt{K3B} & Body effect coefficient of K3 & 0.0 & $V^{-1}$\\
\texttt{W0} & Narrow width parameter & 2.5e-6 & $m$\\
\texttt{DVT0W} & First coefficient of narrow & & \\
               & width effect on threshold voltage & & \\
               & for small channel length & 0.0 & -\\
\texttt{DVT0} & First coefficient of short & & \\
              & channel effect on threshold & 2.2 & -\\
\texttt{DVT1W} & Second coefficient of narrow & & \\
               & width effect on threshold voltage& & \\
               & for small channel length & 5.3e6 & -\\
\texttt{DVT1} & Second coefficient of short & & \\
              & channel effect on threshold & 0.53 & -\\
\texttt{DSUB} & DIBL coefficient exponent in & & \\
              & sub-threshold region & 0.56 & - \\
\texttt{ETA0} & DIBL coefficient in & & \\
              & sub-threshold region & 0.56 & - \\
\texttt{ETAB} & Body-bias coefficient for & & \\
              & the sub-threshold region & -0.07 & -\\
\texttt{TOXM} & $T_{\rm{ox}}$ at which parameters & & \\
              & are extracted & TOXE & $m$\\
\texttt{T} & Temperature & 300.0 & $ ^oK$\\
\texttt{NDEP} & Channel doping concentration & & \\
              & at depletion edge for & & \\
              & zero body bias & 1.7e17 & $cm^{-3}$\\
\texttt{PHIN} & Non-uniform vertical doping & & \\
              & effect on surface potential & 0.0 & $V$\\
\texttt{VBM} & Maximum applied body bias & & \\
             & in VTH0 calculation & -3.0 & $V$\\
\texttt{NSUB} & Substrate doping concentration & 6.0e16 & $cm^{-3}$\\
\texttt{DVT2W} & Body-bias coefficient of & & \\
               & narrow width effect for & & \\
               & small channel length & -0.032 & - \\
\texttt{NSD} & Source/drain doping concentration & 1.0e20 & $cm^{-3}$\\
\texttt{DVT2} & Body-bias coefficient of & & \\
              & short-channel effect on threshold & -0.032 & -\\
\texttt{MINV} & $V_{gsteff}$ fitting parameter & & \\
              & for moderate inversion condition & -0.0 & - \\
\texttt{NFACTOR} & Subthreshold swing factor & 1.0 & - \\
\texttt{CDSC} & Coupling capacitance between & & \\
              & source/drain and channel & 1.0 & $F/m^2$\\
\texttt{CDSCD} & Drain-bias sensitivity of CDSC & 2.4e-4 & $F/Vm^2$\\
\texttt{CDSCB} & Body-bias sensitivity of CDSC & 0.0 & $F/Vm^2$\\
\texttt{CIT} & Interface trap capacitance & 0.0 & $F/m^2$\\
\hline
\end{tabular}
\caption{\label{bsim4.params2} MOS Model Parameter table 2}
\end{table}


\newpage
\begin{table}[H]
\begin{tabular}{|l| l| l| l|}
\hline
\textbf{Parameter} & \textbf{Description} & \textbf{Default} & \textbf{Units} \\
\hline
\texttt{KETA} & Body-bias coefficient of & & \\
              & bulk charge effect & -0.047 & $V^{-1}$\\
\texttt{B0} & Bulk charge effect coefficient & & \\
            & for channel width & 0.0 & $m$\\
\texttt{B1} & Bulk charge effect width offset & 0.0 & $m$\\
\texttt{A0} & Coefficient of channel-length & & \\
            & dependence bulk charge effect & 1.0 & -\\
\texttt{AGS} & Coefficient of $V_{\rm{gs}}$ dependence & & \\
             & of bulk charge effect & 0.0 & $V^{-1}$\\
\texttt{XJ} & S/D junction depth & 1.5e-7 & $m$\\
\texttt{U0} & Low-field mobility & 0.067 & $m^2/Vs$\\
\texttt{UA} & Coefficient of first-order & & \\
            & mobility degradation due & & \\
            & to vertical field & 1.0e-15 & $m/V$\\
\texttt{UB} & Coefficient of second-order & & \\
            & mobility degradation due & & \\
            & to vertical field & 1.0e-19 & $m^2/V^2$\\
\texttt{UC} & Coefficient of mobility degradation & & \\
            & due to body-bias effect & -0.0465e-9 & $m/V$\\
\texttt{EU} & Exponent for mobility degradation & 1.67 & - \\
\texttt{DELTA} & Parameter for DC $V_{dseff}$ & 0.01 & $V$\\
\texttt{PDITS} & Impact of drain-induced & & \\
               & threshold shift on $R_{out}$ & 0.0 & $V^{-1}$\\
\texttt{FPROUT} & Effect of pocket implant & & \\
                & on Rout degradation & 0.0 & $V/m^0.5$\\
\texttt{PDITSL} & Channel-length dependence of & & \\
                & drain-induced $V_{th}$ shift for $R_{out}$ & 0.0 & $m^{-1}$\\
\texttt{PDITSD} & $V_{ds}$ dependence of drain-induced & & \\
                & $V_{th}$ shift for Rout & 0.0 & $V^{-1}$\\
\texttt{PSCBE2} & Second substrate current & & \\
                & induced body-effect parameter & 1.0e-5 & $m/V$\\
\texttt{PSCBE1} & First substrate current induced & & \\
                & body-effect parameter & 4.24e8 & $V/m$\\
\texttt{PDIBLCB} & Body bias coefficient of & & \\
                 & DIBL effect on Rout & 0.0 & $V^{-1}$\\
\texttt{PVAG} & Gate-bias dependence of & & \\
              & Early voltage & 0.0 & - \\
\texttt{PDIBL1} & Parameter for DIBL effect & & \\
                & on $R_{out}$ & 0.0 & - \\
\texttt{PDIBL2} & Parameter for DIBL effect & & \\
                & on $R_{out}$ & 0.0 & - \\
\texttt{AGS} & Coefficient of $V_{\rm{gs}}$ dependence & & \\
             & of bulk charge effect & 0.0 & $V^{-1}$ \\
\texttt{DROUT} & Channel-length dependence of & & \\
               & DIBL effect on $R_{out}$ & 0.56 & -\\
\texttt{PCLM} & Channel length modulation & & \\
              & parameter & 1.3 & - \\
\texttt{A1} & First non-saturation effect & & \\
            & parameter & 0.0 & $V^{-1}$\\
\texttt{A2} & Second non-saturation factor & 1.0 & - \\
\hline
\end{tabular}
\caption{\label{bsim4.params3} MOS Model Parameter table 3}
\end{table}

\newpage
\begin{table}[H]
\begin{tabular}{|l| l| l| l|}
\hline
\textbf{Parameter} & \textbf{Description} & \textbf{Default} & \textbf{Units} \\
\hline
\texttt{RDWMIN} & Lightly-doped drain resistance & & \\
                & per unit width at high $V_{\rm{gs}}$ & & \\
                & and zero $V_{bs}$ & 0.0 & $\Omega$\\
\texttt{RDSW} & Zero bias lightly-doped & & \\
              & drain resistance per unit width & 200.0 & $\Omega$\\
\texttt{PRWG} & Gate-bias dependence of LDD & & \\
              & resistance & 1.0 & $V^{-1}$\\
\texttt{PRWB} & Body-bias dependence of LDD & & \\
              & resistance & 0.0 & $V^{-0.5}$\\
\texttt{WR} & Channel-width dependence & & \\
            &  parameter of LDD resistance & 1.0 & $m$\\
\texttt{WLC} & Coefficient of length dependence & & \\
             & for CV channel width offset & WL & $m$\\
\texttt{WWC} & Coefficient of width dependence & & \\
             & for CV channel width offset & WW & $m$\\
\texttt{WWLC} & Coefficient of length and & & \\
              & width cross term dependence & & \\
              & for CV channel width offset & WWL & $m$\\
\texttt{DWJ} & Offset of the S/D junction & & \\
             & width & WINT & $m$\\
\texttt{CLC} & Constant term for the short & & \\
             & channel model & 1.0e-7 & $m$\\
\texttt{CLE} & Exponential term for the & & \\
             & short channel model & 0.6 & -\\
\texttt{NOFF} & CV parameter in $V_{gsteffCV}$ & & \\
              & for weak to strong inversion & 1.0 & -\\
\texttt{VOFFCV} & CV parameter in $V_{gsteffCV}$ & & \\
                & for weak to strong inversion & 0.0 & $V$\\
\texttt{CF} & Fringing field capacitance & 0.0 & $F/m$\\
\texttt{CKAPPAD} & Coefficient of bias-dependent & & \\
                 & overlap capacitance for the & & \\
                 & drain side & 0.6 & $V$\\
\texttt{CKAPPAS} & Coefficient of bias-dependent & & \\
                 & overlap capacitance for the & & \\
                 & source side & 0.6 & $V$\\
\texttt{LLC} &  Coefficient of length dependence & & \\
              & on CV channel length offset & 0.0 & $m$\\
\texttt{LWC} &  Coefficient of width dependence & & \\
              & on CV channel length offset & 0.0 & $m$\\
\texttt{LWLC} & Coefficient of length and width & & \\
              & cross term dependence on CV & & \\
              & channel length offset & 0.0 & $m$\\
\texttt{WWLC} & Coefficient of length and width & & \\
              & cross term dependence on CV & & \\
              & channel width offset & 0.0 & $m$\\
\texttt{VOFF} & Offset voltage in the & & \\
              & subthreshold region for large & & \\
              & W and L & -0.08 & $V$\\
\texttt{VOFFL} & Channel length dependence of & & \\
               & \texttt{VOFF} & 0.0 & $V$\\
\hline
\end{tabular}
\caption{\label{bsim4.params4} MOS Model Parameter table 4}
\end{table}


\newpage
\begin{table}[H]
\begin{tabular}{|l| l| l| l|}
\hline
\textbf{Parameter} & \textbf{Description} & \textbf{Default} & \textbf{Units} \\
\hline
\texttt{POXEDGE} & Factor for the gate oxide & & \\
                 & thickness in S/D overlap regions & 1.0 & -\\
\texttt{TOXREF} & Nominal gate oxide thickness & & \\
                &for gate dielectric tunnelling & & \\
                &current model & 3.0e-9 & $m$\\
\texttt{NTOX} & Exponent for gate oxide ratio & 1.0 & -\\
\texttt{DLCIG} & Source/drain overlap length & & \\
               & for $I_{\rm{gs}}$ and $I_{\rm{gd}}$ & LINT & $m$\\
\texttt{AIGSD} & parameter for $I_{\rm{gs}}$ and $I_{\rm{gd}}$ & 0.43 & $(Fs^2/g)^{0.5} m^{-1}$\\
\texttt{BIGSD} & parameter for $I_{\rm{gs}}$ and $I_{\rm{gd}}$ & 0.054 & $(Fs^2/g)^{0.5} m^{-1}$\\
\texttt{CIGSD} & parameter for $I_{\rm{gs}}$ and $I_{\rm{gd}}$ & 0.075 & $(Fs^2/g)^{0.5} m^{-1}$\\
\texttt{MOIN} & Coefficient for gate-bias & & \\
              & dependent surface potential & 15.0 & - \\
\texttt{VSAT} & Saturation velocity & 8.0e4 & $m/s$\\
\texttt{PDITSD} & $V_{ds}$ dependence of drain & & \\
                & induced $V_{th}$ shift for Rout & 0.0 & $V^{-1}$\\
\texttt{AIGC} & Parameter for $I_{\rm{gcs}}$ and $I_{\rm{gcd}}$ & 0.43 & $(Fs^2/g)^{0.5} m^{-1}$\\
\texttt{BIGC} & Parameter for $I_{\rm{gcs}}$ and $I_{\rm{gcd}}$ & 0.054 & $(Fs^2/g)^{0.5} m^{-1}$\\
\texttt{CIGC} & Parameter for $I_{\rm{gcs}}$ and $I_{\rm{gcd}}$ & 0.075 & $(Fs^2/g)^{0.5} m^{-1}$\\
\texttt{NIGC} & Parameter for $I_{\rm{gcs}}$ $I_{\rm{gcd}}$, $I_{\rm{gs}}$ & & \\
              & and $I_{\rm{gd}}$ & 1.0 & $(Fs^2/g)^{0.5} m^{-1}$\\
\texttt{PIGCD} & $V_{ds}$ dependence of $I_{\rm{gcs}}$ and $I_{\rm{gcd}}$ & 1.0 & -\\
\texttt{DVTP0} & First coefficient of drain & & \\
               &induced $V_{th}$ shift due to & & \\
               & long channel pocket devices & 0.0 & $m$\\
\texttt{DVTP1} & First coefficient of drain & & \\
               & induced $V_{th}$ shift due to & & \\
               & long channel pocket devices & 0.0 & $V^{-1}$\\
\texttt{PRT} & Temperature coefficient for \texttt{RDSW} & 0.0 & $\Omega - m$\\
\texttt{AT} & Temperature coefficient for & & \\
            & saturation velocity & 3.3e4 & $m/s$\\
\texttt{XT} & Doping Depth & 1.55e-7 & $m$\\
\texttt{ALPHA0} & First parameter of impact & & \\
                &ionization current & 0.0 & $Am/V$\\
\texttt{ALPHA1} & $I_{\rm{sub}}$ parameter for length scaling & 0.0 & $A/V$\\
\texttt{BETA0} & Second parameter of impact & & \\
               & ionization current & 30.0 & $V$\\
\texttt{AGIDL} & Pre-exponential coefficient for & & \\
               & GIDL & 0.0 & $A/V$\\
\texttt{BGIDL} & Exponential coefficient for & & \\
               & GIDL & 2.3e9 & $V/m$\\
\texttt{CGIDL} & Parameter for body-bias effect & & \\
               & on GIDL & 0.5 & $V^3$\\
\texttt{EGIDL} & Fitting parameter for band & & \\
               & bending for GIDL & 0.8 & $V$\\
\texttt{ACDE} & Exponential coefficient for charge & & \\
              & thickness & 1.0 & $m/V$\\
\texttt{DLC} & Channel length offset parameter & LINT & $m$\\
\texttt{DWC} & Channel width offset parameter & WINT & $m$\\
\hline
\end{tabular}
\caption{\label{bsim4.params5} MOS Model Parameter table 5}
\end{table}



%\newpage
%\begin{table}[H]
%\begin{tabular}{|l| l| l| l|}
%\hline
%\textbf{Parameter} & \textbf{Description} & \textbf{Default} & \textbf{Units} \\
%\hline \hline
%\end{tabular}
%\caption{\label{bsim4.params7} MOS Model Parameter table 7}
%\end{table}

\newpage
\begin{table}[H]
\begin{tabular}{|l |l |l |l |}
\hline
\textbf{Parameter} & \textbf{Description} & \textbf{Default} & \textbf{Units} \\
\hline
\texttt{AIGBACC} & Parameter for $I_{gb}$ in accumulation & 0.43 & $m^{-1}$\\
\texttt{BIGBACC} & Parameter for $I_{gb}$ in accumulation & 0.054 & $m^{-1} V^{-1}$\\
\texttt{CIGBACC} & Parameter for $I_{gb}$ in accumulation & 0.075 & $V^{-1}$\\
\texttt{NIGBACC} & Parameter for $I_{gb}$ in accumulation & 1.0 & - \\
\texttt{AIGBINV} & Parameter for $I_{gb}$ in inversion & 0.35 & $m^{-1}$\\
\texttt{BIGBINV} & Parameter for $I_{gb}$ in inversion & 0.03 & $m^{-1} V^{-1}$\\
\texttt{CIGBINV} & Parameter for $I_{gb}$ in inversion & 0.006 & $V^{-1}$\\
\texttt{EIGBINV} & Parameter for $I_{gb}$ in inversion & 1.1 & $V$ \\
\texttt{NIGBINV} & Parameter for $I_{gb}$ in inversion & 3.0 & - \\
\texttt{KT1}     & Temperature coeff for $V_{\rm{TH}}$ & -0.11 & $V$ \\
\texttt{KT1L}    & Channel length for \texttt{KT1} & 0.0 & $Vm$ \\
\texttt{KT2}     & Body bias coeff for $V_{\rm{TH}}$ temp effect & 0.022 & - \\
\hline
\end{tabular}
\caption{\label{bsim4.params6} MOS Model Parameter table 6}
\end{table}

% example in \FDA
%\newline
%\linethickness{0.5mm} \line(1,0){425}
%\newline
%\textit{Example:}
%\newline
%\texttt{nmos3:m1\ 2\ 3\ 0\ 0\ l=1.2u w=20u}
%\newline
\noindent\linethickness{0.5mm} \line(1,0){425}
\newline
\textit{Description:}\\
The BSIM4 model takes a lot of it's characteristics from it's
predecessor, BSIM3 but also adds enough functionality to name it
with a new model number. It uses many new parameters and replaces
some old BSIM3 parameters. It uses a newly formulated smoothing
function for gate-source voltage. It uses more than 200
parameters, uses charge conserving equations for calculation of
various capacitances, has a single equation for modeling current
in all regions of transistor operation, has better modeling for
gate currents, for external parasitics, noise, temperature and
mobility. This chapter deals with the equations and features of
this advanced transistor model that have been modeled in \FDA.

\noindent{\it Channel Width and Length}:\\
The effective channel lengths and widths are less than the values
of \texttt{L} and \texttt{W} on account of diffusion effects.
\texttt{XL} and \texttt{XW} are parameters that account for the
channel length/width offset due to mask/etch effects and process
nonuniformity. The terms \texttt{dL} and \texttt{dW} are provided
for user convenience. They are turned off by default. The
effective length $L_{\rm{EFF}}$ is represented as
\begin{equation}
L_{\rm{EFF}} = \texttt{L} + 2\texttt{XL} - 2\Delta L_{\rm{geom}}\\
\end {equation}
where
\begin{equation}
\Delta L_{\rm{geom}} = {\texttt{LL} \over L^{\texttt{LLN}}} +
{\texttt{LW} \over W^{\texttt{LWN}}} + {\texttt{LWL} \over
L^{\texttt{LLN}}W^{\texttt{LWN}}}
\end{equation}

\noindent The effective width is represented as
\begin{equation}
W_{\rm{EFF}} = {W \over \texttt{NF}} + \texttt{XW} - 2\Delta
W_{\rm{geom}} - 2\Delta W_{\rm{biasdep}}
\end{equation}
where
\begin{equation}
\Delta W_{\rm{geom}} = {\texttt{WL} \over L^{\texttt{WLN}}} +
{\texttt{WW} \over W^{\texttt{WWN}}} + {\texttt{WWL} \over
L^{\texttt{WLN}}W^{\texttt{WWN}}}
\end{equation}

\begin{equation}
\Delta W_{\rm{biasdep}} = \texttt{DWG} (V_{\rm{GSTeff}}) +
\texttt{DWB}(\sqrt{2\phi_f - V_{\rm{BSeff}}} - \sqrt{2\phi_f})
\end{equation}

\noindent{\it Threshold Voltage}:\\
This model attempts to accurately model threshold voltage and
include various channel effects such as DIBL (Drain Induced
Barrier Lowering), Non-uniform vertical doping, body-effect,
charge sharing between the source and drain, short-channel and
pocket implant effects.

\noindent{\it Effective Bulk-Source Voltage}:\\
$V_{\rm{BSeff}}$ is calculated in order to prevent the body bias
from taking unreasonably high values during simulation. It
provides an upper limit on the value of body bias.
\begin{equation}
V_{\rm{BSeff}} = V_{\rm{bc}} + {(V_{\rm{BS}} - V_{\rm{bc}} -
0.001) + \sqrt{(V_{\rm{BS}} - V_{\rm{bc}} - 0.001 )^2 -4 \;
V_{\rm{bc}} \; 0.001} \over 2}
\end{equation}
where $V_{\rm{bc}}$, which represents the maximum allowable
$V_{\rm{BS}}$ is given by
\begin{equation}
V_{\rm{bc}} = 0.9.(2\phi_f - {\tt{K1}^2 \over 4 \tt{K2}^2})
\end{equation}

\noindent The threshold voltage is evaluated as
\begin{eqnarray}
V_{\rm{TH}} & = & \texttt{VTH0} + \delta_{\rm{NP}}.(\Delta
V_{\rm{T,BodyEffect}}
- \Delta V_{\rm{T,ChargeSharing}} - \Delta V_{\rm{T,DIBL}} \nonumber\\
& & + \Delta V_{\rm{T,ReverseShortChannel}} + \Delta
V_{\rm{T,NarrowWidth}} + \Delta V_{\rm{T,SmallSize}} \nonumber\\
& & - \Delta V_{\rm{T,PocketImplant}})
\end{eqnarray}
In certain cases, devices are operated with a positive value of
$V_{\rm{BS}}$. In these cases, the threshold voltage reduces and
drive current increases. The parameters \texttt{K1} and
\texttt{K2} control the value of the body effect term and it is
modeled by
\begin{eqnarray}
\Delta V_{\rm{T,BodyEffect}}& = & [\texttt{K1} \; {\texttt{TOXE}
\over \texttt{TOXM}} \; \sqrt{2 \phi_f - V_{\rm{BSeff}}} -
\texttt{K1}
\sqrt{2 \phi_f}] \; \sqrt{1 + {\texttt{LPEB} \over L_{\rm{EFF}}}} \nonumber \\
& & - \texttt{K2} \; V_{\rm{BSeff}} \; {\texttt{TOXE} \over
\texttt{TOXM}}
\end{eqnarray}
In modern technologies, the threshold voltage first increases as
the effective length decreases before it takes on it's expected
trend of decrease as effective length decreases. To correctly
model the temporary increase of threshold voltage the term used is
\begin{equation}
\Delta V_{\rm{T,ReverseShortChannel}} = \texttt{K1} \;
{\texttt{TOXE} \over \texttt{TOXM}}\; (\sqrt{1 + {\texttt{LPE0}
\over L_{\rm{EFF}}}} - 1) \; \sqrt{2 \phi_f}
\end{equation}
\\
\indent As the channel becomes shorter, the threshold voltage
becomes more dependent on the channel length (SCE, short channel
effects) and on DIBL. As the product of effective length and width
reduces, the exponents in Equation ~\ref{small.size} reduce and
assume a finite value. This suggests that there is a shift in
threshold voltage for smaller devices. The value can be controlled
by the parameters \texttt{DVTOW} and \texttt{DVT1W}. SCE are
represented as
\begin{eqnarray}
\Delta V_{\rm{T,SmallSize}} & = & \texttt{DVT0W} \;
[\exp(-\texttt{DVT1W}\;
{W_{\rm{EFF}} \; L_{\rm{EFF}} \over 2 L_{\rm{tw}}}) \nonumber \\
& & + 2 \; \exp(-\texttt{DVT1W}\; {W_{\rm{EFF}} \; L_{\rm{EFF}}
\over 2 L_{\rm{tw}}})] \; (V_{\rm{bi}} - 2 \phi_f)
\label{small.size}
\end{eqnarray}
\noindent As $V_{\rm{DS}}$ increases in short channel devices,
there is a non-trivial change in the surface potential. As a
result, the barrier blocking the carriers in the drain from
entering the channel diminishes and the device turns on sooner.
Since this barrier lowering is induced by drain source voltage,
this effect is called Drain Induced Barrier Lowering. To model
DIBL, the following term is used.
\begin{eqnarray}
\Delta V_{\rm{T,DIBL}} & = & [\exp(-\texttt{DSUB}\; {L_{\rm{EFF}}
\over 2 L_{\rm{t0}}}) + 2 \exp(-\texttt{DSUB} \;{L_{\rm{EFF}}
\over 2 L_{\rm{t0}}})]
\nonumber \\
& & \times (\texttt{ETA0} + \texttt{ETAB} \; V_{\rm{BSeff}}) \;
V_{\rm{DS}}
\end{eqnarray}
\noindent The actual depletion region in the channel is larger
than what is usually assumed because of fringing fields. Thus, as
the channel width decreases, there is a net increase in the
threshold voltage. This is modeled by
\begin{equation}
\Delta V_{\rm{T,NarrowWidth}} = (\texttt{K3} + \texttt{K3B} \;
V_{\rm{BSeff}})\; {\texttt{TOXE} \over W_{\rm{EFF}} + \texttt{W0}}
\end{equation}
\noindent The influence of charge sharing effects between the
source and drain depends greatly on the size of the channel. It's
value increases as the channel lengths reduce. The effect of
charge sharing on threshold voltage is controlled by parameters
\texttt{DVT0}, \texttt{DVT1} and \texttt{DVT2}. When the effective
channel length is small, the exponents assume a finite value and
have a direct bearing on the value of threshold voltage. Increased
charge sharing  tends to reduce the value of threshold voltage and
it is represented as
\begin{equation}
\Delta V_{\rm{T,ChargeSharing}} = \texttt{DVT0} {0.5 \over
\cosh(\texttt{DVT1} \; L_{\rm{EFF}}/L_t) - 1 }(V_{\rm{bi}} - 2
\phi_f)
\end{equation}
$\Delta V_{\rm{T,PocketImplant}}$ is defined after the calculation
of ideality factor $n$. The built in potential is given as
\begin{equation}
V_{\rm{bi}} = {k [T + 273.15] \over q} \ln{(\texttt{NDEP NSD})
\over n_i^2}
\end{equation}
\noindent The characteristic length is given by:
\begin{equation}
L_t = \left\{\begin{array}{ll}
       {\sqrt{\epsilon_s X_{\rm{dep}} / C_{\rm{oxe}}}}\;(1 + \texttt{DVT2}\;V_{\rm{BSeff}}) & \\
                  & \texttt{DVT2}\;V_{\rm{BSeff}} \geq -0.5\\
       {\sqrt{\epsilon_s X_{\rm{dep}} / C_{\rm{oxe}}}}\;(1 + 3\;\texttt{DVT2}\;V_{\rm{BSeff}})  & \\
     \times (3 + 8\,\texttt{DVT2}\,V_{\rm{BSeff}})^{-1} & \texttt{DVT2}\;V_{\rm{BSeff}} <0.5
              \end{array}
                    \right.
\end{equation}

\begin{equation}
L_{\rm{t0}} = \sqrt{{\epsilon_s\; X_{\rm{dep0}} \over
C_{\rm{oxe}}}}
\end{equation}
where
\begin{equation}
C_{\rm{oxe}} = {\epsilon_{\rm{ox}} \over \texttt{TOXE}}
\end{equation}

\begin{equation}
L_{\rm{tw}} = \left\{\begin{array}{ll}
       {\sqrt{\epsilon_s X_{\rm{dep}} / C_{\rm{oxe}}}}\;(1 + \texttt{DVT2W}\;V_{\rm{BSeff}}) & \\
                  & \texttt{DVT2W}\;V_{\rm{BSeff}} \geq -0.5\\
       {\sqrt{\epsilon_s X_{\rm{dep}} / C_{\rm{oxe}}}}\;(1 + 3\;\texttt{DVT2W}\;V_{\rm{BSeff}})  & \\
     \times (3 + 8\,\texttt{DVT2W}\,V_{\rm{BSeff}})^{-1} & \texttt{DVT2W}\;V_{\rm{BSeff}} <0.5
              \end{array}
                    \right.
\end{equation}


\begin{equation}
X_{\rm{dep}} = {\sqrt{2 \epsilon_s (2\phi_f - V_{\rm{BSeff}})
\over q \;\texttt{NDEP}}}
\end{equation}

\begin{equation} X_{\rm{dep0}} = {\sqrt{2 \epsilon_s (2\phi_f) \over q \;\texttt{NDEP}}}
\end{equation}

\noindent{\it Effective Gate Source voltage}:\\
Care is taken in Equation ~\ref{poly.eqn} to make sure that the
voltage across the poly-silicon gate does not exceed the silicon
band gap voltage.
\begin{equation}
V_{\rm{poly}} = {q \; \epsilon_s \; \texttt{NGATE} \;
C_{\rm{oxe}}^2 \; 10^6 \over 2} [\sqrt{1 + {2(V_{\rm{GS}} -
V_{\rm{FB}} - 2\phi_f) \over q \; \epsilon_s \; \texttt{NGATE} \;
C_{\rm{oxe}}^2 \; 10^6}} - 1]^2
\end{equation}

\begin{equation}
V_{\rm{PolyEff}} = 1.12 - 0.5\;(1.12 - V_{\rm{poly}} - \delta +
\sqrt{(1.12 - V_{\rm{poly}} - \delta)^2 + 4\;\delta\;1.12})
\label{poly.eqn}
\end{equation}

\begin{equation}
V_{\rm{GSeff}} = V_{\rm{GS}} - V_{\rm{PolyEff}}
\end{equation}

\noindent{\it Effective $V_{\rm{GS}}$ - $V_{\rm{TH}}$ Smoothing
Function}:\\
This function smoothes out the characteristics between the
subthreshold and the strong inversion operating regions. It is
approximately equal to $V_{\rm{GS}} - V_{\rm{T}}$ in strong
inversion but becomes proportional to $\exp[q(V_{\rm{GS}} -
V_{\rm{T}})/nkT]$ in the
subthreshold region. \\

\begin{equation}
V_{\rm{GSTeff}} = {n{kT \over q} \ln(1 + \exp({m \over n}
{V_{\rm{GSeff}} - V_{\rm{T}} \over nkT/q })) \over m +
n{C_{\rm{oxe}} \over C_{\rm{dep0}}} \exp[-{(1 - m)(V_{\rm{GSeff}}
- V_{\rm{T}}) - V_{\rm{off}} \over nkT/q}]}
\end {equation}
\\
where
\begin{equation}
m = {1 \over 2} + {\arctan(\texttt{MINV}) \over \pi}
\end{equation}

\begin{equation}
V_{\rm{off}} = \texttt{VOFF} + {\texttt{VOFFL} \over L_{\rm{EFF}}}
\end{equation}

\begin{equation}
C_{\rm{dep0}} = {\epsilon_s \over X_{\rm{dep0}}}
\end{equation}

\begin{equation}
C_{\rm{dep}} = {\epsilon_s \over X_{\rm{dep}}}
\end{equation}

\noindent The ideality factor $n$ is
\begin{eqnarray}
n & = & 1 + \texttt{NFACTOR}\; {C_{\rm{dep}} \over C_{\rm{oxe}}}
{\texttt{CDSC} + \texttt{CDSCD}\;V_{\rm{DS}} +
\texttt{CDSCB}\;V_{\rm{BSeff}} \over C_{\rm{oxe}}} \\
    & & \times {0.5 \over \cosh(\texttt{DVT1} \; L_{\rm{EFF}}/L_t - 1)} + {\texttt{CIT} \over C_{\rm{oxe}}}
\end{eqnarray}
\\
As mentioned earlier, the $\Delta V_{\rm{T}}$ correction due to
pocket implant requires the knowledge of the ideality factor $n$.
Pocket implants near the source and drain regions increase the
drive currents. They also increase the drain conductance. The
pocket implant correction is modeled as
\begin{equation}
\Delta V_{\rm{T,PocketImplant}} = n {kT \over q} \ln
[{L_{\rm{EFF}} \over L_{\rm{EFF}} + \texttt{DVTP0}\;(1 +
\exp(-\texttt{DVTP1}\;V_{\rm{DS}}))}]
\end{equation}

\noindent{\it Mobility characteristics}:\\
The mobility equations are based on the universal-mobility theorem
which predicts the mobility degradation with increasing
$V_{\rm{GS}}$. As an electron moves along the channel due to the
lateral electric field, it is also attracted to the gate due to
the normal electric field. This causes the electron to drift
toward the gate and results in mobility degradation. This effect
is modeled in the equations for mobility. We first define a
variable $\texttt{TEMP}$.
\begin{equation}
\texttt{TEMP} = (\texttt{UA} + \texttt{UC}\;V_{\rm{BSeff}}) \;
({V_{\rm{GSTeff}} + V_{\rm{T-fb-\phi}} \over
\texttt{TOXE}})^\texttt{EU}
\end{equation}
where \texttt{EU} is set to zero if the user suppiled value is
negative and

\begin{equation}
V_{\rm{T-fb-\phi}} = \left\{\begin{array}{ll}
       2\;(\texttt{VTH0} - V_{\rm{FB}} - 2\phi_f) & \mbox{for NMOS} \\
       2.5\;(\texttt{VTH0} - V_{\rm{FB}} - 2\phi_f) & \mbox{for PMOS}
                            \end{array}
                    \right.
\end{equation}
\noindent The effective mobility of the model takes the
generalized form
\begin{equation}
\mu_{\rm{eff}} = {\texttt{U0} \over \texttt{Denom}}
\end{equation}

where
\begin{equation}
\texttt{Denom} = \left\{\begin{array}{ll}
       1 + \texttt{TEMP}                      & \mbox{if \texttt{TEMP}} \geq -0.8 \\
       {(0.6 + \texttt{TEMP}) / (7 + 10\;\texttt{TEMP})} & \mbox{if \texttt{TEMP}} < -0.8
                            \end{array}
                    \right.
\end{equation}

\noindent{\it Current characteristics}:\\
{\it Effective Internal Source-Drain Resistance}:\\
This model only models an internal drain to source resistance,
$R_{\rm{DS}}$ given by
\begin{equation}
R_{\rm{DS}} = {\texttt{RDSWMIN} + \texttt{RDSW} \times
0.5\;[\texttt{TEMP} + \sqrt{\texttt{TEMP}^2 + 0.01}] \over (10^6
\times W_{\rm{EFFCJ}})^{WR}}
\end{equation}

\noindent{\it Bulk-Charge coefficient}:\\
The bulk-charge coefficient should always have a real physical
value, i.e. greater than zero. BSIM4 ensures that this value is
always greater than zero. It is an intermediate variable that aids
in the evaluation of current and saturation voltage.

\begin{eqnarray}
A_{\rm{bulk}} & = & [1 - F_{\rm{doping}} \times
[{\texttt{A0}\;L_{\rm{EFF}} \over
L_{\rm{EFF}} + 2\sqrt{\texttt{XJ}\;X_{\rm{dep}}}} \nonumber \\
           & & (1 - \texttt{AGS}\;V_{\rm{GSTeff}}({L_{\rm{EFF}} \over
L_{\rm{EFF}} + 2\sqrt{\texttt{XJ}\;X_{\rm{dep}}}})^2) + {\texttt{B0} \over W_{\rm{EFF}} + \texttt{B1}}]] \nonumber \\
           & & \times {1 \over 1 + \texttt{KETA}\;V_{\rm{BSeff}}}
\end{eqnarray}
where
\begin{eqnarray}
F_{\rm{doping}} & = & \sqrt{1 + {\texttt{LPEB} \over
L_{\rm{EFF}}}} \times {\texttt{K1} \over 2\sqrt{2\phi_f -
V_{\rm{BSeff}}}} \;{\texttt{TOXE}
\over {\texttt{TOXM}}}  \nonumber \\
             & & + \texttt{K2}\;{\texttt{TOXE} \over \texttt{TOXM}} - \texttt{K3} \times
             {\texttt{TOXE} \over W_{\rm{EFF}} + \texttt{W0}}\; 2\;\phi_f
\end{eqnarray}

\noindent{\it Drain Saturation Voltage}:\\
The value of the drain voltage in the saturation region is given
by
\begin{equation}
\epsilon_{\rm{sat}} = {2\; \texttt{VSAT} \over \mu_{\rm{eff}}}
\end{equation}

\noindent If $R_{\rm{DS}} = 0$, then
\begin{equation}
V_{\rm{DSsat}} = {\epsilon_{\rm{sat}}\; L_{\rm{EFF}}\;
(V_{\rm{GSTeff}} + 2kT/q) \over A_{\rm{bulk}}\;
\epsilon_{\rm{sat}} \; L_{\rm{EFF}} + (V_{\rm{GSTeff}} + 2kT/q)}
\end{equation}
\noindent If $R_{\rm{DS}} \neq 0$, then,
\begin{equation}
V_{\rm{DSsat}} = {-b -\sqrt{b^2 - 4ac} \over 2a}
\end{equation}

\begin{equation}
a = A_{\rm{bulk}}^2 \; W_{\rm{EFF}} \;\texttt{VSAT} \;C_{\rm{oxe}}
\;R_{\rm{DS}} + ({1 \over \lambda} - 1) \;A_{\rm{bulk}}
\end{equation}

\begin{eqnarray}
b & = & - [(V_{\rm{GSTeff} + {2 kT/q}})\;({2 \over \lambda} - 1) + A_{\rm{bulk}}\; \epsilon_{\rm{sat}}\;L_{\rm{EFF}} \nonumber \\
    & & + 3 \;A_{\rm{bulk}}\;(V_{\rm{GSTeff} + {2 kT/q}})\;W_{\rm{EFF}}\;\texttt{VSAT}\;C_{\rm{oxe}}\;R_{\rm{DS}}]
\end{eqnarray}

\begin{equation}
c = (V_{\rm{GSTeff} + {2 kT/q}})\; \epsilon_{\rm{sat}} \;
L_{\rm{EFF}} + 2(V_{\rm{GSTeff} + {2 kT/q}})^{2} W_{\rm{EFF}} \;
\texttt{VSAT}\;C_{\rm{oxe}}\;R_{\rm{DS}}
\end{equation}

\noindent where $\lambda$ is determined by parameters \texttt{A1}
and \texttt{A2}.

\noindent{\it Effective Drain to Source Voltage}:\\
$V_{\rm{DSeff}}$ varies between 0 (when $V_{\rm{DS}} = 0$) and
$V_{\rm{DSsat}}$ in the saturation region, where $V_{\rm{DS}}$ is
fairly high. It is a smoothing function defined to smooth out the
transition between the linear and saturation regions. The
parameter \texttt{DELTA} can be varied between 0.1 and 0.001 to
get greater control of the transition between the linear and
saturation region.
\begin{eqnarray}
V_{\rm{DSeff}} & = & V_{\rm{DSsat}} - 0.5\;(V_{\rm{DSsat}} -
V_{\rm{DS}} - \texttt{DELTA}
\nonumber \\
            & & + \sqrt{(V_{\rm{DSsat} - V_{\rm{DS}} - \texttt{DELTA})^2 +
            4\;\texttt{DELTA}\;V_{\rm{DSsat}}}})
\end{eqnarray}

\noindent{\it Effective oxide calculation}:\\
The drain current, in BSIM4, has a significant amount of drain
current in the strong inversion region. It calculates the maximum
probability of carrier distribution occurs at a distance
$X_{\rm{DC}}$ away from the interface. The oxide capacitance for
the inversion calculation is given by
\begin{equation}
C_{\rm{ox}} = {\epsilon_{\rm{ox}} \over \texttt{TOXP}} \; \| \;
{\epsilon_s \over X_{\rm{DC}}}
\end{equation}

where
\begin{equation}
X_{\rm{DC}} = \left\{\begin{array}{ll}
     1.9 \times 10^{-9} &  \nonumber\\
     \times [1 + {V_{\rm{GSTeff}} + 4(\texttt{VTH0} - V_{\rm{FB}} - 2\phi_f) \over 2 \times 10^8\;\texttt{TOXP}}]^{-0.7} & (V_{\rm{TH0}} - V_{\rm{FB}} - 2\phi_f) \geq 0 \nonumber \\
     & \\
     1.9 \times 10^{-9} & \nonumber \\
     \times [1 + V_{\rm{GSTeff}} / (2 \times 10^8\;\texttt{TOXP})]^{-0.7} & (V_{\rm{TH0}} - V_{\rm{FB}} - 2\phi_f) < 0
                            \end{array}
                    \right.
\end{equation}

\noindent{\it Current Calculations}:\\
This is the dominant portion of drain current. It flows from the
drain to the source through the channel.
\begin{eqnarray}
I_{\rm{DS}}& = & {I_{\rm{DS0}} \over  1 + {R_{\rm{DS}}
I_{\rm{DS0}} \over V_{\rm{DSeff}}}} \; (1 + {1 \over C_{\rm{clm}}}
\ln{V_A \over V_{\rm{Asat}}}) \times (1 +
{V_{\rm{DS}} - V_{\rm{DSeff}} \over V_{\rm{ADIBL}}}) \nonumber \\
       &  & \times (1 + {V_{\rm{DS}} - V_{\rm{DSeff}} \over V_{\rm{ADITS}}}) \times (1 + {V_{\rm{DS}}
         - V_{\rm{DSeff}} \over V_{\rm{ASCBE}}}) \times \texttt{NF}
\end{eqnarray}

where the ideal long channel current in the absence of channel
length modulation and DIBL effects is given by
\begin{equation}
I_{\rm{DS0}} = {W_{\rm{EFF}} \;\mu_{\rm{eff}} \;C_{\rm{ox}}
\;V_{\rm{GSTeff}} \over L_{\rm{EFF}}\;[1 + V_{\rm{DSeff}} /
(\epsilon_{\rm{sat}} L_{\rm{EFF}})]} \; [1 -
{A_{\rm{bulk}}\;V_{\rm{DSeff}} \over 2(V_{\rm{GSTeff}} + 2kT/q)}]
\; V_{\rm{DSeff}}
\end{equation}
\noindent There Early voltage is given as
\begin{equation}
V_A = V_{\rm{Asat}} + V_{\rm{ACLM}}
\end{equation}
where $V_{\rm{Asat}}$ calculates the ideal early voltage in the
absence of short channel effects and $V_{\rm{ACLM}}$ models
channel length modulation.

\begin{eqnarray}
V_{\rm{Asat}} & = & {\epsilon_{\rm{sat}}\;L_{\rm{EFF}} +
V_{\rm{DSsat}} + 2 R_{\rm{DS}}\;\texttt{VSAT} \;C_{\rm{oxe}}
W_{\rm{EFF}}\; V_{\rm{GSTeff}} \over 2/\lambda - 1
+ R_{\rm{DS}}\;\texttt{VSAT}\; C_{\rm{oxe}}\; W_{\rm{EFF}}\; A_{\rm{bulk}}} \nonumber \\
           & & \times [1 - {A_{\rm{bulk}}\;V_{\rm{DSsat}} \over 2\;(V_{\rm{GSTeff}} + 2kT/q)}]
\end{eqnarray}
\noindent The degradation factor due to pocket implantation is
given by
\begin{equation}
F_p = \left\{\begin{array}{ll}
       (1 + \texttt{FPROUT}\;\sqrt{L_{\rm{EFF}}} / (V_{\rm{GSTeff}} + 2kT/q))^{-1} &  \texttt{FPROUT} > 0 \nonumber \\
        1 & \texttt{FPROUT} \leq 0
                            \end{array}
                    \right.
\end{equation}

\noindent To account for the effects of gate-bias on the slope of
$I_{\rm{DS}}$ in the saturation region, we use

\begin{equation}
F_{\rm{VG}} = \left\{\begin{array}{ll}
       1 + \texttt{PVAG} \; V_{\rm{GSTeff}} / (\epsilon_{\rm{sat}}\;L_{\rm{EFF}}) & \texttt{PVAG} \;V_{\rm{GSTeff}} / (\epsilon_{\rm{sat}}\;L_{\rm{EFF}}) > -0.9 \nonumber \\
       & \\
       0.8 + \texttt{PVAG}\;V_{\rm{GSTeff}} / (\epsilon_{\rm{sat}}\;L_{\rm{EFF}}) & \nonumber \\
       \times (17 + 20\;\texttt{PVAG}\;V_{\rm{GSTeff}} / (\epsilon_{\rm{sat}}\;L_{\rm{EFF}}))^{-1} & \\
                            \end{array}
                    \right.
\end{equation}

\begin{equation}
C_{\rm{clm}} = \left\{\begin{array}{ll}
       \texttt{TEMP} & \texttt{PCLM} > 0 \; and \; (V_{\rm{DS}} - V_{\rm{DSeff}}) > 10^{-10} \\
        5.834617425 \times 10^{14} &
                            \end{array}
                    \right.
\end{equation}
\noindent where
\begin{equation}
\texttt{TEMP} = {F_p \over \texttt{PCLM} \;L_{\rm{itl}}}
F_{\rm{VG}} \times (1 + {R_{\rm{DS}} \;I_{\rm{DS0}} \over
V_{\rm{DSeff}}}) \;\times (L_{\rm{EFF}} + {V_{\rm{DSsat}} \over
\epsilon_{\rm{sat}}})
\end{equation}

\begin{eqnarray}
\theta_{\rm{rout}} & = & \texttt{PDIBLC1} \times
[\exp(-\texttt{DROUT}\; L_{\rm{EFF}}/2L_{\rm{t0}}) +
2\exp(-\texttt{DROUT}\;L_{\rm{EFF}}/2L_{\rm{t0}})] \nonumber \\
                & & + \texttt{PDIBLC2}
\end{eqnarray}

\noindent The effect of DIBL on Early voltage is modeled as:
\begin{equation}
V_{\rm{ADIBL}} = \left\{\begin{array}{ll}
       V_{\rm{GSTeff}} + 2kT/q / (\theta_{\rm{rout}}(1 + \texttt{PDIBLCB}\;V_{\rm{BSeff}})) \times F_{\rm{VG}} &  \\
       & \\
       \times [{1 - A_{\rm{bulk}} V_{\rm{DSsat}} / (A_{\rm{bulk}} V_{\rm{DSsat}} + V_{\rm{GSTeff}} + 2kT/q)}]& \theta_{\rm{rout}} \geq 0 \\
       & \\
       5.834617425 \times 10^{10} & \theta_{\rm{rout}} < 0
                            \end{array}
                    \right.
\end{equation}

\noindent The effect of DITS (Drain-induced Threshold Shift) due
to pocket implant is modeled by
\begin{equation}
V_{\rm{ADITS}} = \left\{\begin{array}{ll}
       {F_p \over \texttt{PDITS}} [1 + (1 + \texttt{PDITSL}\;L_{\rm{EFF}}) \times \exp(\texttt{PDITSD}\;V_{\rm{DS}})] & \texttt{PDITS} > 0 \\
                & \\
        5.834617425 \times 10^{14} & else
                            \end{array}
                    \right.
\end{equation}

\noindent Substrate current has an effect on Early voltage and it
is modeled by SCBE (Substrate Current Induced Body Effect).
\begin{equation}
V_{\rm{ASCBE}} = \left\{\begin{array}{ll}
       {L_{\rm{EFF}} \over \texttt{PSCBE2}} \; \exp(\texttt{PSCBE2} \;L_{\rm{itl}} / (V_{\rm{DS}} - V_{\rm{DSeff}})) & \texttt{PSCBE2} > 0 \\
       & \\
       5.834617425 \times 10^{10} & else
                            \end{array}
                    \right.
\end{equation}

\noindent where

\begin{equation}
L_{\rm{itl}} = \sqrt{\epsilon_{s}/\epsilon_{\rm{ox}} \times
\texttt{TOXE}\;\texttt{XJ}}
\end{equation}

\noindent{\it Substrate Currents}:\\
The substrate currents comprise of two parts, one due to Impact
Ionization and the other due to GIDL. When $V_{\rm{DS}}$ is high,
a large voltage is dropped across the depletion region near the
drain. This field accelerates the electrons as they are moving in
the channel. When they generate sufficient energy, the collide
with the semiconductor crystal and generate electron-hole pairs.
This current that is generated flows towards the substrate. This
forms the Impact Ionization current and is denoted by
$I_{\rm{sub}}$.
\begin{eqnarray}
I_{\rm{sub}} & = & \texttt{NF} \times ({\texttt{ALPHA0} \over
L_{\rm{EFF}}} + \texttt{ALPHA1})
(V_{\rm{DS}} - V_{\rm{DSeff}}) \exp[-{\texttt{BETA0} \over V_{\rm{DS}} - V_{\rm{DSeff}}}] \nonumber \\
         & & {I_{\rm{DS0}} \over  1 + {R_{\rm{DS}} I_{\rm{DS0}} \over V_{\rm{DSeff}}}} \; (1 + {1 \over
C_{\rm{clm}}} \ln{V_A \over V_{\rm{Asat}}}) \times (1 +
{V_{\rm{DS}} - V_{\rm{DSeff}} \over V_{\rm{ADIBL}}}) \nonumber \\
         & & \times (1 + {V_{\rm{DS}} - V_{\rm{DSeff}} \over V_{\rm{ADITS}}})
\end{eqnarray}
The contribution due to GIDL is given by
\begin{eqnarray}
I_{\rm{gidl}}& = & \texttt{NF} \times \texttt{AGIDL}\;
W_{\rm{EFFCJ}} [{V_{\rm{DS}} - V_{\rm{GSeff}} -\
               \texttt{EGIDL} \over 3\;\texttt{TOXE}}] \nonumber \\
  & & \times \exp({-3\;\texttt{TOXE} \times \texttt{BGIDL} \over V_{\rm{DS}} - V_{\rm{GSeff}} -
  \texttt{EGIDL}}) \nonumber \\
  & & {V^3_{\rm{DB}} \over \texttt{CGIDL} + V^3_{\rm{DB}}} \nonumber \\
\end{eqnarray}

{\it Gate Currents}:\\
\begin{figure}
\epsfxsize=3in\centerline{\epsfbox{figures/gate_currents.eps}}
\caption{\label{gate.currents}Schematic of the gate current
distribution in the FET}
\end{figure}
As the oxide layer becomes progressively thinner, the tunnelling
currents flowing through the oxide become more significant. This
model considers four tunnelling
currents as shown in Fig ~\ref{gate.currents}. \\
\indent $I_{\rm{gd}}$ is the tunnelling current between the gate
and the heavily-doped drain. $I_{\rm{gcd}}$ denotes the current
that flows from the gate to the channel and then to the drain.
Likewise, $I_{\rm{gs}}$ and $I_{\rm{gcs}}$ are similar tunnelling
currents, but associated with the source junction. The current
$I_{gb}$ represents the current flowing from the gate to the bulk.\\
\noindent The voltage drop across the oxide is given by
\begin{equation}
V_{\rm{ox}} = V_{\rm{FB}} - V_{\rm{FBeff}} + \texttt{K1} \;
\sqrt{\phi_s} + V_{\rm{GSTeff}}
\end{equation}
The first two terms of the above equation represent voltage
dropped in the accumulation region or $V_{oxacc}$ and the
depletion/inversion region or $V_{\rm{oxdepinv}}$. The two channel
tunnelling components are given by
\begin{equation}
I_{\rm{gcs}} = I_{\rm{gc}} \times {-1 + \texttt{PIGCD}\;
V_{\rm{DS}} + \exp(-\texttt{PIGCD}\; V_{\rm{DS}} + 10^{-4}) \over
(\texttt{PIGCD}\; V_{\rm{DS}})^2 + 2 \times 10^{-4}}
\end{equation}
\begin{equation}
I_{\rm{gcd}} = I_{\rm{gc}} \times {1 - (1 +
\texttt{PIGCD}\;V_{\rm{DS}}) + \exp(-\texttt{PIGCD}\;V_{\rm{DS}} +
10^{-4}) \over (\texttt{PIGCD}\;V_{\rm{DS}})^2 + 2 \times 10^{-4}}
\end{equation}
Both these currents have dependencies on the drain-source voltage
$V_{\rm{DS}}$. Generally, these currents do not sum up to
$I_{\rm{gc}}$. However, when $V_{\rm{DS}}$ is zero, they are
identical to each other and equal to half of $I_{\rm{gc}}$, which
is given by
\begin{eqnarray}
I_{\rm{gc}}& = & \texttt{NF} \; W_{\rm{EFF}} \; L_{\rm{EFF}} {A
\over (\texttt{TOXE})^2}
({\texttt{TOXREF} \over \texttt{TOXE}})^{\texttt{NTOX}} \; V_{\rm{GSeff}} \nonumber \\
  & & \times \texttt{NIGC} {kT\over q} \ln [1 + \exp({q(V_{\rm{GSeff}} - \texttt{VTH0}) \over
  kT.\texttt{NIGC}})] \nonumber \\
  & & \exp[-B . \texttt{TOXE}(\texttt{AIGC} - \texttt{BIGC}V_{\rm{oxdepinv}}).(1 +
  \texttt{CIGC}V_{\rm{oxdepinv}})]
\end{eqnarray}
The coefficients used in the above equations are
\begin{eqnarray}
A & = & 4.97232 \times 10^{-7} \\
B & = & 7.45669 \times 10^{11}
\end{eqnarray}
The currents associated with the gate and source/drain regions is
given by
\begin{eqnarray}
I_{\rm{gs}}& = & \texttt{NF} \; W_{\rm{EFF}} \; \texttt{DLCIG} {A
\over (\texttt{TOXE POXEDGE})^2}({\texttt{TOXREF} \over
\texttt{TOXE POXEDGE}})^{\texttt{NTOX}} \; V_{\rm{GS}} \times V^{\prime}_{\rm{GS}} \nonumber \\
  & & \exp[-B\;\texttt{TOXE POXEDGE}(\texttt{AIGSD} - \texttt{BIGSD}\;V_{\rm{GS}}) \nonumber \\
  & & (1 + \texttt{CIGSD} \; V^{\prime}_{\rm{GS}})]
\end{eqnarray}
\begin{equation}
V^{\prime}_{\rm{GS}} = \sqrt{(V_{\rm{GS}} - V_{\rm{fbsd}})^2 +
10^{-4}}
\end{equation}

\begin{eqnarray}
I_{\rm{gd}} & = & \texttt{NF} \; W_{\rm{EFF}} \; \texttt{DLCIG} {A
\over (\texttt{TOXE POXEDGE})^2}({\texttt{TOXREF} \over
\texttt{TOXE POXEDGE}})^{\texttt{NTOX}} \; V_{\rm{GD}} \times V^{\prime}_{\rm{GD}} \nonumber \\
  & & \exp[-B\;\texttt{TOXE POXEDGE}(\texttt{AIGSD} - \texttt{BIGSD}\;V_{\rm{GD}}) \nonumber \\
  & &(1 + \texttt{CIGSD} \; V^{\prime}_{\rm{GD}})]
\end{eqnarray}
\begin{equation}
V^{\prime}_{\rm{GD}} = \sqrt{(V_{\rm{GD}} - V_{\rm{fbsd}})^2 +
10^{-4}}
\end{equation}

The resultant dc equivalent circuit for the transistor is shown in
Figure ~\ref{mos.equivalent}
\begin{figure}
\epsfxsize=3in\centerline{\epsfbox{figures/mos_eq.eps}} \caption{Schematic of
the dc equivalent circuit\label{mos.equivalent}}
\end{figure}

\noindent{\it Charge computation and Conservation}:\\
{\it Basic Formulation}:\\
To ensure charge conservation, terminal charges are used as state
variables along with terminal voltages. $Q_g$, $Q_s$, $Q_d$ and
$Q_b$ are the charges associated with the gate, source, drain and
bulk terminals respectively. The gate charge comprises of the
inversion charge $Q_{\rm{inv}}$, the accumulation charge
$Q_{\rm{acc}}$ and the substrate depletion charge $Q_{\rm{sub}}$. \\
\indent The channel charge comes from the source and drain
terminals while the accumulation and substrate charge is
associated with the substrate.
\begin{eqnarray}
Q_g & = & -(Q_{\rm{sub}} + Q_{\rm{inv}} + Q_{\rm{acc}}) \nonumber \\
Q_b & = & Q_{\rm{acc}} + Q_{\rm{sub}} \nonumber \\
Q_{\rm{inv}} & = & Q_d + Q_s
\end{eqnarray}
The substrate charge can be divided further into two components:
the substrate charge at zero source-drain bias ($Q_{\rm{sub0}}$)
and a non-uniform substrate charge in the presence of a drain bias
($\delta Q_{\rm{sub}}$). The gate charge now becomes
\begin{equation}
Q_g = -(Q_{\rm{sub0}} + \delta Q_{\rm{sub}} + Q_{\rm{inv}} +
Q_{\rm{acc}})
\end{equation}
The total charge is computed by integrating the charge along the
channel. SPICE3 provides three options in BSIM4 whereby a user can
select the percentage of charge distribution between the source
and drain. The options are a 0/100 distribution which implies that
no channel charge is associated with the source and it is all
assigned to the drain, a 50/50 partition which divides the charges
equally between the source and drain and a 40/60 partition wherein
the total charge in the channel is divided in a 40:60 ratio
between the source and drain. \FDA uses a 40/60 charge
partitioning scheme between the source and drain terminals because
that is the closest to a physical situation in the channel.

\noindent{\it Effective $V_{\rm{BS}}$, $V_{GB}$}:\\
This is a smoothing function required for C-V calculations.
\begin{equation}
V_{\rm{BSeffCV}} = \left\{\begin{array}{ll}
       V_{\rm{BSeff}} &  V_{\rm{BSeff}} < 0 \nonumber \\
       \phi_s - {\phi_s \over \phi_s + V_{\rm{BSeff}}} & V_{\rm{BSeff}} \geq 0
                            \end{array}
                    \right.
\end{equation}

\begin{equation}
V_{\rm{GBeffCV}} = V_{gse} - V_{\rm{BSeffCV}}
\end{equation}

\noindent{\it Effective $V_{\rm{GS}} - V_{\rm{T}}$}:\\
This is also a smoothing function used in the C-V calculation.
\begin{equation}
V_{\rm{GSTeffCV}} = \texttt{NOFF}\;{nkT \over q}\; \ln[1 +
\exp({V_{\rm{GSeff}} - V_{\rm{T}} - \texttt{VOFFCV} \over
\texttt{NOFF} \;nkT/q})]
\end{equation}

\noindent{\it Modified Bulk Charge coefficient}:\\
For C-V calculations, the reduced bulk charge coefficient is used
instead of the coefficient used in the DC calculations.
\begin{eqnarray}
A_{\rm{bulk0}} & = & [1 - F_{\rm{doping}} \times
[{\texttt{A0}\;L_{\rm{EFF}} \over
L_{\rm{EFF}} + 2\sqrt{\texttt{XJ}\;X_{\rm{dep}}}} + {\texttt{B0} \over W_{\rm{EFF}} + \texttt{B1}}]] \nonumber \\
           & & \times {1 \over 1 + \texttt{KETA}\;V_{\rm{BSeff}}}
\end{eqnarray}
Using this value of reduced bulk-charge coefficient, the
bulk-charge coefficient for C-V calculations is given by
\begin{equation}
A_{\rm{bulkCV}} = A_{\rm{bulk0}}[1 + ({\texttt{CLC} \over
L_{\rm{EFF}}})^{\texttt{CLE}}]
\end{equation}

\noindent{\it The Terminal Charges}:\\
The effective oxide thickness
\begin{equation}
C^{\prime}_{\rm{oxeff}}= {\epsilon_{\rm{ox}} \over \texttt{TOXP}}
\| {\epsilon_s \over X_{\rm{DCeff}}}
\end{equation}
where
\begin{eqnarray}
X_{\rm{DCeff}} & = & X_{\rm{DCmax}}  \\
& &- {X_{\rm{DCmax}} - X_{\rm{DC}} - \delta_x +
\sqrt{(X_{\rm{DCmax}} - X_{DC} - \delta_x )^2 + 4 \; \delta_x \;
X_{\rm{DCmax}}} \over 2} \nonumber
\end{eqnarray}
The various terms inside the above equation are given by
\begin{equation}
X_{\rm{DC}} = {L_{\rm{Deb}} \over 3}
\exp[\texttt{ACDE}({\texttt{NDEP} \over 2 \times 10^{16}})^{-0.25}
\; {V_{\rm{GBeff}} - V_{\rm{fbzb}}\over 10^8 \times
\texttt{TOXP}}]
\end{equation}
\begin{equation}
L_{\rm{Deb}} = \sqrt{{\epsilon_s \;k [T + 273.15]/q \over q
\texttt{NDEP} 10^6}}
\end{equation}
\begin{equation}
X_{\rm{DCmax}} = {L_{\rm{Deb}} \over 3}
\end{equation}
\begin{equation}
\delta_x = 10^{-3} \texttt{TOXP}
\end{equation}
The effective oxide is re-calculated for evaluation of the
accumulation charge.
\begin{equation}
C_{\rm{oxeff}} = C^{\prime}_{\rm{oxeff}} \times W_{\rm{EFF}}
\times L_{\rm{EFF}} \times \texttt{NF}
\end{equation}
The accumulation charge is given by
\begin{equation}
Q_{\rm{acc}} = C_{\rm{oxeff}}\;(V_{\rm{FBeffCV}} - V_{\rm{FBCV}})
\end{equation}
The substrate charge is given by
\begin{equation}
Q_{\rm{sub0}}= C_{\rm{oxeff}}\; (\texttt{K1} {\texttt{TOXE} \over
\texttt{TOXM}}) \; \sqrt{\phi_{\rm{s,dep}}}
\end{equation}
When the transistor enters the sub-threshold region, another value
of $X_{\rm{DC}}$ is required. This value is used in the evaluation
of $Q_{\rm{inv}}$ and $\delta Q_{\rm{sub}}$.
\begin{equation}
C^{\prime}_{\rm{oxinv}}= {\epsilon_{\rm{ox}} \over \texttt{TOXP}}
\| {\epsilon_s \over X_{\rm{DCinv}}}
\end{equation}
where
\begin{equation}
X_{\rm{DCinv}} = \left\{\begin{array}{ll}
     1.9 \times 10^{-9} &  \nonumber\\
     \times [1 + {V_{\rm{GSTeffCV}} + 4(\texttt{VTH0} - V_{\rm{FB}} - 2\phi_f) \over 2 \times 10^8\;\texttt{TOXP}}]^{-0.7} & (V_{\rm{TH0}} - V_{\rm{FB}} - 2\phi_f) \geq 0 \nonumber \\
     & \\
     1.9 \times 10^{-9} & \nonumber \\
     \times [1 + V_{\rm{GSTeffCV}} / (2 \times 10^8\;\texttt{TOXP})]^{-0.7} & (V_{\rm{TH0}} - V_{\rm{FB}} - 2\phi_f) < 0
                            \end{array}
                    \right.
\end{equation}
The above equation is identical to $X_{\rm{DC}}$ used for I-V
calculations, except that $V_{\rm{GSTeff}}$ is replaced by
$V_{\rm{GSTeffCV}}$.
\begin{equation}
C_{\rm{oxinv}} = C^{\prime}_{\rm{oxinv}} \times W_{\rm{EFF}}
\times L_{\rm{EFF}} \times \texttt{NF}
\end{equation}
In addition to the new $X_{\rm{DC}}$, the surface potential is not
constant as in the I-V case and needs to be re-calculated.
\begin{equation}
\phi_{\delta} = \left\{\begin{array}{ll}
       \frac{k T}{q} \ln[{1 + V_{\rm{GSTeffCV}}(V_{\rm{GSTeffCV}} + 2\texttt{K1}\;(\texttt{TOXE} / \texttt{TOXM})\;(\phi_s))}] & \nonumber\\
       - \frac{k T}{q} \; \ln[\texttt{MOIN}\; \texttt{K1}^2 (\texttt{TOXE}/\texttt{TOXM})^2 \;(kT/q)] & \texttt{K1} > 0 \nonumber \\
       & \\
       {k T \over q} \ln[1 + V_{\rm{GSTeffCV}}(V_{\rm{GSTeffCV}} + \sqrt{\phi_s}) / (0.25 \times \texttt{MOIN} (k T /q))] & \texttt{K1} \leq 0 \\
                            \end{array}
                    \right.
\end{equation}
\begin{equation}
V_{\rm{DSsatCV}} = {V_{\rm{GSTeffCV}} - \phi_{\delta} \over
A_{\rm{bulkCV}}}
\end{equation}

\begin{eqnarray}
V_{\rm{DSeffCV}}& = & V_{\rm{DSsatCV}} - {V_{\rm{DSsatCV}} - V_{\rm{DS}} - 0.02 \over 2} \nonumber \\
            & & - \sqrt{(V_{\rm{DSsatCV}} - V_{\rm{DS}} - 0.02)^2 + 4\; 0.02\; V_{\rm{DSsatCV}}\over 2}
\end{eqnarray}

\noindent Based on these calculations, the inversion charge can be
written as
\begin{eqnarray}
Q_{\rm{inv}}& = & - C_{\rm{oxinv}} [(V_{\rm{GSTeffCV}} -
\phi_{\delta} - {A_{\rm{bulkCV}} V_{\rm{DSeffCV}}\over
2} ) \nonumber \\
        &  & {A^2_{\rm{bulkCV}} V^2_{\rm{DSeffCV}} \over 12(V_{\rm{GSTeffCV}} - \phi_{\delta} - A_{\rm{bulkCV}} V_{\rm{DSeffCV}}/2 + 10^{-20})}]
\end{eqnarray}
The factor of $10^{-20}$ exists to mainly prevent the denominator
from going to a
negative value when the rest of the terms go close to zero.\\
The substrate charge in the presence of a drain bias is given by
\begin{eqnarray}
\delta Q_{\rm{sub}} & = & C_{\rm{oxinv}}[{1 - A_{\rm{bulkCV}} \over 2}V_{\rm{DSeffCV}} \\
       & & - {(1 - A_{\rm{bulkCV}})\;A_{\rm{bulkCV}} \;V^2_{\rm{DSeffCV}} \over 12\,(V_{\rm{GSTeffCV}} - \phi_{\delta} - A_{\rm{bulkCV}}\; V_{\rm{DSeffCV}}/2 + 10^{-20})}]
\end{eqnarray}

\noindent Finally, the four charges at the respective terminals
are given by
\begin{equation}
Q_g = -Q_{\rm{inv}} - \delta Q_{\rm{sub}} + Q_{\rm{acc}} +
Q_{\rm{sub0}}
\end{equation}
\begin{equation}
Q_b = \delta Q_{\rm{sub}} - Q_{\rm{acc}} - Q_{\rm{sub0}}
\end{equation}
For a 40/60 charge partition scheme, the charge at the source and
drain regions is
\begin{eqnarray}
Q_s & = & - {C_{\rm{oxinv}} \over 2(V_{\rm{GSTeffCV}} -
\phi_{\delta} - {A_{\rm{bulkCV}} V_{\rm{DSeffCV}}
\over 2})^2} \nonumber \\
  & & \times [ (V_{\rm{GSTeffCV}} - \phi_{\delta})^3 \nonumber \\
  & & - {4 \over 3}(V_{\rm{GSTeffCV}} - \phi_{\delta})^2 A_{\rm{bulkCV}} V_{\rm{DSeffCV}} \nonumber \\
  & & + {2 \over 3}(V_{\rm{GSTeffCV}} - \phi_{\delta})^2 A_{\rm{bulkCV}} V_{\rm{DSeffCV}} \nonumber \\
  & & - {2 \over 15} A^3_{\rm{bulkCV}} V^3_{\rm{DSeffCV}}
\end{eqnarray}

\begin{eqnarray}
Q_d & = & - {C_{\rm{oxinv}} \over 2(V_{\rm{GSTeffCV}} -
\phi_{\delta} - {A_{\rm{bulkCV}} V_{\rm{DSeffCV}}
\over 2})^2} \nonumber \\
 & & \times [ (V_{\rm{GSTeffCV}} - \phi_{\delta})^3 \nonumber \\
 & & - {5 \over 3}(V_{\rm{GSTeffCV}} - \phi_{\delta})^2 A_{\rm{bulkCV}} V_{\rm{DSeffCV}} \nonumber \\
 & & + (V_{\rm{GSTeffCV}} - \phi_{\delta})^2 A_{\rm{bulkCV}} V_{\rm{DSeffCV}} \nonumber \\
 & & - {1 \over 5} A^3_{\rm{bulkCV}} V^3_{\rm{DSeffCV}}
\end{eqnarray}

\noindent The net currents at the gate, source and drain is given
by
\begin{eqnarray}
i_G(t)& = & I_{\rm{gcs}} + I_{\rm{gcd}} + I_{\rm{gs}} + I_{\rm{gd}} + {dQ_g \over dt} \\
i_S(t)& = & -I_{\rm{DS}} - I_{\rm{gs}} - I_{\rm{gcs}} + {dQ_s \over dt} \\
i_D(t)& = & I_{\rm{DS}} + I_{\rm{sub}} + I_{\rm{gidl}} -
I_{\rm{gcd}} - I_{\rm{gd}} + {dQ_d \over dt}
\end{eqnarray}
\noindent\linethickness{0.5mm}\line(1,0){425}
\newline
\textit{Notes:}\\
There is no equivalent SPICE element.\\
\linethickness{0.5mm} \line(1,0){425}
\newline
\textit{Version:}\\
2002.08.01 \\
% Credits
\linethickness{0.5mm} \line(1,0){425}
\newline
\textit{Credits:}\\
\begin{tabular}{l l l l}
Name & Affiliation & Date & Links \\
Nikhil Kriplani & NC State University & August 2002 & \epsfxsize=1in\epsfbox{figures/logo.eps} \\
nmkripla@unity.ncsu.edu & & & www.ncsu.edu    \\
\end{tabular}
%Publications
\newline
\noindent \linethickness{0.5mm} \line(1,0){425}
\newline
\textit{Publications:}
\begin{enumerate}
\item M. B. Steer, C. Christoffersen, S. Velu and N. Kriplani, ``Global Modeling of
RF and Microwave Circuits,'' Mediterranean Microwave Conf. Digest, June 2002.
\end{enumerate}
\end{document}
