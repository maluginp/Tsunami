\documentclass{article}
\usepackage{epsf}\usepackage{here}
\usepackage{graphicx}
\oddsidemargin 0.25in \evensidemargin 0.25in
\topmargin 0.0in
\textwidth 6.5in \textheight 8.5in
\headheight 0.18in \footskip 0.16in
\leftmargin -0.5in \rightmargin -0.5in

%
% KEYWORD
%
\newcommand{\keywordtable}[1]{
        \sloppy
        \hyphenation{ca-pac-i-t-an-ce}
        \begin{center}
    \sf
        \begin{tabular}[t]
        {|p{0.58in}|p{3.07in}|p{0.55in}|p{0.60in}|}
        \hline
        \multicolumn{1}{|c}{\bf Name} &
        \multicolumn{1}{|c}{\parbox{2.77in}{\bf Description}}  &
        \multicolumn{1}{|c}{\bf Units} &
        \multicolumn{1}{|c|}{\bf Default} \X
        #1
        \end{tabular}
        \end{center}
    }

\newcommand{\keywordtwotable}[2]{
        \sloppy
        \hyphenation{ca-pac-i-t-an-ce}
        \begin{center}
    \sf
        \begin{tabular}[t]
        {|p{0.58in}|p{2.38in}|p{0.55in}|p{0.60in}|p{0.53in}|}
        \hline
        \multicolumn{1}{|c}{\bf Name} &
        \multicolumn{1}{|c}{\parbox{2.20in}{\bf Description}}  &
        \multicolumn{1}{|c}{\bf Units} &
        \multicolumn{1}{|c}{\bf Default} &
        \multicolumn{1}{|c|}{\bf #1} \X
        #2
        \end{tabular}
        \end{center}
    }

\newcommand{\kw}[2]{
     \samepage{
     \noindent {\sl #1} \vspace{-0.5in} \\
     \keywordtable{#2} }}

\newcommand{\kwtwo}[3]{
     \samepage{
     \noindent {\sl #1} \vspace{-0.4in} \\
     \keywordtwotable{#2}{#3} }}

\newcommand{\keyword}[1]{\kw{Keywords:}{#1}}
\newcommand{\keywordtwo}[2]{\kwtwo{Keywords:}{#1}{#2}}
\newcommand{\modelkeyword}[1]{\kw{Model Keywords}{#1}}
\newcommand{\modelkeywordtwo}[2]{\kwtwo{Model Keywords}{#1}{#2}}

\newcommand{\myline}{\\[-0.1in]
\noindent \rule{\textwidth}{0.01in} \newline}

\newcommand{\myThickLine}{\\[-0.1in]
\noindent \rule{\textwidth}{0.02in} \newline}


% FORM
\newcommand{\form}[1]{\samepage{\noindent
 {\sl Form} \myline
% \hspace*{\fill} % For some reason \fill = 0 when \pspiceform{} is used?
\offset
\it  \offsetparbox{#1}}
\\[0.1in]}

% ELEMENT FORM
\newcommand{\elementform}[1]{\samepage{\noindent
 {\sl Element Form} \myline
% \hspace*{\fill} % For some reason \fill = 0 when \pspiceform{} is used?
\offset
\it  \offsetparbox{#1}}
\\[0.1in]}

% MODEL FORM
\newcommand{\modelform}[1]{\samepage{\noindent
 {\sl Model Form} \myline
% \hspace*{\fill} % For some reason \fill = 0 when \pspiceform{} is used?
\offset
\it  \offsetparbox{#1}}
\\[0.1in]}

% LIMITS
\newcommand{\mylimits}[1]{\samepage{\noindent
 {\sl Limits} \myline
 \hspace*{\fill} \it  \offsetparbox{#1}}
 \vshift}

% EXAMPLE
\newcommand{\example}[1]{\samepage{\noindent
{\sl Example} \myline
\offset \tt  \offsetparbox{#1}}
 \vshift}

% PSPICE88 EXAMPLE
\newcommand{\pspiceexample}[1]{\samepage{\noindent
{\sl \pspice\ Example} \myline
\offset \tt  \offsetparbox{#1}}
 \vshift}

% MODEL TYPES
\newcommand{\modeltype}[1]{\samepage{\noindent
{\sl Model Type} \myline
 \hspace*{\fill} \tt \offsetparbox{#1}}
 \\[0.1in]}

% MODEL TYPES
\newcommand{\modeltypes}[1]{\samepage{\noindent
{\sl Model Types:} \myline
 \hspace*{\fill} \tt \offsetparbox{\tt #1}}
 \vshift}

% OFFSET ENUMERATE
\newcommand{\offsetenumerate}[1]{
     \offset \hspace*{-0.1in} {\begin{enumerate} #1 \end{enumerate}}}

% NOTE
\newcommand{\note}[1]{
\vshift\samepage{\noindent {\sl Note}\myline\vspace{-0.24in}}
 \offsetenumerate{#1} }

% SPECIAL NOTE
\newcommand{\specialnote}[2]{
\vshift\samepage{\noindent {\sl #1}\myline\vspace{-0.24in}}\\#2}

\newcommand{\dc}{\mbox{\tt DC}}
\newcommand{\ac}{\mbox{\tt AC}}
\newcommand{\SPICE}{\mbox{\tt SPICE}}
\newcommand{\m}[1]{{\bf #1}}                           % matrix command  \m{}

% ////// Changing nodes to terminals///////
% print terminals in \tt and enclose in a circle use outside
\newcommand{\terminal}[1]{\: \mbox{\tt #1} \!\!\!\! \bigcirc }
%
% set up environment for example
%
\newcounter{excount}
\newcounter{dummy}
\newenvironment{eg}{\vspace{0.1in}\noindent\rule{\textwidth}{.5mm}
   \begin{list}
   {{\addtocounter{excount}{1}
   \em Example\/ \arabic{chapter}.\arabic{excount}\/}:}
   {\usecounter{dummy}
   \setlength{\rightmargin}{\leftmargin}}
   }{\end{list} \rule{\textwidth}{.5mm}\vspace{0.1in}}
%
% set up environment for block
% currently this draws a horizontal line at the start of block and another
% at the end of block.
%
\newenvironment{block}{\vspace{0.1in}\noindent\rule{\textwidth}{.5mm}
   }{\rule{\textwidth}{.5mm}\vspace{0.1in}}
%


%
% set up wide descriptive list
%
\newenvironment{widelist}
    {\begin{list}{}{\setlength{\rightmargin}{0in} \setlength{\itemsep}{0.1in}
    \setlength{\labelwidth}{0.95in} \setlength{\labelsep}{0.1in}
\setlength{\listparindent}{0in} \setlength{\parsep}{0in}
    \setlength{\leftmargin}{1.0in}}
    }{\end{list}}

\newcommand{\STAR}{\hspace*{\fill} * \hspace*{\fill}}

\newcommand{\sym}[1]{\hspace*{\fill} ($#1$)}

\newcommand{\optionitem}[2]{
\item[{\tt #1}{#2}]\label{.OPTION#1}\index{.OPTIONS, #1}\index{#1}}

\newcommand{\error}[1]{\vspace{0.1in}\noindent{\tt #1}\\}


\begin{document}
\noindent{\LARGE\textbf{MESFET Materka-Kacprzac Model}
\hspace{\fill}\textbf{mesfetm}}\\
\hrulefill\linethickness{0.5mm}\line(1,0){425}
\normalsize
\newline
% the mesfet figure
\begin{figure}[h]
\centerline{\epsfxsize=1.5in\epsfbox{figures/b.ps}} \caption{B --- MESFET
element.}
\end{figure}
\newline
\linethickness{0.5mm}\line(1,0){425}
\newline
\textit{\FDA Form:}
%\newline
%\newline
$\tt mesfetm $:$\langle \tt{instance\ name}\rangle$ $n_1\ n_2\
n_3\ $ $\langle \tt{parameter\ list}\rangle$
\newline
\begin{tabular}{r l}
$n_1$ & is the drain node \\
&  \\
$n_2$ & is the gate node \\
&  \\
$n_3$ & is the source node \\
& \\
parameter list & see table 1 for parameter list
\end{tabular}\\
\newline
%\vspace{2mm}
% explanation of terms in the SPICE form
\newline
\begin{tabular}{r l}
$n_1$ & is the drain node \\
& \\
$n_2$ & is the gate node \\
&  \\
$n_3$ & is the source node \\
& \\
\textit{ModelName} & is the model name \\
&  \\
\textit{Area} & is the optional relative area factor. (Units: none; Default: 1; Symbol:Area)\\
\end{tabular}
% example in TRANSIM
\newline
%\vspace{2mm}
\linethickness{0.5mm}\line(1,0){425}
\newline
\textit{Example:}\\
\newline
%\newline
In \textit{\FDA:}\\
\texttt{mesfetm:b1\ 2\ 0\ 3}
\newline
% example in SPICE
%\vspace{4mm}
\newline
In \texttt{SPICE:}
\newline
\texttt{B1 \ 3 \ 7 \ 4 GAAS12 0.5}
\newline
% The Materka-Kacprzac model
%\vspace{6mm}
\newpage
\noindent\textbf{\large{$\amalg$ The Materka-Kacprzac Model}}
\normalsize
%\vspace{6mm}
\newline
\linethickness{0.5mm}\line(1,0){425}
\newline
% parameters
\textit{Model Parameters:}
\newline
\newline
\begin{tabular}{|r|l|c|c|}
\hline
\textbf{Name} & \textbf{Description} & \textbf{Units} & \textbf{Default} \\
\hline
\texttt{AFAB} & Slope factor of breakdown current ($AFAB$) & $1/V$ & 0.0 \\
\hline
\texttt{AFAG} & Slope factor of gate conduction current ($AFAG$) & $1/V$ & 38.696 \\
\hline
\texttt{AREA} & Area Multiplier ($AREA$) & - & 1.0 \\
\hline
\texttt{C10} & Gate source Schottky barrier capacitance for ($C_{10}$) & F & 0.0 \\
\hline
\texttt{CFO} & Gate drain feedback capacitance for ($C_{F0}$) & F & 0.0 \\
\hline
\texttt{CLS} & Constant parasitic component of gate-source capacitance ($C_{LS}$) & F & 0.0 \\
\hline
\texttt{E} & Constant part of power law parameter ($E$) & - & 2.0 \\
\hline
\texttt{GAMA} & Voltage slope parameter of pinch-off voltage ($\gamma$) & $1/V$ & 0.0 \\
\hline
\texttt{IDSS} & Drain saturation current for ($I_{DSS}$) & A & 0.1 \\
\hline
\texttt{IG0} & Saturation current of gate-source Schottky barrier ($I_{G0}$) & A & 0.0 \\
\hline
\texttt{K1} & Slope parameter of gate-source capacitance ($K_1$) & $1/V$ & 1.25 \\
\hline
\texttt{KE} & Dependence of power law on $V_{GS}$ , ($K_E$) & $1/V$ & 0.0 \\
\hline
\texttt{KF} & Slope parameter of gate-drain feedback capacitance ($K_F$) & $1/V$ & 1.25 \\
\hline
\texttt{KG} & Drain dependence on $V_{GS}$ in the linear region, ($K_G$) & $1/V$ & 0.0 \\
\hline
\texttt{KR} & Slope factor of intrinsic channel resistance ($K_R$) & $1/V$ & 0.0 \\
\hline
\texttt{RI} & Intrinsic channel resistance for ($R_{I}$) & $\Omega$ & 0.0 \\
\hline
\texttt{SL} & Slope of the drain characteristic in the saturated region, ($S_L$) & S & 0.15 \\
\hline
\texttt{SS} & Slope of the drain characteristic in the saturated region ($S_S$) & S & 0.0 \\
\hline
\texttt{T} & Channel transit-time delay ($\tau$) & s & 0.0 \\
\hline
\texttt{VBC} & Breakdown voltage ($V_{BC}$) & V & $10^{10}$ \\
\hline
\texttt{VP0} & Pinch-off voltage for ($V_{P0}$) & V & -2.0 \\
\hline
\end{tabular}\\[0.05in]
% Device equations start

\noindent The physical constants used in the model evaluation are
\begin{center}
\begin{tabular}{|l|l|l|}
\hline
$k$ & Boltzman's constant           &  $1.3806226\,10^{-23}$~J/K\\
$q$ & electronic charge             & $1.6021918\,10^{-19}$~C\\
\hline
\end{tabular}
\end{center}

\noindent\underline{Standard Calculations}
\begin{equation}
V_{TH} = (k T)/q
\end{equation}
$T$ is the analysis temperature
\newline
\begin{center}
\begin{tabular}{r l}
$V_{DS}$ & Intrinsic drain source voltage \\
$V_{GS}$ & Intrinsic gate source voltage \\
$V_{GD}$ & Intrinsic gate drain voltage
\end{tabular}
\end{center}
%\vspace{6mm}
%\newline
\line(1,0){425}
\newline
\noindent\textit{Device Equations:}
\newline
%\vspace{2mm}
% Current characteristics

\noindent\underline{Current Characteristics}
\begin{eqnarray}
I_{DS}&=&Area I_{DSS} \left[ 1 + S_S {{V_{DS}} \over {I_{DSS}}} \right]
         \left[ 1 - {{V_{GS}(t-\tau)} \over {V_{P0} + \gamma V_{DS}}}
         \right]^{\textstyle (E + K_E V_{GS}(t-\tau))} \nonumber\\
      & &\hspace*{1in} \times \mbox{tanh}\left[  {{S_L V_{DS}} \over
                {I_{DSS} ( 1 - K_G V_{GS}(t-\tau))}} \right]
       \\
I_{GS}&=&Area I_{G0} \left[ e^{\textstyle A_{FAG}V_{GS}} -1 \right]
         - I_{B0} \left[ e^{\textstyle -A_{FAB}(V_{GS}+ V_{BC}})  \right]
       \\
I_{GD}&=&Area I_{G0} \left[ e^{\textstyle A_{FAG}V_{GD}} -1 \right]
         - I_{B0} \left[ e^{\textstyle -A_{FAB}(V_{GD}+ V_{BC}})  \right]
       \\
R_I&=& \left\{ \begin{array}{ll}
       R_{10} ( 1 - K_R V_{GS} )/Area  & K_R V_{GS} < 1.0\\
       0                         & K_R V_{GS} \ge 1.0\\
      \end{array} \right.
\end{eqnarray}

% Capacitance equations
\noindent\underline{Capacitance}\\

$C_{LVL}$ = 1 (default) for the standard Materka-Kacprzak capacitance model described below is used.
The Materka-Kacprzak capacitances are
\begin{eqnarray}
C'_{DS}&=& C_{DS} \\
C'_{GS}&=&\left\{ \begin{array}{ll}
          \left[ C_{10}( 1 - K_1 V_{GS} )^{M_{GS}} + C_{1S} \right]
           & K_1V_{GS} < F_{CC}\\
          \left[ C_{10}( 1 - F_{CC} )^{M_{GS}} + C_{1S} \right]
           & K_1V_{GS} \ge F_{CC}\\
         \end{array} \right. \\ %}
C'_{GD}&=&\left\{ \begin{array}{ll}
         Area \left[ C_{F0}( 1 - K_1 V_1 )^{M_{GD}} \right]
           & K_1V_1 < F_{CC}\\
         Area \left[ C_{F0}( 1 - F_{CC} )^{M_{GD}} \right]
           & K_1V_1 \ge F_{CC}\\
         \end{array} \right.
\end{eqnarray}


\textit{Version:}\\
2000.09.01 \\
% Credits
\linethickness{0.5mm} \line(1,0){425}
\newline
\textit{Credits:}\\
\begin{tabular}{l l l l}
Name & Affiliation & Date & Logo \\
Carlos E. Christofferson & NC State University & Sept 2000 & \epsfxsize=1in\epsfbox{figures/logo.eps} \\
www4.ncsu.edu/\~\ cechrist & & & \\
\end{tabular}
\end{document}
